%&latex

\documentclass[10pt,landscape]{article}
%\documentclass[10pt]{article}

%
% Bring in the AMS math environment
%
\usepackage{amsmath}

%
% Bring in the common page setup
%
\usepackage{dynenv}

%
% Define vector and quaternion macros
%
\usepackage{dynmath}


%
% Bring in the hyper ref environment
%
\usepackage[colorlinks,bookmarks]{hyperref}
\hypersetup{
   pdftitle={Dynamics Package Mathematical Nomenclature},
   pdfauthor={David Hammen}}

\newcommand{\dynmath}{\texorpdfstring{\tt dynmath}{dynmath}}

\newcommand{\symhdr}[1]{%
\multicolumn{4}{l}{\rule{0pt}{3ex}\parbox[c]{0.6\textwidth}{#1}} \\
\cline{2-4} \rule{3em}{0pt} & {\bf Symbol} & {\bf Represents} & {\bf Units (MKS)}\\ \cline{2-4} \cline{2-4}}

\newcommand{\acchdr}[1]{%
\multicolumn{4}{l}{\rule{0pt}{3ex}\parbox[c]{0.6\textwidth}{#1}} \\
\cline{2-4} \rule{3em}{0pt} & {\bf Description} & {\bf Remark} & {\bf Example}\\ \cline{2-4} \cline{2-4}}

\def\purpwidth{0.35\textwidth}
\def\argswidth{0.15\textwidth}

\newcommand{\mlentry}[2]{%
  \minipage[t]{#1}{\flushleft{#2}\endflushleft}\endminipage}


\newcommand{\slinethree}{\rule{0pt}{0.2ex}&\rule{0pt}{0.2ex}&\rule{0pt}{0.2ex}&\rule{0pt}{0.2ex}}
\newcommand{\slinefour}{\rule{0pt}{1.2ex}&&&&}


\begin{document}

\title{Dynamics Package Mathematical Nomenclature}
\author{David Hammen}
\date{12/21/05}

\pdfbookmark{Title Page}{titlepage}
\maketitle

\tableofcontents

\section*{Introduction}
This note describes the preferred nomenclature for the dynamics package
documentation and describes a set of \LaTeX\ macros that implement
the nomenclature.


\pagebreak
\section{Nomenclature}
This section describes the preferred nomenclature for the dynamics package
documentation. The preferred nomenclature
\begin{itemize}
\item Uses International System (SI) units with standard SI abbreviations.
\item Represents scalars in plain math font, vectors and matrices in bold math font,
and quaternions in caligraphy font.
\item Represents vectors, matrices, and quaternions in a standard, intuitive style
with an obvious translation to variable names. For example,
$\framerelvect A x a b$ is the vector from point $a$ to point $b$ as expressed in
reference frame $A$. This is denoted as the ``arrow-separated'' format.
Alternative respresentations such as 
\dynmathcommamode $\relvect x a b$ \dynmatharrowmode (``comma-separated'')
and
\dynmathstackedmode $\relvect x a b$ \dynmatharrowmode (``stacked'')
were
discussed but discarded for various reasons. 
\item Quaternions that represent a transformation from one frame to another
are represented as unit left transformation quaternions,
consistent with the quaternion functions defined in the Trick core.
\end{itemize}

The following tables depict the preferred nomenclature.
In cases where alternative representation schemes exist,
the preferred form is listed first followed by alternatives

 \pagebreak
\subsection{Basics}\label{sec:nomen_basics}

\begin{tabular}{l||l|l|l|}
\acchdr{{\bf{Display style for vectors, matrices, and quaternions}}}
\rule{0pt}{3ex} & Scalar &
  in plain math font & $s$ \\
\rule{0pt}{1.5ex} & Vector &
  in bold math font& $\vect x$ \\
\rule{0pt}{1.5ex} & Matrix &
  in bold math font& $\mat T$ \\
\rule{0pt}{1.5ex} & Quaternion&
  in caligraphy font, uppercase & $\quat Q$ \\
\cline{2-4}
\symhdr{{\bf{Scalar Symbols}}}
\rule{0pt}{3ex} & $\alpha$ & Angular acceleration & $r/s^2$ \\
\rule{0pt}{1.5ex} & $\alpha, \beta, \theta, \phi, \psi$ & Angle & $r$ \\
\rule{0pt}{1.5ex} & $\omega$ & Angular rate & $r/s$ \\
\rule{0pt}{1.5ex} & $d, l, s$ & Distance or length & $m$ \\
\rule{0pt}{1.5ex} & $m$ & Mass & $kg$ \\
\rule{0pt}{1.5ex} & $v, s$ & Speed & $m/s$ \\
\cline{2-4}
\symhdr{{\bf{Vector Symbols}}}
\rule{0pt}{3ex} & $\vect \alpha$ & Angular acceleration & $r/s^2$ \\
\rule{0pt}{1.5ex} & $\vect \tau$ & Torque & $Nm$ \\
\rule{0pt}{1.5ex} & $\vect \omega$ & Angular velocity & $r/s$ \\
\rule{0pt}{1.5ex} & $\vect a$ & Acceleration & $m/s^2$ \\
\rule{0pt}{1.5ex} & $\vect L$ & Angular momentum & $Nms \; (kg \, m^2/s)$ \\
\rule{0pt}{1.5ex} & $\vect v$ & Velocity & $m/s$ \\
\rule{0pt}{1.5ex} & $\vect x$ & Position & $m$ \\
\cline{2-4}
\symhdr{{\bf{Matrix Symbols}}}
\rule{0pt}{3ex} & $\mat I$ & Inertia tensor & $kg \, m^2$ \\
\rule{0pt}{1.5ex} & $\mat T$ & Transformation matrix & $--$ \\
\cline{2-4}
\symhdr{{\bf{Quaternion Symbols}}}
\rule{0pt}{3ex} & $\quat Q$ & Left transformation quaternion & $--$ \\
\cline{2-4}
\end{tabular}

\pagebreak

\subsection{Adornments}\label{sec:nomen_addorn}

\begin{tabular}{l||l|l|l|}
\acchdr{{\bf{Vectors}}}
\rule{0pt}{3ex} & Vector from origin to $b$ &Subscripted $b$& $\absvect x b$ \\
%\cline{2-4}
\rule{0pt}{3ex} & Vector from $a$ to $b$ &
 Arrow, comma, stacked formats &
\makebox[1.5cm]{$\relvect x a b$}
\makebox[1.5cm]{\dynmathcommamode $\relvect x a b$ \dynmatharrowmode}
\makebox[1.5cm]{\dynmathstackedmode $\relvect x a b$ \dynmatharrowmode} \\
%\cline{2-4}
\rule{0pt}{3ex} & Vector expressed in frame $A$ &&
\makebox[1.5cm]{$\framevect A x$}
\makebox[1.5cm]{\dynmathcommamode $\framevect A x$ \dynmatharrowmode}
\makebox[1.5cm]{\dynmathstackedmode $\framevect A x$ \dynmatharrowmode} \\
%\cline{2-4}
\rule{0pt}{3ex} & Vector from $a$ to $b$ expressed in frame $A$ &&
\makebox[1.5cm]{$\framerelvect A x a b$}
\makebox[1.5cm]{\dynmathcommamode $\framerelvect A x a b$ \dynmatharrowmode}
\makebox[1.5cm]{\dynmathstackedmode $\framerelvect A x a b$ \dynmatharrowmode} \\
%\cline{2-4}
\rule{0pt}{3ex} & Time derivative of above, observer in frame $A$&&
\makebox[1.5cm]{$\framerelvdot A x a b$}
\makebox[1.5cm]{\dynmathcommamode $\framerelvdot A x a b$ \dynmatharrowmode}
\makebox[1.5cm]{\dynmathstackedmode $\framerelvdot A x a b$ \dynmatharrowmode} \\
\rule{0pt}{3ex} & Vector time derivative, observer in frame $A$&
  Dot format & $\framedot A {\vect x}$ \\
\rule{0pt}{3ex} & &
  $or$ d/dt format & $\frac{d}{dt_A}{\vect x}$ \\
\cline{2-4}
\acchdr{{\bf{Matrices}}}
\rule{0pt}{3ex} & Transformation from $A$ to $B$ &
 Arrow, comma, stacked formats &
\makebox[1.5cm]{$\tmat A B$}
\makebox[1.5cm]{\dynmathcommamode $\tmat A B$ \dynmatharrowmode}
\makebox[1.5cm]{\dynmathstackedmode $\tmat A B$ \dynmatharrowmode} \\
%\cline{2-4}
\rule{0pt}{3ex} & Matrix product &
  No operator & $\MxM{\tmat B C}{\tmat A B}$ \\
%\cline{2-4}
\rule{0pt}{3ex} & Single matrix transpose &
  Superscript $\top$ & $\matT T$ \\
%\cline{2-4}
\rule{0pt}{3ex} & Matrix product transpose &
  Superscript $\top$ & $\left(\MxM{\tmat B C}{\tmat A B}\right){^\top}$ \\
\cline{2-4}
\acchdr{{\bf{Quaternions}}}
\rule{0pt}{3ex} & Quaternion from $A$ to $B$ &
 Arrow, comma, stacked formats &
\makebox[1.5cm]{$\tquat A B$}
\makebox[1.5cm]{\dynmathcommamode $\tquat A B$ \dynmatharrowmode}
\makebox[1.5cm]{\dynmathstackedmode $\tquat A B$ \dynmatharrowmode} \\
%\cline{2-4}
\rule{0pt}{3ex} & Quaternion product &
  No operator & $\QxQ{\tquat B C}{\tquat A B}$ \\
%\cline{2-4}
\rule{0pt}{3ex} & Single quaternion conjugate &
  Superscript $\star$ & $\quatconj Q$ \\
%\cline{2-4}
\rule{0pt}{3ex} & Quaternion product conjugate &
  Superscript $\star $ & $\quatconjlr{\QxQ{\tquat B C}{\tquat A B}}$ \\
%\cline{2-4}
\rule{0pt}{3ex} & Quaternion components &
  Scalar $+$ vector $or$ four-vector&
  $\quatsv{q_s}{\vect{q_v}} \;or\; \bmatrix q_s \\ q_x \\ q_y \\ q_z\endbmatrix$ \\
\cline{2-4}
\end{tabular}

\pagebreak
\section{{\dynmath}\ Macros}

This section provides a brief description of the macros defined in
\verb|dynmath.sty| in the form of tables that describe the macros,
their arguments, sample usage of the macros, and the displayed math that results.

Note: All of the macros defined in \verb|dynmath.sty| assume {\LaTeX} is in math mode.

\pagebreak
\subsection{Vector Macros}

\begin{tabular}{||l|l|l|l|l|} \hline
{\bf Command} & {\bf Purpose} & {\bf Arguments} & {\bf Example} & {\bf Display} \\ \hline \hline
\slinefour \\
\verb|\vect| &
  \mlentry{\purpwidth}{Display a symbol that represents a vector (typically a lowercase letter)} &
  \mlentry{\argswidth}{Symbol} &
  \verb|\vect{x}| & $\vect{x}$ \\ \slinefour \\
\verb|\vhat| &
  \mlentry{\purpwidth}{Display a symbol that represents a unit vector} &
  \mlentry{\argswidth}{Symbol} &
  \verb|\vhat{x}| & $\vhat{x}$ \\ \slinefour \\
\verb|\vdot| &
  \mlentry{\purpwidth}{Time derivative of a vector} &
  \mlentry{\argswidth}{Symbol} &
  \verb|\vdot{x}| & $\vdot{x}$ \\ \slinefour \\
\verb|\framevect| &
  \mlentry{\purpwidth}{Vector represented in a specific frame} &
  \mlentry{\argswidth}{1. Frame\\ 2. Vector} &
  \verb|\framevect B x| & $\framevect B x$  \\ \slinefour \\
  \verb|\framevdot| &
  \mlentry{\purpwidth}{Time derivative of a vector represented in a specific frame} &
  \mlentry{\argswidth}{1. Frame\\ 2. Vector} &
  \verb|\framevdot B x| & $\framevdot B x$  \\ \slinefour \\
\verb|\relvect| &
  \mlentry{\purpwidth}{Vector between two items} &
  \mlentry{\argswidth}{1. Vector\\ 2. Source\\ 3. Destination} &
  \verb|\relvect x a b| & $\relvect x a b$  \\ \slinefour \\
\verb|\relvdot| &
  \mlentry{\purpwidth}{Time derivative of a vector between two items} &
  \mlentry{\argswidth}{1. Vector\\ 2. Source\\ 3. Destination} &
  \verb|\relvdot x a b| & $\relvdot x a b$  \\ \slinefour \\
\verb|\framerelvect| &
  \mlentry{\purpwidth}{Vector between two items} &
  \mlentry{\argswidth}{1. Frame\\ 2. Vector\\ 3. Source\\ 4. Destination} &
  \verb|\framerelvect B x a b| & $\framerelvect B x a b$  \\ \slinefour \\
\verb|\framerelvdot| &
  \mlentry{\purpwidth}{Time derivative of a vector between two items} &
  \mlentry{\argswidth}{1. Frame\\ 2. Vector\\ 3. Source\\ 4. Destination} &
  \verb|\framerelvdot B x a b| & $\framerelvdot B x a b$  \\ \slinefour \\
\verb|\vectxyz| &
  \mlentry{\purpwidth}{Construct a vector from its components} &
  \mlentry{\argswidth}{Three components} &
  \verb|\vectxyz x y z| & $\vectxyz x y z$  \\ \slinefour \\
\hline
\end{tabular}

\pagebreak

\subsection{Quaternion Macros}
\
\begin{tabular}{||l|l|l|l|l|} \hline
{\bf Command} & {\bf Purpose} & {\bf Arguments} & {\bf Example} & {\bf Display} \\ \hline \hline
\slinefour \\
\verb|\quat| &
  \mlentry{\purpwidth}{Display a symbol that represents a quaternion (typically Q)} &
  \mlentry{\argswidth}{Symbol} &
  \verb|\quat{Q}| & $\quat{Q}$ \\ \slinefour \\
\verb|\qdot| &
  \mlentry{\purpwidth}{Time derivative of a quaternion} &
  \mlentry{\argswidth}{Symbol} &
  \verb|\qdot{Q}| & $\qdot{Q}$ \\ \slinefour \\
\verb|\quatconj| &
  \mlentry{\purpwidth}{Conjugate of a quaternion} &
  \mlentry{\argswidth}{Symbol} &
  \verb|\quatconj{Q}| & $\quatconj{Q}$ \\ \slinefour \\
\verb|\quatconjdot| &
  \mlentry{\purpwidth}{Time derivative of the conjugate of a quaternion} &
  \mlentry{\argswidth}{Symbol} &
  \verb|\quatconjdot{Q}| & $\quatconjdot{Q}$ \\ \slinefour \\
\verb|\quatsv| &
  \mlentry{\purpwidth}{Construct a quaternion from a scalar and $3$-vector} &
  \mlentry{\argswidth}{1. Scalar\\ 2. Vector} &
  \minipage[t]{0pt}{\verb|\quatsv| \\ \verb|  {q_s}| \\ \verb|  {\vect{q_v}}|}\endminipage &
  \raisebox{-1.5ex}{$\quatsv{q_s}{\vect{q_v}}$} \\ \slinefour \\
\verb|\quattrot| &
  \mlentry{\purpwidth}{Construct a transformation quaternion from an angle and unit vector} &
  \mlentry{\argswidth}{1. Angle\\ 2. Unit vector} &
  \minipage[t]{0pt}{\verb|\quattrot| \\ \verb|  \theta| \\ \verb|  {\vhat u}|}\endminipage &
  \raisebox{-2.5ex}{$\quattrot{\theta}{\vhat u}$} \\ \slinefour \\
\verb|\conjop| &
  \mlentry{\purpwidth}{Conjugate operator} &
  \mlentry{\argswidth}{None} &
  \verb|\conjop \quat Q| & $\conjop \quat Q$ \\ \slinefour \\
\verb|\scalarpart| &
  \mlentry{\purpwidth}{Scalar part operator} &
  \mlentry{\argswidth}{None} &
  \verb|\scalarpart \quat Q| & $\scalarpart \quat Q$ \\ \slinefour \\
\verb|\vectorpart| &
  \mlentry{\purpwidth}{Vector part operator} &
  \mlentry{\argswidth}{None} &
  \verb|\vectorpart \quat Q| & $\vectorpart \quat Q$ \\ \slinefour \\
\verb|\QBI| &
  \mlentry{\purpwidth}{Inertial-to-body quaternion} &
  \mlentry{\argswidth}{None} &
  \verb|\QBI| & $\QBI$ \\ \slinefour \\
\hline
\end{tabular}


\pagebreak

\subsection{Matrix Macros}

\begin{tabular}{||l|l|l|l|l|} \hline
{\bf Command} & {\bf Purpose} & {\bf Arguments} & {\bf Example} & {\bf Display} \\ \hline \hline
\slinefour \\
\verb|\mat| &
  \mlentry{\purpwidth}{Display a symbol that represents a matrix (typically an uppercase letter)} &
  \mlentry{\argswidth}{Symbol} &
  \verb|\mat{T}| & $\mat{T}$ \\ \slinefour \\
\verb|\framemat| &
  \mlentry{\purpwidth}{Matrix represented in some reference frame} &
  \mlentry{\argswidth}{1. Frame\\ 2. Matrix} &
  \verb|\framemat B \inertia| & $\framemat B \inertia$ \\ \slinefour \\
\verb|\diagmatrix| &
  \mlentry{\purpwidth}{$3\times3$ diagonal matrix} &
  \mlentry{\argswidth}{Three diagonal elements} &
  \verb|\diagmatrix 1 2 3| & $\diagmatrix 1 2 3$ \\ \slinefour \\
\verb|\identmatrix| &
  \mlentry{\purpwidth}{$3\times3$ identity matrix} &
  \mlentry{\argswidth}{None} &
  \verb|\identmatrix| & $\identmatrix$ \\ \slinefour \\
\verb|\inertia| &
  \mlentry{\purpwidth}{Inertia matrix} &
  \mlentry{\argswidth}{None} &
  \verb|\inertia| & $\inertia$ \\ \slinefour \\
\hline
\end{tabular}

\pagebreak

\subsection{Multiplication Macros}

The preferred nomenclature for depicting the product of two (or more) composite objects
is ``implicit multiplication'':  The operands are written with no intervening operator.
However,
a small amount of white space between the operands helps to distinguish the operands.
The multiplication macros provide a default amount of white space between operands.

\begin{tabular}{||l|l|l|l|l|} \hline
{\bf Command} & {\bf Purpose} & {\bf Arguments} & {\bf Example} & {\bf Display} \\ \hline \hline
\slinefour \\
\verb|\MxM| &
  \mlentry{\purpwidth}{Product of two matrices} &
  \mlentry{\argswidth}{Matrix 1\\ Matrix 2} &
  \minipage[t]{2.5cm}{\verb|\MxM| \\ %
    \verb| {\tmat B C}| \\ %
    \verb| {\tmat A B}|}\endminipage &
  $\MxM{\tmat B C}{\tmat A B}$ \\[8ex]
\verb|\QxVxQ | &
  \mlentry{\purpwidth}{Product of a quaternion, a vector, and a quaternion} &
  \mlentry{\argswidth}{Quaternion 1\\ Vector\\ Quaternion 2} &
  \minipage[t]{2.5cm}{\verb|\QxVxQ| \\ %
    \verb| {\quat Q}| \\ %
    \verb| {\vect x}| \\ %
    \verb| {\quatconj Q}|}\endminipage &
    \raisebox{-1.5ex}{$\QxVxQ{\quat Q}{\vect x}{\quatconj Q}$} \\ \slinefour \\
\hline
\end{tabular}

Similar macros are defined for
\begin{itemize}
\item the product of three matrices (\verb|MxMxM|)
\item the product of a matrix and vector (\verb|MxV|)
\item the product of two or three quaternions (\verb|QxQ|, \verb|QxQxQ|)
\item the product of a quaternion and a vector (\verb|QxV|, \verb|VxQ|)
\end{itemize}

The multiplication macros take an optional argument via which an explicit multiplication operator can be specified. For example,

\begin{tabular}{||l|l|l|l|l|} \hline
{\bf Command} & {\bf Purpose} & {\bf Option} & {\bf Example} & {\bf Display} \\ \hline \hline
\slinefour \\
\verb|\QxQ| &
  \mlentry{\purpwidth}{Implicit product of two quaternions} &
  \mlentry{\argswidth}{None} &
  \minipage[t]{2.5cm}{\verb|\QxQ| \\ %
    \verb| {\quat{Q}_1}| \\ %
    \verb| {\quat{Q}_2}|}\endminipage &
  $\QxQ{{\quat Q}_1}{{\quat Q}_2}$ \\[8ex]
\verb|\QxQ| &
  \mlentry{\purpwidth}{Explicit product of two quaternions} &
  \mlentry{\argswidth}{\ttfamily \textbackslash circ} &
 \minipage[t]{2.5cm}{\verb|\QxQ[\circ]| \\ %
    \verb| {{\quat Q}_1}| \\ %
    \verb| {{\quat Q}_2}|}\endminipage &
  $\QxQ[\circ]{{\quat Q}_1}{{\quat Q}_2}$ \\[8ex]
\hline
\end{tabular}


\pagebreak

\subsection{Miscellaneous Macros}

\begin{tabular}{||l|l|l|l|l|} \hline
{\bf Command} & {\bf Purpose} & {\bf Arguments} & {\bf Example} & {\bf Display} \\ \hline \hline
\slinefour \\
\verb|\abs| &
  \mlentry{\purpwidth}{Absolute value} &
  \mlentry{\argswidth}{Scalar expression} &
  \verb|\abs{x}| & $\abs{x}$ \\ \slinefour \\
\verb|\norm| &
  \mlentry{\purpwidth}{Euclidian norm} &
  \mlentry{\argswidth}{Vector or quaternion expression} &
  \verb|\norm{\vect x}| & $\norm{\vect x}$ \\ \slinefour \\
\verb|\framedot| &
  \mlentry{\purpwidth}{Frame-dependent time derivative} &
  \mlentry{\argswidth}{1. Frame\\ 2. Expression} &
  \verb|\framedot B {\vect x}| & $\framedot B {\vect x}$ \\ \slinefour \\
\hline
\end{tabular}


\end{document}
