%%%%%%%%%%%%%%%%%%%%%%%%%%%%%%%%%%%%%%%%%%%%%%%%%%%%%%%%%%%%%%%%%%%%%%%%%
%
% File: Model.tex
%
% Purpose: Top level document for Model.  Should not need to be edited.
%
%%%%%%%%%%%%%%%%%%%%%%%%%%%%%%%%%%%%%%%%%%%%%%%%%%%%%%%%%%%%%%%%%%%%%%%%%

\newcommand\documentHistory{
{\bf Author} & {\bf Date} & {\bf Description} \\ \hline \hline
\ModelAuthor & \ModelAuthDate & Initial Version \\ \hline
}


\documentclass[twoside,11pt,titlepage]{report}

%
% Bring in the common page setup
%
\usepackage{dynenv}

%
% Bring in the model-specific commands
%
\usepackage{DYNCOMP}

%
% Bring in the graphics environment
%
\usepackage{graphicx}
\usepackage{float}

%
% Bring in the hyper ref environment
%
\usepackage[colorlinks,plainpages=false]{hyperref}
%  keywords for pdfkeywords are separated by commas
\hypersetup{
   pdftitle={\DYNCOMPDesc},
   pdfauthor={\ModelAuthor},
   pdfkeywords={\ModelKeywords},
   pdfsubject={\DYNCOMPDesc}}

\begin{document}

%%%%%%%%%%%%%%%%%%%%%%%%%%%%%%%%%%%
% Front matter
%%%%%%%%%%%%%%%%%%%%%%%%%%%%%%%%%%%
\pagenumbering{roman}

\docid{verif/SIM\_dyncomp}
\docrev{3.0}
\date{\RELEASEMONTH\ \RELEASEYEAR}
\modelname{\DYNCOMPDesc}
\doctype{}
\author{\ModelAuthor}
\managers{
  Robert O. Shelton \\ Project Manager \\
  Michael T. Red \\ Simulation and Graphics Branch Chief \\
  R. Matt Ondler \\ Software, Robotics, and Simulation Division Chief}
\pdfbookmark{Title Page}{titlepage}
\makeDynenvTitlepage

\pdfbookmark{Abstract}{abstract}
%%%%%%%%%%%%%%%%%%%%%%%%%%%%%%%%%%%%%%%%%%%%%%%%%%%%%%%%%%%%%%%%%%%%%%%%
%
% Purpose: dyncomp abstract
%
%
%%%%%%%%%%%%%%%%%%%%%%%%%%%%%%%%%%%%%%%%%%%%%%%%%%%%%%%%%%%%%%%%%%%%%%%%%

\begin{abstract}
Regression testing of the JSC Engineering Dynamics (JEOD) software is required in order
to assure the user and any clients regarding the validity of the results of the simulation.
There are many levels of software verification; the focus of \DYNCOMPDesc\ is the
establishment of set of tests using the numerical output of the simulation compared
against other simulation data and simple analytic unit tests. The goal of this
subset of validation testing of JEOD is a demonstration that the software meets the
purposes and requirements of the over all project when new versions of the software
is issued.
\end{abstract}


\pdfbookmark{Contents}{contents}
\tableofcontents
\vfill

\pagebreak

%%%%%%%%%%%%%%%%%%%%%%%%%%%%%%%%%%%
% Main Document Body
%%%%%%%%%%%%%%%%%%%%%%%%%%%%%%%%%%%
\pagenumbering{arabic}
\setcounter{chapter}{0}

%----------------------------------
\chapter{Introduction}\hyperdef{part}{intro}{}\label{ch:intro}
%----------------------------------
\section{Introduction}
After any revision of the JEOD software there is a process of regression
testing. Each version of JEOD is verified and validated in two ways.  All releases
are tested for accuracy against trajectories of actual vehicles.  They are also
regression tested by comparing the results of individual model vetification
simulations, and by running combined simulations such as \DYNCOMPDesc.


The present version, as it becomes available, is tested against the last version
and checked for differences. This means all code,test runs, and documents are
checked against the previous JEOD as a truth model. Any differences between code,
documents and test runs for models that did not change have to be explained or
resolved.
Once resolution has been achieved, the data files for regression testing are
archived in
house in the JEOD directories. Any change -- whether it be a code change, model
addition, document changed or test run addition -- is documented in the top
level JEOD documents.
These documents are made
available, along with version notes at the time of JEOD official release.

The documentation for SIM\_dyncomp for this release is contained in the
following document:
A Process for Comparing Dynamics of Distributed Space Systems Simulations\cite{siw}.
It is hyperlinked here:
\DYNCOMLINK

The validation of these simulations are explained in the paper.
Note the validation data consisted of independent simulation to simulation
comparisons
and simple analytical models what could be computed independent of JEOD.

\section{Document History}
%%% Status of this and only this document.  Any date should be relevant to when
%%% this document was last updated and mention the reason (release, bug fix, etc.)
%%% Mention previous history aka JEOD 1.4-5 heritage in this section.
%%% Mention that JEOD.pdf is the parent document.

\begin{tabular}{||l|l|l|l|} \hline
\DocumentChangeHistory
\end{tabular}


\section {SIM\_dyncomp S\_define}
For reference the SIM\_dyncomp S\_define is included here.  It relies
extensively on the standard JEOD S-modules:

\subsection{Preamble}

\begin{verbatim}
//===========================TRICK HEADER=====================
// PURPOSE:
//=============================================================================
// This simulation provides a reference implementation that can be used for
// comparison to other orbital dynamics implementations.  It is intended to
// provide a reference set for dynamics comparison.
//
// This simulation models a single vehicle in orbit around the Earth.  There
// are many adjustable configuration parameters that can be used to test out
// a suite of test cases with specific behavior.  These test cases form the
// basis for simulation to simulation comparison.
//
//          sys - Trick runtime executive and data recording routines
//         time - Representations of different clocks used in the sim
//     dynamics - Orbital dynamics
//          env - Environment: ephemeris, gravity
//          sun - Sun planetary model
//         moon - Moon planetary model
//        earth - Earth planetary model
//       vehicle - Space vehicle dynamics model
//
//=============================================================================
\end{verbatim}

\subsection{Configuration}
\begin{itemize}
\item Define the job-calling intervals

\begin{verbatim}
// Define job calling intervals
#define LOW_RATE_ENV  60.00    // Low-rate environment update interval
#define DYNAMICS       0.03125 // Vehicle and plantary dynamics interval (32Hz)
\end{verbatim}

\item Include the standard job-priority definitions

\begin{verbatim}
// Define the phase initialization priorities.
#include "default_priority_settings.sm"
\end{verbatim}

\item Bring in the trick-system and jeod-system top-level modules

\begin{verbatim}
 // Include the default system classes:
#include "sim_objects/default_trick_sys.sm"
#include "jeod_sys.sm"
\end{verbatim}

\end{itemize}

\subsection{Time}
Use the standard jeod\_time.sm sim-module; in addition to the default TAI
clock, add UT1, UTC, TT, and GMST.  Also specify that the clocks should have
their calendar representations computed out.

\begin{verbatim}
// Set up desired time types and include the JEOD time S_module
#define TIME_MODEL_UT1
#define TIME_MODEL_UTC
#define TIME_MODEL_TT
#define TIME_MODEL_GMST
#define TIME_CALENDAR_UPDATE_INTERVAL  DYNAMICS
#include "jeod_time.sm"
\end{verbatim}

\subsection{Dynamics}
Use the standard dynamics sim-object.
\begin{verbatim}
#include "dynamics.sm"
\end{verbatim}

\subsection{Environment}
Use the standard environment sim-object, with sun, moon and earth planetary
bodies.

\begin{verbatim}
#include "environment.sm"
#include "sun_basic.sm"
#include "moon_basic.sm"
\end{verbatim}
For earth, use the full earth model with a spherical harmonic gravity, MET
atmosphere, and RNP.  For legacy reasons, we use the (older) GEMT1 gravity
model rather than the default GGM02C model.
\begin{verbatim}
#define JEOD_OVERRIDE_GGM02C_WITH_GEMT1
#include "earth_GGM02C_MET_RNP.sm"
\end{verbatim}


\subsection{Vehicle}
We want some specific capabilities for our vehicle, so we will build a new
vehicle sim-object based off one of the standard S-modules.

\subsubsection{Model Headers}
\begin{verbatim}
/*****************************************************************************
Vehicle Sim Object
Purpose:(Provides the vehicle object)
*****************************************************************************/
// Include headers for classes that this class contains:
##include "dynamics/body_action/include/dyn_body_init_lvlh_rot_state.hh"
##include "dynamics/derived_state/include/euler_derived_state.hh"
##include "dynamics/derived_state/include/lvlh_derived_state.hh"
##include "dynamics/derived_state/include/orb_elem_derived_state.hh"
##include "dynamics/dyn_body/include/force.hh"
##include "dynamics/dyn_body/include/torque.hh"
##include "interactions/gravity_torque/include/gravity_torque.hh"
##include "environment/atmosphere/include/atmosphere.hh"
##include "interactions/aerodynamics/include/aero_drag.hh"

// Include default data classes:
##include "interactions/aerodynamics/data/include/aero_model.hh"
\end{verbatim}

\subsubsection{Inheritance}
Note - we use the sim-object version found in Base.  This is just the
definition, without the instantiation.
\begin{verbatim}
// include the base class defintion.
#include "Base/vehicle_atmosphere.sm"
class VehicleSimObject : public VehicleAtmSimObject
{
\end{verbatim}

\subsubsection{Content}
Note - this is the content in addition to the content of the
VehicleAtmSimObject.
\begin{verbatim}
 public:
  DynBodyInitLvlhRotState lvlh_init;
  EulerDerivedState     euler;
  LvlhDerivedState      lvlh;
  EulerDerivedState     lvlh_euler;
  OrbElemDerivedState   orb_elem;
  Force  force_extern;
  Torque torque_extern;
  GravityTorque grav_torque;
  SphericalHarmonicsGravityControls  earth_grav_control;
  SphericalHarmonicsGravityControls  moon_grav_control;
  SphericalHarmonicsGravityControls  sun_grav_control;
  AerodynamicDrag  aero_drag;

  // Instances for default data:
  AerodynamicDrag_aero_model_default_data    aero_drag_default_data;

  //Constructor
  VehicleSimObject( DynManager    & dyn_manager_,
                    METAtmosphere & atmos_,
                    WindVelocity  & wind_)
     :
     VehicleAtmSimObject( dyn_manager_, atmos_, wind_)

  {
    //
    //Default data jobs
    //
    ("default_data") aero_drag_default_data.initialize ( &aero_drag );

    //
    // Initialization jobs
    //
    P_DYN  ("initialization") euler.initialize( dyn_body,
                                                dyn_manager );
    P_DYN  ("initialization") lvlh.initialize( dyn_body,
                                               dyn_manager );
    P_DYN  ("initialization") lvlh_euler.initialize( lvlh.lvlh_frame,
                                                     dyn_body,
                                                     dyn_manager );
    P_DYN  ("initialization") orb_elem.initialize( dyn_body,
                                                   dyn_manager );
    ("initialization") euler.update( );
    ("initialization") pfix.update( );
    ("initialization") lvlh.update( );
    ("initialization") lvlh_euler.update( );
    ("initialization") orb_elem.update( );
    ("initialization") grav_torque.initialize( dyn_body );
    //
    // Environment class jobs
    //
    (DYNAMICS, "environment") euler.update( );
    (DYNAMICS, "environment") lvlh.update( );
    (DYNAMICS, "environment") lvlh_euler.update( );
    (DYNAMICS, "environment") orb_elem.update( );
    //
    // Derivative class jobs
    //
    P_BODY   ("derivative") aero_drag.aero_drag(
          dyn_body.composite_body.state.trans.velocity,
          &atmos_state,
          dyn_body.structure.state.rot.T_parent_this,
          dyn_body.composite_properties.mass,
          dyn_body.composite_properties.position);
    P_PGRAV   ("derivative") grav_torque.update();
  }

 private:
  VehicleSimObject (const VehicleSimObject&);
  VehicleSimObject & operator = (const VehicleSimObject&);

};
\end{verbatim}

\subsubsection{Instantiation}
\begin{verbatim}
VehicleSimObject vehicle ( dynamics.dyn_manager,
                           earth.atmos,
                           earth.wind_velocity);
\end{verbatim}

\subsection{Collect statements}
Collect forces and torques
\begin{verbatim}
vcollect vehicle.dyn_body.collect.collect_effector_forc CollectForce::create {
   vehicle.force_extern
};
vcollect vehicle.dyn_body.collect.collect_environ_forc CollectForce::create {
};
vcollect vehicle.dyn_body.collect.collect_no_xmit_forc CollectForce::create {
   vehicle.aero_drag.aero_force
};
vcollect vehicle.dyn_body.collect.collect_effector_torq CollectTorque::create {
   vehicle.torque_extern
};
vcollect vehicle.dyn_body.collect.collect_environ_torq CollectTorque::create {
};
vcollect vehicle.dyn_body.collect.collect_no_xmit_torq CollectTorque::create {
   vehicle.aero_drag.aero_torque,
   vehicle.grav_torque.torque
};
\end{verbatim}

\subsection{Integration Loop}
\begin{verbatim}
IntegLoop sim_integ_loop  (DYNAMICS) dynamics, earth, vehicle;
\end{verbatim}

%%%%%%%%%%%%%%%%%%%%%%%%%%%%%%%%%%%%%%%%%%%%%%%%%%%%%%%%%%%%%%%%%%%%%%%%%
% Bibliography
%%%%%%%%%%%%%%%%%%%%%%%%%%%%%%%%%%%%%%%%%%%%%%%%%%%%%%%%%%%%%%%%%%%%%%%%%
\newpage
\pdfbookmark{Bibliography}{bibliography}
\bibliography{dynenv,DYNCOMP}
\bibliographystyle{plain}

%\pagebreak
%\appendix

\end{document}
