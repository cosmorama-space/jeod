\setcounter{chapter}{0}

%----------------------------------
\chapter{Introduction}\hyperdef{part}{intro}{}\label{ch:intro}
%----------------------------------

\section{Model Description}
%%% Incorporate the intro paragraph that used to begin this Chapter here.
%%% This is location of the true introduction where you explain why this 
%%% model exists.
%%% Identify the Model context within JEOD.

The JEOD \LvlhFrameDesc\ enables the definition of a reference frame with
respect to the position and velocity vectors of an on-orbit object and
automatically handles the updates to that frame. This model is envisioned
primarily to be used in tracking the relative motion of one or more
orbital vehicles with respect to another. However, making the \LvlhFrameDesc\
distinct from the standard JEOD reference frames model allows the
\LvlhFrameDesc\ to be anchored to (and therefore defined by the motion of)
objects that are not DynBodies if so desired.

When used in concert with the LvlhRelativeDerivedState derived
class introducted with JEOD 3.2, the \LvlhFrameDesc\ can be used to track
relative motion in either rectilinear or curvilinear coordinates. However,
the \LvlhFrameDesc\ simply defines the LVLH reference frame itself.


\section{Document History}
%%% Status of this and only this document.  Any date should be relevant to when 
%%% this document was last updated and mention the reason (release, bug fix, etc.)
%%% Mention previous history aka JEOD 1.4-5 heritage in this section.
%%% Mention that JEOD.pdf is the parent document.

\begin{tabular}{||l|l|l|l|} \hline
\DocumentChangeHistory
\end{tabular}

\section{Document Organization}
This document is formatted in accordance with the 
NASA Software Engineering Requirements Standard~\cite{NASA:SWE} 
and is organized into the following chapters:

\begin{description}
%% longer chapter descriptions, more information.

\item[Chapter 1: Introduction] - 
This introduction contains three sections: description of model, document history, and organization.  
The first section provides the introduction to the \LvlhFrameDesc\ and its reason 
for existence.  It also contains a brief description of the interconnections with other models, and 
references to any supporting documents.  The second section displays the history of this document which includes
author, date, and reason for each revision; it also lists the document that is parent to this one.  The final
section contains a description of the how the document is organized.

\item[Chapter 2: Product Requirements] - 
Describes requirements for the \LvlhFrameDesc.

\item[Chapter 3: Product Specification] - 
Describes the underlying theory, architecture, and design of the \LvlhFrameDesc\ in detail.  It is organized into
three sections: Conceptual Design, Mathematical Formulations, and Detailed Design.

\item[Chapter 4: User Guide] - 
Describes how to use the \LvlhFrameDesc\ in a Trick simulation.  It contains information that will be useful to
both users and developers of simulations. The \LvlhFrameDesc\ is not intended to be extended, so there is no
discussion of how to do so.

\item[Chapter 5: Inspections, Tests, and Metrics] -  
The inspections, tests, and metrics describes the procedures and results that demonstrate the satisfaction of the
requirements for the \LvlhFrameDesc.

\end{description}


%----------------------------------
\chapter{Product Requirements}\hyperdef{part}{reqt}{}\label{ch:reqt}
%----------------------------------
This model shall meet the JEOD project requirements specified in the 
\hyperref{file:\JEODHOME/docs/JEOD.pdf}{part1}{reqt}{JEOD} top-level document.

%%% Format for the model Requirements is open.  It should include requirements for this model 
%%% only and use requirement tags like the one below.
%\requirement{...}
%\label{reqt:...}
%\begin{description}
%  \item[...]\ \newline
%    The documentation for the model shall include
%
%    \subrequirement{}
%    \label{reqt:...}
%      Software requirements specification.
%      
%    ...
%   
%  \item[title]\ \newline
%    text
%
%  ...
%
%\end{description}

\requirement{Project Requirements}
\label{reqt:toplevel}
\begin{description}
\item[Requirement:]\ \newline
  This model shall meet the JEOD project-wide requirements specified in
  the \JEODid\ Top-Level Document.

\item[Rationale:]\ \newline
  This is a project-wide requirement.

\item[Verification:]\ \newline
  Inspection
\end{description}


\requirement{Create and Manage Required Connections}
\label{reqt:lvlhmanage}
\begin{description}
\item[Requirement:]\ \newline
  The \LvlhFrameDesc\ shall create and manage all the connections and
  data structures necessary to enable successful model operation.

\item[Rationale:]\ \newline
  The model must be able to manage itself in order to function.

\item[Verification:]\ \newline
  Inspection, test
\end{description}


\requirement{Define LVLH Frame}
\label{reqt:lvlhdefine}
\begin{description}
\item[Requirement:]\ \newline
  The \LvlhFrameDesc\ shall correctly calculate the rotational states of
  its LVLH reference frame with respect to the user-specified planetary
  inertial frame when requested.

\item[Rationale:]\ \newline
  This is the purpose of the \LvlhFrameDesc.

\item[Verification:]\ \newline
Test
\end{description}


%----------------------------------
\chapter{Product Specification}\hyperdef{part}{spec}{}\label{ch:spec}
%----------------------------------

\section{Conceptual Design}
The \LvlhFrameDesc\ is a relatively simple model, whose main function is to
calculate the translational and rotation states of a standard JEOD reference
frame with respect to a designated planet-centered inertial reference frame.
However, it exists as a standalone model because the calculations that
govern the orientation of a local-vertical, local-horizontal frame with
respect to a planet are unique from other frames, yet common enough in
spaceflight applications to merit them being encapsulated as their own model.

As implemented in JEOD, an LVLH reference frame is defined using the position
and velocity vectors of the body or object with which it is associated. The
vectors used are measured with respect to, and expressed in, a specified
planet-centered inertial reference frame (does not have to be the frame in
which the object is being integrated). The position vector defines the
\textit{negative} Z-axis, and the angular momentum vector (position cross
velocity) defines the \textit{negative} Y-axis. Consistent with the
right-hand rule, the positive X-axis results from the vector cross of +Y and
+Z to complete the orthogonal set.


\section{Mathematical Formulations}

The only calculations performed by the \LvlhFrameDesc\ are the ones used to
determine the orientation and angular velocity of the LVLH frame with respect
to the specified planet-centered inertial reference frame. It is not necessary
to calculate its position or translational velocity, since those are by definition
identical to those of the object to which it is anchored.

The three vectors that define LVLH are as follows:
\begin{itemize}
 \item Opposite to the radial vector from planet center ($\hat k = -\hat r$)
 \item Opposite to the angular momentum vector ($\hat j = -\hat r \times \hat v$)
 \item Completing the right handed coordinate system ($\hat i = \hat j \times \hat k$)
\end{itemize}

The $\hat i$ axis also represents the projection of the velocity vector onto
the instantaneous horizontal plane. For circular orbits, the $\hat i$ axis is
tangent to the vehicle orbit, but that is a characteristic rather than a definition.

Suppose the position and velocity of an object are known in a given planet-
centered inertial reference frame as:
\begin{equation*}
 \vec r_{inrtl} = [r_{x,inrtl} ~ r_{y,inrtl} ~ r_{z,inrtl}]
\end{equation*}
\begin{equation*}
 \vec v_{inrtl} = [v_{x,inrtl} ~ v_{y,inrtl} ~ v_{z,inrtl}]
\end{equation*}

It is desired to define an LVLH reference frame fixed at the object's center
of mass, with orientation measured with respect to the given inertial frame. The
resulting transformation matrix from inertial to LVLH would be:

\begin{equation*}
 \vec x_{lvlh} = T_{inrtl \rightarrow lvlh} ~ \vec x_{inrtl}
\end{equation*}

\begin{equation*}
 T_{inrtl \rightarrow lvlh} = \begin{bmatrix} p_1 & p_2 & p_3 \\
                                             -h_1 & -h_2 & -h_3 \\
                                             -r_1 & -r_2 & -r_3 \end{bmatrix}
\end{equation*}

where $r_i$ are the components of the unit radial vector,
$h_i$ are the components of the unit angular momentum vector,
and $p_i$ are the components resulting from the cross product of
$\hat r$ with $\hat h$.

The LVLH frame is rotating with respect to the inertial frame at a rate of one
revolution per orbit. Expressed in the LVLH frame, that relative angular velocity
is oriented along the $-\hat j$ axis only, by definition. Thus, the angular
velocity is:

\begin{equation*}
 \vec \omega_{lvlh/inrtl:lvlh}  = - \lvert \omega \rvert \hat j
\end{equation*}


\section{Detailed Design}
See the \href{file:refman.pdf}{Reference Manual}\cite{api:lvlh_frame}
for a summary of member data and member methods for all classes.

\subsection{Process Architecture}
The process architecture for the \LvlhFrameDesc\ is trivial; the \LvlhFrameDesc\
comprises the usual \textit{initialize} and \textit{update} methods, as
well as four methods for setting subject and planet names and pointers.
Finally, there is the protected method \textit{compute\_lvlh\_frame},
which calculates both the transformation matrix and angular velocity defined
in the previous section.

\subsection{Functional Design}
This section describes the functional operation of the methods in each class.

The \LvlhFrameDesc\ contains only one class, LvlhFrame. It contains the
following methods:
\begin{enumerate}

\funcitem{initialize}
This method sets up the model properly, does error checking on provided
object and planet names, and makes connections to the reference frame tree.

\funcitem{update}
The \textit{update} method calls \textit{compute\_lvlh\_frame}, which
determines the current orientation and angular velocity of the LVLH frame
with respect to the specified planet. If the specified planet is the
integration frame of the subject object with which the LVLH frame is
associated, then \textit{update} is effectively a pass-through. But
if the subject is integrated in a different frame, then \textit{update}
calculates the object's state with respect to the planet before calling
\textit{compute\_lvlh\_frame}.

\funcitem{set\_subject\_name}
This method allows the user to set the subject object's name via
function call.

\funcitem{set\_planet\_name}
This method allows the user to set the reference planet's name via
function call.

\funcitem{set\_subject\_frame}
Allows the user to set the subject object's body reference frame
pointer directly, via function call.

\funcitem{set\_planet}
Allows the user to set the reference planet pointeri directly, via
function call.

\funcitem{compute\_lvlh\_frame}
This method calculates the current orientation and angular velocity
of the LVLH reference frame with respect to the reference planet's
inertial frame.

\end{enumerate}


%----------------------------------
\chapter{User Guide}\hyperdef{part}{user}{}\label{ch:user}
%----------------------------------
The \LvlhFrameDesc\ is relatively straightforward to use. Simply include
the object in one's Trick S\_define, along with the \textit{initialize}
and \textit{update} function calls, then provide the names of the subject
object and planet either by using the provided setter methods or by
doing so directly. The model should then perform correctly. The
\LvlhFrameDesc\ is not intended to be extended via derived classes, as
it is a simple, single-purpose model.

An example inclusion of the \LvlhFrameDesc\ in one's Trick simulation is
as follows. In the S\_define, include something similar to the following
header, object declaration, and function calls:

\begin{verbatim}
##include "utils/lvlh_frame/include/lvlh_frame.hh"

public:
  LvlhFrame               lvlh_frame;

// Initialization jobs
P_DYN ("initialization") lvlh_frame.initialize (dyn_manager);
      ("initialization") lvlh_frame.update ();

// Environment class jobs
(1.0, "environment") lvlh_frame.update ();
\end{verbatim}

Note that this example assumes a dynamics manager exists elsewhere in
the simulation with the name ``dyn\_manager''. Also notice that the
\textit{update} method should be run once at initialization, following
the call to \textit{initialize}, and then on a recurring basis as an
``environment'' job. Here, the frequency is once a second.

Once the object has been included in the simulation as above, something
like the following should be added to the input file of the Trick
simulation to enable proper model operation:

\begin{verbatim}
# Configure the LVLH reference frame.
vehicle.lvlh_frame.set_subject_name ("vehicle.composite_body")
vehicle.lvlh_frame.set_planet_name ("Earth")
\end{verbatim}

This snippet assumes there is a planet that exists in the sim named
``Earth'', as well as a DynBody with the name ``vehicle''.


%----------------------------------
\chapter{Inspections, Tests, and Metrics}\hyperdef{part}{ivv}{}\label{ch:ivv}
%----------------------------------

\section{Inspection}\label{sec:inspect}

\inspection{Top-level Inspection}
\label{inspect:TLI}
This document structure, the code, and associated files have been inspected,
and together satisfy requirement~\traceref{reqt:toplevel}.

\inspection{Connections Inspection}
\label{inspect:design}
The model performs setup of itself during initialization, which includes
establishing appropriate connections between its internal pointers and the
corresponding specified structures outside of it, after error-checking the
user inputs.

Therefore by inspection, the \LvlhFrameDesc\ partially satisfies
requirement~\traceref{reqt:lvlhmanage}.


\section{Tests}\label{sec:tests}
This section describes the tests conducted to verify and validate
that the \LvlhFrameDesc\ satisfies the requirements levied against it.
All verification and validation test source code, simulations and procedures
are archived in the JEOD directory
{\tt models/utils/lvlh\_frame/verif}.\relax


\test{State Calculations}\label{test:calcs}
\begin{description}
\item[Background]\ \newline
The primary purpose of this test
is to determine whether the \LvlhFrameDesc correctly calculates the
rotational states of its LVLH reference frame with respect to the user-
specified planet's inertial frame. Implicitly, the test will also
demonstrate that the model can properly manage its connections and
internal data structures that are necessary for those calculations.

\item[Test description]\ \newline
This test involves checking the calculations performed by the \LvlhFrameDesc\
to ensure that the transformation matrix and angular velocity of its LVLH
frame is calculated correctly with respect to a given reference planet.

For this test, a simple Trick simulation of a single vehicle in orbit about
a single planet was employed. The vehicle is provided to the \LvlhFrameDesc\
for use as the subject body for the LVLH reference frame, since an LVLH frame
must be associated with a moving object in order to be mathematically valid.
The planet in the simulation is of course provided to the \LvlhFrameDesc\ as
the reference planet for the LVLH frame.

Once the state of the LVLH frame has been determined by the model, the output
is then checked by an independent, non-Trick set of routines. The results are
then compared against the Trick logfile for accuracy evaluation.

\item[Test directory] {\tt SIM\_LVLH\_Frame}

\item[Success criteria]\ \newline
A comparison of the output of the \LvlhFrameDesc\ and
\textit{compute\_relative\_state} should agree to within approximately
numerical error.

\item[Test results]\ \newline
All output data confirmed expectations, within numerical precision.

\item[Applicable Requirements]\ \newline
This test completes the satisfaction of requirement~\traceref{reqt:lvlhmanage},
and satisfies the requirement~\traceref{reqt:lvlhdefine}.

\end{description}


\newpage
\boilerplatetraceability

\newpage
\boilerplatemetrics
