%%%%%%%%%%%%%%%%%%%%%%%%%%%%%%%%%%%%%%%%%%%%%%%%%%%%%%%%%%%%%%%%%%%%%%%%%%%%%%%%
%
% ivv.tex
%
% Inspection, verification, and validation of the <model name>
%
% Usage instructions:
% This file should comprise a
% - The \chapter command, which must not be changed.
% - A number of \section command. You can change the labels but do not
%   change the names, and do not add any sections of your own.
% - Text to fill in the contents of the sections (which you must provide).
%
% Feel free to:
% - Change the labels on the \section commands.
% - Use your own \subsection commands and below. Structure is good.
% - Use figures and tables.
% - Split the file into parts if it gets too big.
%
%%%%%%%%%%%%%%%%%%%%%%%%%%%%%%%%%%%%%%%%%%%%%%%%%%%%%%%%%%%%%%%%%%%%%%%%%%%%%%%%

\chapter{Inspection, Verification, and Validation}
\hyperdef{part}{ivv}{}\label{ch:ivv}

\section{Inspection}\label{sec:inspect}
This section describes the inspections of the \MODELTITLE.

% \inspection{Top-level Inspection}
\label{inspect:TLI}
This document structure, the code, and associated files have been inspected, and together satisfy requirement~\ref{reqt:toplevel}.

\inspection{Representations}
\label{inspect:representations}
The model design hinges on a single class, the Orientation class.
This class defines two enumerations, one of which identifies each of the required
representation schemes. Public member data exist to represent each required
representation scheme.
Non-static member functions exist to
\begin{itemize}
\item Ensure that some desired representation is consistent with input data,
\item Set the orientation object to a specified value in any of the supported
representation schemes (setters),
\item Retrieve the equivalent value of an orientation in any of the supported
representation schemes (getters).
\end{itemize}

By inspection, the \ModelDesc satisfies
requirements~\ref{reqt:representations} and~\ref{reqt:data_access}.


\inspection{Euler Angles}
\label{inspect:euler}
The Orientation class also defines an enumeration that identifies all twelve
Euler angle sequences. The six aerodynamics sequences defined in the latter are
numerically identical to the corresponding values in the Trick implementation.
Each of the three static conversion methods that pertain to Euler angles handle
all twelve sequences.

By inspection, the \ModelDesc satisfies
requirement~\ref{reqt:euler_angles}.


\inspection{Mathematical Formulation}
\label{inspect:math}
The implementations of the static conversion methods
implement the corresponding algorithms as described in
section~\ref{sec:mathematics}.

By inspection, the \ModelDesc satisfies
requirement~\ref{reqt:conversions}.

\newpage
\section{Test}
This section describes various tests conducted to demonstrate
that the \MODELTITLEx satisfies the requirements levied against it.
The tests described in this section
are archived in the JEOD directory FIXME.


% \test{Simulation Interface Simulation}
\label{test:local_verif}
\begin{description}
\item[Background]
This test is an extremely simple simulation which
creates a Trick sim\_object containing 
an instance of a JeodTrickSimInterface to be tested.
One other Trick sim\_object creates a test object
with a single scheduled job.  The 
test object calls JeodTrickSimInterface::get\_job\_cycle() as
a scheduled job, thus the output can be compared
to the cycle time of the scheduled job.

The test object also includes private fields which are initialized by the 
Trick input processor and are logged by Trick.

\item[Test directory] {\tt verif} \\
This is a standard verification directory containing
the simulation directory  {\tt SIM\_sim\_interface}
along with src and include directories for the test class code.

\item[Success criteria]
The simulation includes initialization and logging of private fields 
which also reflect the output of 
JeodTrickSimInterface::get\_job\_cycle(). The logged data should be
identical to the 
reference data in the SET\_test\_val directory.

\item[Test results]
Passed.

\item[Applicable requirements]
This test demonstrates the satisfaction of
requirements \traceref{reqt:hidden_data_visibility},
\traceref{reqt:sim_engine_interface},
and \traceref{reqt:job_cycle}.
\end{description}


\test{Container Simulation}
\label{test:container_model_sim}
\begin{description}
\item[Background]
This simulation, located in the \CONTAINER\ verification directory,
exercises the checkpoint/restart capabilities of the model.
For a complete description of this test, see the
\hypermodelrefinside{CONTAINER}{part}{ivv} for details.
\item[Test Directory]
\verb|models/utils/container/verif/SIM_container_T10|
\item[Test Results]
Passed.
\item[Applicable Requirements]
This test demonstrates the satisfaction of
requirements \traceref{reqt:allocated_data_visibility},
\traceref{reqt:sim_engine_interface},
\traceref{reqt:checkpoint_restart},
and \traceref{reqt:addr_name_xlate}.
\end{description}


\test{Memory Simulation}
\label{test:memory_model_sim}
\begin{description}
\item[Background]
This simulation, located in the \MEMORY\ verification directory,
tests the ability of the \MEMORY\ to allocate and deallocate
memory. This in turn tests the ability of this model to
make those allocations visible to the simulation engine.
 For a complete description of this test, see the
\hypermodelrefinside{MEMORY}{part}{ivv} for details.
\item[Test Directory]
\verb|models/utils/memory/verif/SIM_memory_T10|
\item[Test Results]
Passed.
\item[Applicable Requirements]
This test demonstrates the satisfaction of
requirements
\traceref{reqt:allocated_data_visibility},
\traceref{reqt:sim_engine_interface},
and \traceref{reqt:checkpoint_restart}.
\end{description}


\test{Message Handler Simulation}
\label{test:message_model_sim}
\begin{description}
\item[Background]
This simulation, located in the \MESSAGE\ verification directory,
exercises the Trick-based implementation of the abstract
class MessageHandler.
For a complete description of this test, see the
\hypermodelrefinside{MESSAGE}{part}{ivv} for details.
\item[Test Directory]
\verb|models/utils/message/verif/SIM_message_handler_verif_T10|
\item[Test Results]
Passed.
\item[Applicable Requirements]
This test demonstrates the satisfaction of
requirements \traceref{reqt:sim_engine_interface}
and \traceref{reqt:trick_message_handler}.
\end{description}


\test{Propagated Planet Simulation}
\label{test:sim_prop_planet}
\begin{description}
\item[Background]
This simulation, located in the \EPHEMERIDES\ verification directory,
tests the ability of the \EPHEMERIDES\ to propagate a planet as an
alternative to using an ephemeris model.
The Trick 10 version was constructed to serve as a test case
of the multiple integration group capability provided by this model. 
 For a complete description of this test, see the
\hypermodelrefinside{EPHEMERIDES}{part}{ivv} for details.
\item[Test Directory]
\verb|models/environment/ephemerides/verif/SIM_prop_planet_T10|
\item[Test Results]
Passed.
\item[Applicable Requirements]
This test demonstrates the satisfaction of
requirements \traceref{reqt:allocated_data_visibility},
\traceref{reqt:sim_engine_interface},
\traceref{reqt:integ_interface},
\traceref{reqt:checkpoint_restart},
\traceref{reqt:addr_name_xlate},
and \traceref{reqt:multiple_integ_groups}.
\end{description}



\newpage
\section{Metrics}

Table~\ref{tab:coarse_metrics} presents coarse metrics on the source
files that comprise the model.
\input{coarse_metrics}

Table~\ref{tab:metrix_metrics} presents the extended cyclomatic complexity (ECC)
of the methods defined in the model.
\input{metrix_metrics}


\newpage
\section{Requirements Traceability}
This section is intentionally left blank for this release.
%Table~\ref{tab:reqt_ivv_xref} summarizes the inspections and tests
that demonstrate the satisfaction of the requirements levied on the model.

\begin{table}[htp]
\centering
\caption{Requirements Traceability}
\label{tab:reqt_ivv_xref}
\vspace{1ex}
\centering
\begin{tabular}{||l @{\hspace{4pt}} l|l @{\hspace{2pt}} l @{\hspace{4pt}} l|} \hline
\multicolumn{2}{||l|}{\bf Requirement} &
\multicolumn{3}{l|}{\bf Inspection or test} \\ \hline\hline
\ref{reqt:toplevel} & Project Requirements &
     Insp. & \ref{inspect:TLI}     & Top-level Inspection
\tabularnewline[4pt]
\ref{reqt:representations} & Supported Representations &
     Insp. & \ref{inspect:representations} & Representations \\
  && Test  & \ref{test:eigen}     & Eigen Rotation Test \\
  && Test  & \ref{test:euler}     & Euler Angles Test \\
  && Test  & \ref{test:instance}  & Instance Test
\tabularnewline[4pt]
\ref{reqt:data_access} & Data Access &
     Insp. & \ref{inspect:representations} & Representations \\
  && Test  & \ref{test:instance}  & Instance Test
\tabularnewline[4pt]
\ref{reqt:euler_angles} & Euler Angles &
     Insp. & \ref{inspect:euler}  & Design Inspection \\ 
  && Test  & \ref{test:euler}     & Euler Angles Test \\
  && Test  & \ref{test:instance}  & Instance Test
\tabularnewline[4pt]
\ref{reqt:conversions} & Conversions &
     Insp. & \ref{inspect:math}   & Mathematical Formulation \\
  && Test  & \ref{test:eigen}     & Eigen Rotation Test \\
  && Test  & \ref{test:euler}     & Euler Angles Test
\tabularnewline[4pt]
\hline
\end{tabular}
\end{table}
