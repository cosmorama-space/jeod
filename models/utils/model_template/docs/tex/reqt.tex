%%%%%%%%%%%%%%%%%%%%%%%%%%%%%%%%%%%%%%%%%%%%%%%%%%%%%%%%%%%%%%%%%%%%%%%%%%%%%%%%
%
% reqt.tex
%
% Requirements on the <model name>
%
% Usage instructions:
% This file should comprise a
% * The \chapter command, which must not be changed.
% * The toplevel requirement, which also should not be changed.
% * A set of model-specific requirements.
% * And nothing else.
%
%%%%%%%%%%%%%%%%%%%%%%%%%%%%%%%%%%%%%%%%%%%%%%%%%%%%%%%%%%%%%%%%%%%%%%%%%%%%%%%%

\chapter{Product Requirements}\hyperdef{part}{reqt}{}
\label{ch:reqt}

% All models must have this top-level requirement as requirement #1.

\requirement{Top-level requirement}
\label{reqt:toplevel}
\begin{description:}
\item[Requirement]
  The \MODELTITLEx shall meet the JEOD project requirements specified in
  the \JEODidx \hyperref{file:\JEODHOME/docs/JEOD.pdf}{part1}{reqt}
  {top-level document}.
\item[Rationale]
  This model shall, at a minimum, meet all external and
  internal requirements applied to the \JEODidx release.
\item[Verification]
  Inspection 
\end{description:}


% Add model-specific requirements per the following format:
% \requirement{Requirement name}
% \label{reqt:some_id}
% \begin{description:}
% \item[Requirement]
% What the model is to do, in requirementese.
% \item[Rationale]
% Additional verbiage that describes / motivates the requirement.
% \item[Verification]
% Verification technique(s); choose one of:
%   Inspection
%   Test
%   Inspection, test
% \end{description:}
%
% The Requirement name should be a short name for the requirment. This
% name will appear in the .pdf file.
% The given \label will not appear as text in the .pdf file. This label is
% how you refer back to the requirement in your inspections, tests, and
% traceability table.
% The "description:" items must be labeled as Requirement, Rationale, and
% Verification. Do not use any other labels.
% The "Requirement" shall use shall and must be verifiable. Be careful of the
% word 'and'. This indicates that you have multiple requirements lurking about.
% Split the requirement into multiple requirements or use the \subrequirement
% command.
% The "Rationale" is less encumbered of formalism than the requirement proper.
% Here you can elaborate on the intent of the requirement and explain why the
% requirement exists.
% The "Verification" is the easiest to fill in. It is either "Inspection",
% "Test", or "Inspection, test".
