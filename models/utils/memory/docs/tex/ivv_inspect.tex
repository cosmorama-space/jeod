%%%%%%%%%%%%%%%%%%%%%%%%%%%%%%%%%%%%%%%%%%%%%%%%%%%%%%%%%%%%%%%%%%%%%%%%%%%%%%%%
% ivv_inspect.tex
% Inspections of the Memory Model
%
% 
%%%%%%%%%%%%%%%%%%%%%%%%%%%%%%%%%%%%%%%%%%%%%%%%%%%%%%%%%%%%%%%%%%%%%%%%%%%%%%%%

\inspection{Top-level Inspection}
\label{inspect:TLI}
This document structure, the code, and associated files have been inspected.
With the exception of the use of the \emph{mutable} keyword (forbidden per the
coding standards), the \ModelDesc satisfies requirement~\ref{reqt:toplevel}.
The use of the \emph{mutable} keyword in the model has been granted a waiver.

\inspection{Design Inspection}
\label{inspect:design} 
Table~\ref{tab:design_inspection} summarizes the key elements of the
implementation of the \ModelDesc that satisfy requirements levied on the model.
By inspection, the \ModelDesc satisfies
requirements~\ref{reqt:alloc} to~\ref{reqt:reporting}.

\begin{longtable}{||l @{\hspace{4pt}} p{1.38in} |p{3.95in}|}
\caption{Design Inspection}
\label{tab:design_inspection} \\[6pt]
\hline
\multicolumn{2}{||l|}{\bf Requirement} & \bf{Satisfaction}
\\ \hline\hline
\endfirsthead

\caption[]{Design Inspection (continued from previous page)} \\[6pt]
\hline
\multicolumn{2}{||l|}{\bf Requirement} & \bf{Satisfaction}
\\ \hline\hline
\endhead

\hline \multicolumn{3}{r}{{Continued on next page}} \\
\endfoot

\hline
\endlastfoot

\ref{reqt:alloc} & Memory Allocation  &
  The \ModelDesc externally-usable memory allocation macros ultimately expand
  into calls to memory manager methods that allocate and initialize memory.
\tabularnewline[4pt]
\ref{reqt:registration} & \raggedright Memory Registration &
  The memory manager uses its memory interface object (established via
  the non-default constructor) to register all allocated memory with the
  simulation engine.
\tabularnewline[4pt]
\ref{reqt:free} & \raggedright Memory Deallocation &
  The externally-usable memory deallocation macros ultimately expand into calls
  to memory manager methods that destroy and deallocate memory.
\tabularnewline[4pt]
\ref{reqt:base_class_pointer_free} & \raggedright Free from Base Pointer &
  The memory manager accommodates for the fact that a derived class pointer
  upcast to a base class pointer does not always point to the same location in
  memory as the original derived class pointer. Memory allocated as a derived
  class can be deleted from a base class pointer to the allocated memory.
\tabularnewline[4pt]
\ref{reqt:deregistration} & \raggedright Memory Deregistration &
  The memory manager uses its memory interface object to inform the simulation
  engine that previously allocated memory is being released.
\tabularnewline[4pt]
\ref{reqt:threads} & \raggedright Thread Safety &
  The memory manager was designed, implemented, inspected, and tested with
  thread safety being one of the key goals.
\tabularnewline[4pt]
\ref{reqt:tracking} & \raggedright Memory Tracking &
  The memory manager maintains a table of all extant memory allocations.
\tabularnewline[4pt]
\ref{reqt:reporting} & \raggedright Memory Reporting &
  The memory manager provides the ability to report memory transactions
  and to report memory that has not been freed at the time the memory manager
  goes out of scope.
\tabularnewline[4pt]
\end{longtable}


\inspection{Peer Review}
\label{inspect:review}
The \ModelDesc underwent an extensive peer review on July 29, 2010.
While the primary focus of the peer review was thread safety, the review
covered the entirety of the model. Findings were captured in the
JEOD issue management system. Key findings included
\begin{itemize}
\item Eliminate use of the volatile keyword;
  many compilers do not handle this keyword properly.
\item Make the errors detected in the sensitive sections of
  begin\_atomic\_block and end\_atomic\_block fatal errors.
  Making these errors non-fatal might lead to deadlock.
\item Handle exceptions that might be thrown in a function called from
  within a sensitive block of code.
\end{itemize}
With the implementation of these findings, the \ModelDesc satisfies
requirement~\ref{reqt:threads}.
