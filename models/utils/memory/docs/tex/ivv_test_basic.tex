\test{Basic}\label{test:basic}
\begin{description}
\item[Test Directory]
{\tt models/utils/memory/verif/unit\_tests/basic}

\item[Test Description] This unit test exercises 
the basic ability of the \ModelDesc to allocate and deallocate memory.
The test allocates and later deletes the following types of data:
\begin{enumerate}
\item An object of a primitive type via {\tt JEOD\_ALLOC\_PRIM\_OBJECT},
\item Arrays of a primitive type via {\tt JEOD\_ALLOC\_PRIM\_ARRAY},
\item An array of pointers to structured data
  via {\tt JEOD\_ALLOC\_CLASS\_POINTER\_ARRAY},
\item An array of a structured type via {\tt JEOD\_ALLOC\_CLASS\_ARRAY},
\item Objects of a structured type via {\tt JEOD\_ALLOC\_CLASS\_OBJECT}.
\end{enumerate}

To run the test, enter the test directory and type the command
{\tt make build} to build the test program and then type the command
{\tt ./test\_program -verbose} to run the test program with debugging
enabled.

\item[Success Criteria]
The success criteria involve analyzing the debug output that results
from running the program in debug mode:
\begin{itemize}
\item Each memory allocation must indicate that
an appropriately-sized block of memory is allocated,
the allocated memory is registered the simulation interface,
the allocated memory is registered with the \ModelDesc, and
the allocated memory is initialized as expected.

\item Each memory deallocation must indicate that
the allocated memory: is deregistered with the simulation interface,
is properly destructed and freed, and
is deregistered with the \ModelDesc.

\item The summary output must report that all memory has been freed.
\end{itemize}

\item[Test Results]
The test passes.

\item[Applicable Requirements]
This test demonstrates the partial or complete satisfaction of the 
following requirements:
\begin{itemize}
\item \ref{reqt:alloc}. The {\tt JEOD\_ALLOC} macros properly
  allocate and initialize blocks of memory.
\item \ref{reqt:registration}. The model registers allocated memory with
  the simulation engine.
\item \ref{reqt:free}. The {\tt JEOD\_DELETE} macros properly
  destruct and delete blocks of memory.
\item \ref{reqt:base_class_pointer_free}. The {\tt JEOD\_DELETE} macros
  function properly when the target of the deletion is a base class pointer
  with a different address than that of the allocated memory.
\item \ref{reqt:deregistration}. The model deregisters released memory
  with the simulation engine.
\end{itemize}

\end{description}