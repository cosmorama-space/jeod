\section{Conceptual Design}
The main purpose of the \MessageHandlerDesc\ is to provide a common interface
for display or logging of debugging, informational, warning, error or failure messages.  
The \MessageHandlerDesc\ allows format and verbosity of messages to be
tailored to the needs of the user by means of input variables.  
MessageHandler is a base class with one pure virtual method 
(MessageHandler::process\_message) which must be implemented in a derived
class in order to create a functional \MessageHandlerDesc.  The 
TrickMessageHandler is such a derived class and
provides an implementation of process\_message using the Trick methods
exec\_terminate for fatal errors and send\_hs for all other messages.
If other avenues of output were needed, then a different subclass of
MessageHandler could be substituted for TrickMessageHandler in order
to provide the desired functionality.



\section{Mathematical Formulations}
The \MessageHandlerDesc\ includes no mathematical algorithms.
\section{Detailed Design}
The \MessageHandlerDesc\ includes a protected field,
MessageHandler::suppression\_level, which 
the user can set in the input file in order to control which messages will be suppressed.  There are also public static integer constants 
which indicate levels of severity.
\begin{table}
\begin{tabular}{lll}
Constant & Value & Meaning \\
MessageHandler::Fail & -1 & Prints a message and terminates simulation \\
MessageHandler::Error & 0 & Most severe error which does not terminate the simulation \\
MessageHandler::Warning & 9 & Indicates results are suspect \\
MessageHandler::Notice & 99 & Indicates an informational message \\
MessageHandler::Debug & 999 & Intended for messages to inform users of expected events
\end{tabular}
\caption{Symbolic Constants for Severity Levels}
\end{table}
\newpage
The public methods of the \MessageHandlerDesc\ are:
\begin{itemize}
\item Fail -- deliver a failure message and terminate the simulation
\item Error -- deliver an error message of severity MessageHandler::Error
\item Warn -- Deliver a warning message with severity MessageHandler::Warning
\item Inform -- Deliver an informational message with severity MessageHandler::Notice
\item Debug -- Deliver a debugging message with severity MessageHandler::Debug
\item Send\_message -- Deliver a generic message with severity
and labeling specified by the user
\item Va\_send\_message -- Just like send\_message except that an argument
list defined by starg.h replaces the variadic arguments of send\_message
\end{itemize}
All of these methods are declared static, thus, there is effectively
only one MessageHandler in any given simulation.  These static methods
are all wrappers for the virtual method process\_message, which, due
to limitations of C++ cannot be a static method.  
The programming technique used to circumvent this issue is to include
a static field (handler, of type MessageHandler) in the MessageHandler
class.  The default initialization of handler is NULL, and upon first
instantiation of MessageHandler, handler is set to this instance.
Subsequent instantiations of the MessageHandler class have no effect on the value of the 
static field handler.
With the introduction of the 
sim\_interface model, the MessageHandeler is constructed automatically.  
See Chapter ~\ref{ch:user} for details.

For the user, the implications are as follows:
\begin{itemize}
\item There will only be one instance of
the MessageHandler, and that one is created
automatically with the construction
of the JeodSimulationInterface.
\item Changes made to instance fields such as suppression\_level, 
suppress\_id and suppress\_location are only effective if performed on
the initial instance of MessageHandler
\end{itemize}

Additional details of the \MessageHandlerDesc\ design can be found in
\hyperref{file:refman.pdf}{}{}
{JEOD Message Model Reference Manual}~\cite{message_refman}.

.	\section{Inventory}

All \MessageHandlerDesc\ files are located in the directory
\$JEOD\_HOME/models/utils/message.
Tables~\ref{tab:source_files} to~\ref{tab:verification_files}
list the configuration-managed files in this directory.

\input{inventory}

