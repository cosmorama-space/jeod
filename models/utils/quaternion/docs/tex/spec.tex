\chapter{Product Specification}\hyperdef{part}{spec}{}\label{ch:spec}

\section{Conceptual Design}

\subsection{Basic Concepts}
The JEOD Quaternion class represents quaternions as
comprising a real scalar and an imaginary vectorial part,
thereby  satisfying requirements~\ref{reqt:encapsulation}
and~\ref{reqt:data}.
The Quaternion class defines methods needed to satisfy
requirements~\ref{reqt:orient} and~\ref{reqt:conversion}.
% Note that the Quaternion class does  
% not provide the full set of quaternionic mathematical operations on  
% quaternions described in Appendix~\ref{sec:app_math}.

\subsection{Trades}
\subsubsection{Data Representation}
Quaternions can be represented by a variety of data representation  
schemes. These include:
\begin{itemize}
\item A scalar real part and imaginary vectorial part. This approach  
explicitly acknowledges that quaternions are an extension of the  
complex numbers.
\item Four separate scalars, typically labeled w, x, y, and z. This  
approach has the advantage of simplicity. The key problem with this  
approach is that the real and imaginary parts are not well-
distinguished.
\item A four vector. This approach has the advantage of encapsulating the  
data in a single item that can easily be shared with other  
simulations, including those that do not use the JEOD Quaternion  
class. The  key problem with this approach is that it is not clear
which element denotes the real part of a quaternion. Some place the  
real part as the first (zeroth) element of the four vector while  
others place it as the last element.
\item A unit vector and an angle. This approach has the advantage of ease  
of visualization. The key problem with this approach is one must  
invoke trigonometric functions to compute the product of two  
quaternions.
\end{itemize}

The JEOD Quaternion class represents quaternions as comprising a  
scalar real part and imaginary vectorial part. As noted above, this approach
explicitly acknowledges that quaternions are an extension of the  
complex numbers.

\subsubsection{Use of Quaternions to Represent Orientation}
There are two coin-toss types of decisions to be made in using  
quaternions to represent orientation in three-space:
\begin{itemize}
\item Rotation versus transformation. Rotation is the conjugate of  
transformation. JEOD uses quaternions to represent transformations  
rather than rotations.
\item Left versus right quaternions.  Any quaternion can be used to  
represent a rotation or transformation of a three vector via either  
$\bmatrix 0 \\ x'\endbmatrix = Q \bmatrix 0 \\ x\endbmatrix Q^{\ast}$ or
$\bmatrix 0 \\ x'\endbmatrix = Q^{\ast} \bmatrix 0 \\ x\endbmatrix Q$ .
The only difference between these two forms is whether the quaternion appears
in an unconjugated form to the left or right of the vector to be rotated /  
transformed. Both forms are mathematically correct.
\end{itemize}

In addition to the above concerns is the issue of using quaternions in  
general versus unit quaternions to represent transformations. Taking  
the conjugate of a quaternion in general requires negating the  
quaternion's imaginary part and scaling by the inverse of the  
quaternion's magnitude. Because the magnitude of a unit quaternion is  
one by definition, the conjugate of a unit quaternion is  
easily constructed by negating the imaginary part of the quaternion.  
Converting a unit quaternion to the equivalent transformation matrix  
is also much simpler if the quaternion is a unit quaternion.

The JEOD Quaternion class uses left unit transformation quaternions.

\subsection{Functionality}
The minimal functionality needed to encapsulate quaternions as an  
object is the ability to construct a quaternion. The JEOD Quaternion  
class provides a default constructor that initializes a quaternion to  
all zeros. The class also provides convenience constructors whose functionality
could be achieved by using the default constructor in combination with  
one of the conversion methods.

The functionality needed to enable use of quaternions to represent the  
orientation of objects and reference frames in three-space includes:
\begin{itemize}
\item Normalization,
\item Multiplying by a real scalar,
\item Multiplying by a vector (interpreted as a pure imaginary quaternion), and
\item Multiplying by another quaternion, in various forms.
\end{itemize}

The JEOD Quaternion class provides methods to convert to/from:
\begin{itemize}
\item Transformation matrices;
\item Eigen rotations; and
\item Four vector representation of quaternions, with the scalar part  
being the zeroth element of the four vector.
\end{itemize}
Future versions of the Quaternion package may provide additional  
conversion capabilities.


\section{Mathematical Formulation}
This section summarizes key equations used in the implementation
of the \ModelDesc. See appendix~\ref{sec:app_math}
for a detailed description of the quaternions of the mathematical
formulation of the quaternions.

\subsection{Quaternion Representation}
The \ModelDesc represents quaternions as comprising a real scalar
and an imaginary vectorial part:
\begin{equation}
\quat Q = \quatsv {q_s} {\vect {q_v}}
\end{equation}
See section~\ref{sec:app_rep} for details.

\subsection{Quaternion Multiplication}
The product of two quaternions $\quat{Q}_1$ and $\quat{Q}_2$,
\begin{align*}
\quat{Q}_1 &= \quatsv {q_{1_s}} {\vect {q_{1_v}}} \\
\quat{Q}_2 &= \quatsv {q_{1_2}} {\vect {q_{1_2}}}
\end{align*}
is
\begin{equation}
\QxQ{\quat{Q}_1}{\quat{Q}_2} = 
  \quatsv
    {q_{1_s} q_{2_s} - \vect{q_{1_v}} \!\cdot \vect{q_{2_v}}}
    {q_{1_s} \vect{q_{2_v}} +
     q_{2_s} \vect{q_{1_v}} +
     \vect{q_{1_v}} \!\times \vect{q_{2_v}}}
  \label{eqn:mform:quat_prod_short}
\end{equation}
See section~\ref{sec:app_arith} for details.

\subsection{Quaternion Norm}
The norm, or magnitude of a quaternion $\quat Q$,
\begin{equation*}
\quat Q = \quatsv {q_s} {\vect {q_v}}
\end{equation*}
is given by
\begin{equation}
\label{eqn:mform:qnormsq}
||\quat Q||^2 = {q_s}^2 + {q_v}^2
\end{equation}
See section~\ref{sec:app_arith} for details.

\subsection{Quaternion Normalization}
The normalized quaternion based on a quaternion $\quat Q$,
\begin{equation*}
\quat Q = \quatsv {q_s} {\vect {q_v}}
\end{equation*}
is formed by scaling the quaternion by the inverse of the quaternion's norm:
\begin{equation}
\label{eqn:mform:qnorminv_exact}
\quat Q_{\text{normalized}} = \frac{1}{||\quat Q||} \quat Q
\end{equation}

Suppose the norm of the quaternion in question is close to one:
$||\quat Q||^2 = 1+\epsilon$, where $|\epsilon| \lll 1$.
The approximation
\begin{equation}
\label{eqn:mform:qnorminv_approx}
\frac{1}{||\quat Q||} \approx \frac 2 {1+||\quat Q||^2}
\end{equation}
is computationally much less expensive than the square root function
but is just as accurate as is the square root formulation for small $\epsilon$.

JEOD uses equation~\ref{eqn:mform:qnorminv_approx} in lieu of
equation~\ref{eqn:mform:qnorminv_exact}
when $|\epsilon|<2.107342*\times 10^{-8}$.

\subsection{Quaternion Derivative}
The time derivative of left transformation quaternion from frame A to frame B,
$\tquat {\:A} B$,
is
\begin{equation}
\tquatdot {\:A} B =
  \QxQ{\quatsv 0{-\frac{1}{2} {\framerelvect B \omega A B}}}{\tquat {\:A} B}
  \label{eqn:mform:quat_qdot}
\end{equation}
See section~\ref{sec:app_time_deriv} for details.


\section{Detailed Design}

The Quaternion model is located in the JEOD models directory
utils/quaternions.
One class, the Quaternion class, implements the Quaternion model.
The Quaternion model API is described in \cite{quat:api}.
