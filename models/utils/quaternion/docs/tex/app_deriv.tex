\section{Quaternion Time Derivative}\label{sec:app_time_deriv}

This section develops the time derivative of a left transformation quaternion.

The time derivative of a vector $\vect x$ is observer-dependent.
The relation between the time derivative of $\vect x$ as observed in an inertial
frame $I$ and the time derivative of $\vect x$ as observed in a frame $B$ rotating at
a rate $\omega$ with respect to the inertial frame\cite{Goldstein} is\begin{align}
\framedot{I}{\vect x} &= \framedot{B}{\vect x} + \vect \omega \times \vect x
  \label{eqn:quat_xdot_in_frame} \\
\intertext{where}
  \framedot{F}{\vect x}\;&\text{is}
    \; \text{the time derivative of $\vect x$ as observed in frame $F$} \nonumber\\
\end{align}

This can be expressed in quaternion form as
\begin{align}
\QBI & \QxQ
    {\left(
      \frac{d}{dt}
      \left(
        \QxVxQ
          {\QBIconj}
          {\framevect B x}
          {\QBI}
      \right)
    \right)}
    {\QBIconj}
  = \quatsv0{\framevdot B x}
    + \quatsv0{\framerelvect B \omega I B \times \framevect B x}
\label{eqn:quat_xdot_quat_form_1} \\
\intertext{where}
  \QBI\;&\text{is}
    \; \text{the left transformation quaternion from frame $I$ to frame $B$} \nonumber\\
  \framerelvect B \omega I B\;&\text{is}
    \; \text{the angular velocity of frame $B$ with respect to frame $I$, expressed in frame $B$}
      \nonumber\\
  \framevect B x\;&\text{is}
    \; \text{an arbitrary vector $\vect x$, expressed in frame $B$} \nonumber
\end{align}

The derivative in the left-hand side of equation~\eqref{eqn:quat_xdot_quat_form_1} expands to
\begin{equation}
\begin{split}
\text{\makebox[3.0em]{}}&\text{\makebox[-3.0em]{}}
 \frac{d}{dt} \left(\QxVxQ{\QBIconj}{\framevect B x}{\QBI}\right) =\\
   & \QxVxQ{\QBIconjdot}{\framevect B x}{\QBI}%\\
  \;+\; \QxVxQ{\QBIconj}{\framevdot B x}{\QBI} %\\
  \;-\; \quatconjlr{\QxVxQ{\QBIconjdot}{\framevect B x}{\QBI}}
\end{split}
\end{equation}

Applying the above to the left-hand side of equation~\eqref{eqn:quat_xdot_quat_form_1} and simplifying yields
\begin{equation}
\begin{split}
\text{\makebox[3.0em]{}}&\text{\makebox[-3.0em]{}}
\QxQxQ
  {\QBI}
  {\left(\frac{d}{dt} \left(\QxVxQ{\QBIconj}{\framevect B x}{\QBI} \right) \right)}
  {\QBIconj} = \\
& \QxQ{\QBI}{\QxV{\QBIconjdot}{\framevect B x}}
   - \quatconjlr{\QxQ{\QBI}{\QxV{\QBIconjdot}{\framevect B x}}}
  +  \quatsv0{\framevdot B x}
\end{split}\label{eqn:quat_xdot_quat_form_lhs}
\end{equation}

Using
\begin{equation}
\quatsv0{\framerelvect B \omega I B \times \framevect B x} =
\frac{1}{2}\left(\left(\QxQ{\quatsv0{\framerelvect B \omega I B}}{\quatsv0{\framevect B x}}\right)-
\quatconjlr{\QxQ{\quatsv0{\framerelvect B \omega I B}}{\quatsv0{\framevect B x}}}\right)
\end{equation}

The right-hand side of equation~\eqref{eqn:quat_xdot_quat_form_1} expands to
\begin{equation}
\begin{split}
\text{\makebox[3.0em]{}}&\text{\makebox[-3.0em]{}}
\frac{d}{dt} \quatsv0{\framevect B x}
    + \quatsv0{\framerelvect B \omega I B \times \framevect B x} = \\
   & \frac{d}{dt} \quatsv0{\framevect B x}
  + \frac{1}{2} \left(\left(\QxQ{\quatsv0{\framerelvect B \omega I B}}{\quatsv0{\framevect B x}}\right)
  - \quatconjlr{\QxQ{\quatsv0{\framerelvect B \omega I B}}{\quatsv0{\framevect B x}}}\right)
\end{split}\label{eqn:quat_xdot_quat_form_rhs}
\end{equation}

Equating
equations~\eqref{eqn:quat_xdot_quat_form_lhs}
and~\eqref{eqn:quat_xdot_quat_form_rhs}
and eliminating the common term $\quatsv0{\framevdot B x}$ yields
\begin{equation}
\begin{split}
& \QxQxQ{\QBI}{\QBIconjdot}{\quatsv0{\framevect B x}}
   - \quatconjlr{\QxQxQ{\QBI}{\QBIconjdot}{\quatsv0{\framevect B x}}} = \\
\qquad &  \left(\frac{1}{2} \QxQ{\quatsv0{\framerelvect B \omega I B}}{\quatsv0{\framevect B x}}\right)
  - \quatconjlr{\frac{1}{2} \QxQ{\quatsv0{\framerelvect B \omega I B}}{\quatsv0{\framevect B x}}}
\end{split} \label{eqn:quat_xdot_quat_form_2}
\end{equation}

Since equation \eqref{eqn:quat_xdot_quat_form_2} must be satisfied for \emph{any} vector $\vect x$,
\begin{equation}
\QxQ{\QBI}{\QBIconjdot} - \quatconjlr{\QxQ{\QBI}{\QBIconjdot}} = 
\left(\frac{1}{2} \quatsv0{\framerelvect B \omega I B}\right)
  - \quatconjlr{\frac{1}{2}\quatsv0{\framerelvect B \omega I B}}
  \label{eqn:quat_xdot_quat_form_3}
\end{equation}

Note that equation ~\eqref{eqn:quat_xdot_quat_form_3} is of the form
\begin{equation*}
\quat{Q}_a - \quatconj{Q}_a = \quat{Q}_b - \quatconj{Q}_b
\end{equation*}
Such a form requires that the vector parts of $\quat{Q}_a$ and $\quat{Q}_b$ be equal;
the scalar parts are unconstrained.
Since the scalar parts of $\QBI \QBIconjdot$ and $\quatsv0{\framerelvect B \omega I B}$
are zero, equation~\eqref{eqn:quat_xdot_quat_form_3} reduces to
\begin{align}
  \QxQ{\QBI}{\QBIconjdot} &= \frac{1}{2} \quatsv0{\framerelvect B \omega I B} \\
\intertext{or}
  \QBIdot &=  \QxQ{\quatsv0{-\frac{1}{2} {\framerelvect B \omega I B}}}{\QBI}
  \label{eqn:quat_qdot}
\end{align}

The \ModelDesc uses equation~\eqref{eqn:quat_qdot} to form the
derivative of the inertial-to-body left transformation quaternion $\QBI$.

Solving equation~\eqref{eqn:quat_qdot} for the body rate yields
\begin{align}
  \quatsv0{\framerelvect B \omega I B} &= 2 \QxQ{\QBI}{\QBIconjdot} 
  \label{eqn:quat_body_rate_from_qbidot}
\end{align}

Differentiating equation~\eqref{eqn:quat_qdot} again yields the second derivative
of the inertial-to-body left transformation quaternion $\QBI$:
\begin{align}
  \QBIdotdot &=
     \QxQ{\quatsv0{-\frac{1}{2} {\framerelvdot B \omega I B}}}{\QBI} +
     \QxQ{\quatsv0{-\frac{1}{2} {\framerelvect B \omega I B}}}{\QBIdot} \\
    &= 
     \QxQ
       {\quatsv{-\frac{1}{4} \norm{\framerelvect B \omega I B}^2}{-\frac{1}{2} {\framerelvdot B \omega I B}}}
     {\QBI}
    \label{eqn:quat_qdot_deriv}
\end{align} 
