\chapter{Integration Techniques}\label{app:overview}

\section{Overview}

The appendices that follow each describe a single technique for solving an
an initial value problem.
Some techniques natively address a first order ODE, others, a second order ODE.
A first order ODE solver can readily be adapted to solve a second order ODE,
but at the expense of ignoring some of the geometry of the problem.
The converse is not true for a second order ODE solver.
Most techniques, at their heart, address a scalar problem. These scalar-based
techniques can readily be adapted to solving a vector problem, but once again
at the expense of ignoring some of the geometry of the problem.
A small number (only one in JEOD 3.0) explicitly address a vector ODE problem
and do take advantage of the connections between the components of the vector.

\section{Classification}

A numerical integration technique can be classified in a number of ways:
\begin{itemize}
\item Whether the technique is a solver for a first or second order ODE.
\item Whether the technique takes advantage of the vectorial nature of
the problem.
\item Whether the technique uses derivatives from previous integration
intervals. The single step integrators do not use prior history.
The multistep integrations do, which means these techniques need to be
"primed" by some other integration technique.
\item Whether the technique adaptively splits the integration interval
into smaller cycles. LSODE is the only adaptive technique in JEOD 3.0.
\item Whether the technique subdivides an integration cycle into stages.
The Runge Kutta intergrators are multistage. The linear multistep integrators
are single stage (but typically are multistep).
\end{itemize}

\section{Implemented and Provided Techniques}

The implementation of an integration technique includes classes for solving
initial value problems and an integrator constructor that creates instances
of those solvers. The integrator constructor derives from the the \erseven
Integration model base class {\tt er7\_utils::IntegratorConstructor}.
This base class defines four overridable methods for constructing state
integrators:\begin{description}
\item[{\tt create\_first\_order\_ode\_integrator}]
Creates an integrator for a first order ODE initial value problem.
These integrators take a state vector and its time derivative as arguments.
The goal of the integration is to advance the state vector to some
desired end time.
\item[{\tt create\_second\_order\_ode\_integrator}]
Creates an integrator for a second order ODE initial value problem.
These integrators take a state vector and its first and second derivative
as arguments.
The goal of the integration is to advance the state vector and its first
derivative to some desired end time.
\item[{\tt create\_generalized\_deriv\_second\_order\_ode\_integrator}]
Creates an integrator for a generalized derivative second order ODE initial
value problem.
These integrators take a generalized position vector, a generalized velocity
vector, and the time derivative of generalized velocity as arguments.
The time derivative of generalized position is computed by some function
of generalized position and velocity; this function is provided as an
argument to the constructor function.
The goal of the integration is to advance generalized position and velocity
to some desired end time.
\item[{\tt create\_generalized\_step\_second\_order\_ode\_integrator}]
Creates an integrator for a second order Lie group initial value problem.
\end{description}

The base implementation of each of these is to generate an error message
and return a null pointer. Using a null pointer to perform integration will
cause the program to crash. The "Implemented and Provided Techniques" section
of each of the technique-specific appendices that follow contains a table that
indicates which of these four interfaces the technique supports.
This support can be
\begin{itemize}
\item Direct -- The integrator constructor returns a solver implemented in the
context of the technique,
\item Indirect -- The integrator constructor returns a solver that is
compatible with the technique, or
\item Non-existent -- The technique does not override the default
implementation.
\end{itemize}

\section{Mathematical Description}

The "Mathematical Description" section of each of the technique-specific
appendices that follow contains a description of the mathematics behind the
technique. References are provided if possible.

\subsection{First Order ODE}

Most of the provided techniques target a scalar first order ODE. Adapting
such a technique to a vectorial first order ODE is trivial: Simply apply
the technique individually to each element of the vector. This comes at a
cost of ignoring some of the geometry of the problem. LSODE specifically
targets a vector first order ODE and can take geometry into account.

Some of the provided techniques target second order ODEs. These techniques
typically cannot be used to solve first order ODE initial value problems. 
These techniques provide a surrogate first order solver that is compatible with
the second order ODE technique when asked to create a solver for a
first order ODE initial value problem.

\subsection{Second Order ODE}

All of the provided techniques provide solvers for second order ODEs. A first
order ODE solver can readily be adapted to solving a second order ODE
initial value problem, either by state doubling or by applying the technique
to position first and velocity second. The remaining techniques explicitly
solve a second order ODE initial value problem.

\subsection{Generalized Position / Generalized Velocity ODE}

All of the provided techniques provide solvers for generalized position /
generalized velocity initial value problems. The constructor for the
solver takes the functions that computes the first and second derivatives of
generalized position as arguments. The same mechanisms used to adapt a
first order ODE solver to a second order ODE problem can be employed using
these functions. These functions can similarly be used by second order
ODE solvers to adapt that solver to a generalized position / generalized
velocity problem.

\subsection{Lie Group Second Order ODE}

Lie group solvers only exist for some of the integration techniques.
A Lie group problem can always be recast as a generalized position /
generalized velocity problem, so techniques that do not provide a Lie
group integrator can still be used to solve these kinds of problems
(but at the cost of accuracy).

\section{Usage Instructions}

The "Usage Instructions" section of each of the technique-specific
appendices that follow contains a description of how to use the technique.
