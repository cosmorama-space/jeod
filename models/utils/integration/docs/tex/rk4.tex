\chapter{Classical Fourth Order Runge Kutta (RK4)}\label{app:rk4}

\section{Overview}

The classical fourth order Runge Kutta integrator is one of the most widely
used techniques for solving initial value problems.
The technique addresses initial value problems for first order ODEs.
As the name suggests, this technique is a Runge Kutta method,
a single step (i.e., no past history), multi-stage integrator.

\section{Classification}

TBS

\section{Implemented and Provided Techniques}

Table \ref{tab:rk4_method_problems} lists the types of initial value problems
that can be used in conjunction with an RK4-based integrator
constructor. As noted in the table, RK4 implementations exist
for all four types of initial value problems addressed by the
ER7 Utilities integration suite.

\begin{table}[htp]
\centering
\caption{RK4 Initial Value Problems}
\label{tab:rk4_method_problems}
\vspace{1ex}
\begin{tabular}{ll}\hline
\bf{Problem Type} & \bf{Support} \\
\hline\hline
\tt{FirstOrderODE} & Implemented \\
\tt{SimpleSecondOrderODE} & Implemented \\
\tt{GeneralizedDerivSecondOrderODE} & Implemented \\
\tt{GeneralizedStepSecondOrderODE} & Implemented \\
\hline
\end{tabular}
\end{table}

\section{Mathematical Description}

The ER7 utilities RungeKutta4 integration technique is the standard
fourth order, four stage Runge Kutta method.
Table~\ref{tab:rk4_butcher} specifies the Butcher tableau for this method.
 
\begin{table}[htp]
\centering
\caption{RK4 Butcher Tableau}
\label{tab:rk4_butcher}
\vspace{1.5ex}
\begin{tabular}{c|cccc}
0 &&&& \\
1/2 & 1/2 &&& \\
1/2 & 0 & 1/2 && \\
1 & 0 & 0 & 1 & \\
\hline
& 1/6 & 1/3 & 1/3 & 1/6 
\end{tabular}
\end{table}

The equations that govern propagation via the standard RK4 method are given by
equations~(\ref{eqn:rk4_0}) to~(\ref{eqn:rk4_3}).

Stage 0:
\begin{equation}
\label{eqn:rk4_0}
\begin{split}
t_1 &= t_i + \frac{\Delta t}2 \\
\dots_0 &= \dot s(t_i,s(t_i)) \\
s_1(t_1) &= s(t_i) + \frac{\Delta t}2 \, \dots_0
\end{split}
\end{equation}
Stage 1:
\begin{equation}
\label{eqn:rk4_1}
\begin{split}
t_2 &= t_1 = t_i + \frac{\Delta t}2 \\
\dots_1 &= \dot s(t_1,s_1(t_1)) \\
s_2(t_2) &= s(t_i) + \frac{\Delta t}2 \, \dots_1
\end{split}
\end{equation}
Stage 2:
\begin{equation}
\label{eqn:rk4_2}
\begin{split}
t_3 &= t_f = t_i + \Delta t \\
\dots_2 &= \dot s(t_2,s_1(t_2)) \\
s_3(t_3) &= s(t_i) + \Delta t \, \dots_2
\end{split}
\end{equation}
Stage 3 (final):
\begin{equation}
\label{eqn:rk4_3}
\begin{split}
\dots_3 &= \dot s(t_3,s_1(t_3)) \\
t_4 &= t_3 = t_f = t_i + \Delta t \\
\bar {\dots} &=
  \frac{\dots_0 + 2\dot s_1 + 2\dots_2+\dot s_3} 6 \\
s(t_f) &= s(t_i) + \Delta t \, \bar {\dots}
\end{split}
\end{equation}

\section{First Order ODE}

The \erseven implementation of the fourth order Runge-Kutta technique applies
equations~(\ref{eqn:rk4_0}) to~(\ref{eqn:rk4_3})
to each element of a vector-valued first order ODE initial value problem.

\section{Second Order ODE}

The \erseven implementation of the fourth order Runge-Kutta technique applies
equations~(\ref{eqn:rk4_0}) to~(\ref{eqn:rk4_3})
to each element of the position vector and then
to each element of the velocity vector to solve a
vector-valued second order ODE initial value problem.

\section{Generalized Position / Generalized Velocity ODE}

The \erseven implementation of the fourth order Runge-Kutta technique applies
equations~(\ref{eqn:rk4_0}) to~(\ref{eqn:rk4_3})
to each element of the position vector (using the generalized position
derivative function to compute the derivative of the position vector)
and then to each element of the velocity vector to solve a
vector-valued generalized position / generalized velocity initial value problem.


\section{Lie Group Second Order ODE}

TBS


\section{Usage Instructions}
The classical fourth order Runge-Kutta technique is the default integration
technique for a monolithic JEOD simulation.

To specifically select this technique,
set the {\tt{jeod\_integ\_opt}} element of the simulation's
{\tt{DynManagerInit}} object to {\tt{trick.Integration.RungeKutta4}}
in the appropriate Python input file.
