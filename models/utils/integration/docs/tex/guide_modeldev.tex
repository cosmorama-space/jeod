This section has two primary components:  how to add new integration 
techniques to the \ModelDesc, and how to use the \ModelDesc in other 
extensions of JEOD(see \textref{Using This Model in JEOD 
 Extensions}{sec:user_modeldev_2}).

\subsection{Adding New Integration Techniques}\label{sec:user_modeldev_1}
A wide range of numerical integration techniques exist,
only a few of which are implemented by JEOD. This section
describes what needs to be done to implement a numerical integration
algorithm using the \ModelDesc architecture.

A model extender will typically need to write four classes:
\begin{itemize}
\item A class that derives from the TwoStateIntegrator class.
This derived class will implement the desired numerical integration algorithm.

The new TwoStateIntegrator class needs to define
\begin{itemize}
\item A non-default constructor for this class, including arguments typically 
specifying
\begin{itemize}
 \item The number of elements in the zeroth-derivative (e.g. position) 
 specification, and its derivative,
 \item The number of elements in the first-derivative (e.g. velocity) 
 specification, and its derivative,
 \item An instance of a method used to compute the derivative of the 
 zeroth-derivative state (typically from the zeroth- and first-derivative 
 states).  This method originates from the template structure in the 
 \textit{RestartableStateIntegrator} class, and is discussed further in the 
 next section (see \textref{Using This Model in JEOD 
 Extensions}{sec:user_modeldev_2}),
 \item A reference to the Integration Controls appropriate to the integrator.
\end{itemize}
This constructor should then, in turn, call the base-class TwoStateIntegrator 
constructor with the same set of arguments.

\item An \verb+integrate+ method that implements the algorithm.  This method 
must have the following arguments:
\begin{itemize}
 \item A double representing the time-step (this is the cycle-time-step, for 
 the dynamic-time)
 \item An unsigned integer representing the cycle-stage to which the 
 integrator is moving, the target-stage.  This should come into the integrator 
 from the controls, via the integration group.  That way, all integrators in 
 the group are assured to have the same target.
 \item Four double-pointers representing, in order, the time-derivative of the 
 first-derivative-state, the first-derivative-state, the time-derivative of 
 the zeroth-derivative-state, and the zeroth-derivative-state.  These should 
 all point to arrays appropriately sized according to the values input into 
 the constructor (first two arguments, outlined above).
\end{itemize} 

\item Data members to contain internal storage needed by the algorithm,
\item A constructor and destructor to allocate and release memory,
\item A \verb+delete_self+ method to make an instance delete itself,
\item A \verb+reset+ method to make an instance of the class reset itself.  
This method can be empty if the integration algorithm is self-starting, but 
must be defined.

\end{itemize}


\item A class that derives from the TimeIntegrator class.
This derived class will advance time in a manner consistent with the
numerical integration algorithm.

The new TimeIntegrator class needs to define
\begin{itemize}
\item An \verb+integrate+ method that advances time per the dictates of the
desired numerical integration algorithm.
The time integrator and state integrator must march in unison.
\item A \verb+delete_self+ method to make an instance delete itself,
\item A \verb+reset+ method to make an instance of the class reset itself.
\item a non-default constructor that takes a reference argument to the 
appropriate Integration Controls, and passes that on through to the 
TimeIntegrator non-default constructor.
\end{itemize}

\item A class that derives from the IntegrationControls class.

There are no required methods for the new class (the base class, 
IntegrationControls, is an instantiable class); however, the following methods 
may be necessary:
\begin{itemize}
 \item If the new 
integrator represents a multi-cycle process, the new class must set the 
\verb+multi_cycle+ flag to true in the constructor, and define
\begin{itemize}
 \item A \verb+start_integration_tour+ method if any additional processes are 
 required when a new integration tour (not just an 
 integration cycle) starts.
 \item An \verb+end_integration_cycle+ method that tests for completion of the 
 integration tour at the end of each integration cycle, and provides a 
 suitable 
 location for processes that occur only at the end of an integration cycle or 
 an integration tour.  The return value from this method should be 
 \textit{true} if the tour has completed, and \textit{false} otherwise. 
\end{itemize}
\item  A \verb+reset+ method if any additional verification or assignments 
need to be made when the integrator is reset (default method is empty).
\item While the \verb+integrate+ method is a virtual method (so can easily be 
redefined for the new class), note that this method controls the entire 
sequencing of the integration process and is critically important.  It has 
been carefully constructed to provide the flexibility to accommodate a very 
broad array of integration techniques; it may be easier to tweak the 
integration algorithm to fit into this existing framework than to rewrite the 
framework.

\end{itemize}

 Note -

Integrators flagged as multi-cycle get two additional method calls 
(\verb+start_integration_tour+ and \verb+end_integration_cycle+) regardless of 
how many cycles they take per tour.  This feature can be useful for more than 
just monitoring multiple cycles. For example, the Beeman and ABM4 methods are 
multi-step, single-cycle methods, but the multi-cycle flag is set to 
assist with the priming process.  In the class PrimingIntegrationControls, the 
\verb+start_integration_tour+ method is used to monitor the priming process of 
these integrators; once priming is complete, the multi-cycle flag is reset to 
false, and the method is not revisited.

\item A class that derives from the IntegratorConstructor class. This derived 
class will create instances of the above three classes.  The following methods 
must be defined:
\begin{itemize}
 \item A \verb+create_two_state_integrator+ method.  This method takes the 
 four arguments listed above for the constructor of the TwoStateIntegrator 
 class.  These values are passed directly through to the instantiation of the 
 appropriate TwoStateIntegrator-derived class.  Note - these values are set 
 externally to the \ModelDesc with default values within JEOD; for more 
 information, see \textref{Using This Model in JEOD 
 Extensions}{sec:user_modeldev_2}).
 \item  A \verb+create_time_integrator+ method.  This method takes the 
 integration controls reference as an argument, and passes it through to the 
 instantiation of the appropriate TimeIntegrator-derived class.
 \item A \verb+create_integration_controls+ method.  This method takes no 
 arguments, and instantiates the appropriate IntegrationControls-derived class.
 \item A \verb+delete_self+ method.  This method deletes the Integrator 
 Constructor itself.
\end{itemize}

\end{itemize}

General note -
 
While the constructor must always create instances of a two-state integrator, 
a 
time integrator, and an integration controls, existing classes can be re-used 
if the functionality is identical.  Examples:
\begin{itemize}
 \item Most integrators use the default IntegrationControls.
 \item The Symplectic Euler method uses the time integrator from the Euler 
 method, since their procession in time are equivalent.
\end{itemize}




The ability to extend the model was tested in test~\ref{test:extensibility}.



\subsection{Using This Model in JEOD Extensions}\label{sec:user_modeldev_2}
Within JEOD, the \ModelDesc is purely used for the integration of a 
six-degree-of-freedom body, but the actual mathematical models within the 
\ModelDesc are far more generic, and can be applied to any second-order system.
Of particular note is the recognition that the first-derivative-state, while 
intrinsically related to the zeroth-derivative-state, does not have to be a 
time-derivative of it.  The example already in JEOD is the integration of the 
angular state, where the orientation and rate of orientation change are 
specified as quaternions, while the angular rate and rate of change of angular 
rate are specified as Cartesian vectors.

This same system is therefore useful in many other applications, such as a 
Hamiltonian system utilizing generalized momenta and generalized coordinates 
as the two states.  Indeed, the two states may be completely unassociated as 
time-derivatives.

In JEOD, the Simple6DofDynBody (see the Dynamics Body 
documentation~\cite{dynenv:DYNBODY}) implements two 
integration generators, one for the translational state, and one for the 
rotational state.  These are instantiated in 
\textit{dynamics/dyn\_body/simple\_6dof\_dyn\_body.hh}:
\begin{itemize}
 \item \verb+SimpleRestartableIntegrator<3> trans_integ_generator;+
 \item \verb+RestartableIntegrator<4, 3, Quaternion::compute_left_quat_deriv>+ 
        \newline \ \ \ \ \ \ \ \verb+rot_integ_generator;+
\end{itemize}

These structures are both templatized in 
\textit{restartable\_state\_integrator.hh}, with the 
SimpleRestartableIntegrator 
a restricted case of the more general RestartableIntegrator.

In the latter, general, implementation, the arguments are:
\begin{enumerate}
 \item The number of entries in the first state array (state[0], referred to 
 as the zeroth-derivative-state in JEOD applications)
 \item The number of entries in the second state array (state[1], referred to 
 as the first-derivative-state in JEOD applications)
 \item The specified method for creating the derivative of state[0] from 
 state[0] and state[1].
\end{enumerate}

With the integrator generators so specified, then control passes to the 
IntegrableObject (in JEOD, this is usually a Simple6DofDynBody, which is a 
DynBody, which is an IntegrableObject (among other things)).  Somewhere in the 
creation of the IntegrableObject-derived instance, (for a Simple6DofDynBody, 
this is in the method \textit{Simple6DofDynBody::create\_integrators}), the 
generator method \textit{create\_integrator} is called, passing arguments of 
the appropriate instance of the IntegratorConstructor and IntegrationControls 
that will be used to integrate the state of this object.

The generator (an instance of the RestartableIntegrator class) method 
\textit{create\_integrator} calls the IntegratorConstructor method 
\textit{create\_two\_state\_integrator}, with 4 arguments: the three arguments 
used in the 
template discussed above, and the integration controls just passed in.

Thus, an integrable object generates its two-state integrator that was 
discussed in section~\ref{sec:user_modeldev_1}.  Note that the 
integration-controls and time-integrator that the integrator-constructor 
creates are created elsewhere, in the generation of an integration group.

To extend this architecture, it is really only necessary to implement a 
different IntegrableObject (other than a Simple6DofDynBody), and in that 
implementation, put a different method in for the generation of the 
time-derivative of state[0].  The Simple6DofDynBody has a translational 
two-state, and a rotational two-state, with each two-state representing 3 
degrees of freedom - hence 6 degrees of freedom total.  The new object can, in 
principle, have any number of two-states, hence any number of degrees of 
freedom; the only requirement is that all of the two-state sets be mutually 
independent.
