\chapter{Fourth Order Adams-Bashforth-Moulton}\label{app:abm4}

Adams-Bashforth-Moulton methods combine an explicit Adams-Bashforth
integrator as a predictor and an implicit Adams-Moulton integrator
as a corrector. The fourth order Adams-Bashforth-Moulton method
implemented in JEOD uses the four step Adams-Bashforth integrator
as a predictor and the three step Adams-Moulton integrator as a corrector.

\subsubsection{Maintaining the Derivative History}
All but the simplest of Adams-Bashforth-Moulton methods require a history
of prior derivatives. JEOD uses the fourth order Adams-Bashforth-Moulton method
which requires four sets of prior derivatives.

On the first step of an integration cycle the JEOD implementation of the
ABM4 method saves the input derivatives in a derivative history buffer.
How derivatives are saved depends on whether the requisite set of four prior
derivatives have been gathered. The method operates in priming mode until the
requisite set of four prior derivatives have been gathered.

In priming mode, the derivatives are stored consecutively in the slot designated
by a primer counter which advances from zero to three as derivatives are
gathered. Once primed, the derivative history buffer is treated as a circular
array. New derivatives are copied into the slot occupied by the oldest data
and pointers are advanced circularly.

\subsubsection{Integrating State}
The state is advanced on each integration step. The primer is used to
advance the state while the method is in priming mode.
Once the requisite four sets of derivatives from the primer have been gathered,
the ABM4 integrator uses a two stage predictor-corrector technique to advance
the state.
The equations that govern propagation via the ABM4 method are given by
equations~(\ref{eqn:abm4_0}) and~(\ref{eqn:abm4_1})\cite{hamming:1986}.

Stage 0 (predictor):
\begin{equation}
\label{eqn:abm4_0}
\begin{split}
\bar {\dot{\vect s}}_{AB} &=
  \frac{55\dot{\vect s}_0 - 59\dot{\vect s}_{-1} +
        37\dot{\vect s}_{-2} - 9\dot{\vect s}_{-3}} {24} \\
t_1 &= t_f = t_i + \Delta t \\
\vect s_1(t_1) &= \vect s(t) + \Delta t \, \bar {\dot{\vect s}}_{AB}
\end{split}
\end{equation}

Stage 1 (corrector):
\begin{equation}
\label{eqn:abm4_1}
\begin{split}
\dot{\vect s}_1 &= \dot{\vect s}(t_1,\vect s(t_1)) \\
\bar {\dot{\vect s}}_{AM} &=
  \frac{9\dot{\vect s}_1 +19\dot{\vect s}_0 -
        5\dot{\vect s}_{-1} + \dot{\vect s}_{-2}} {24} \\
t_2 &= t_1 = t_f = t_i + \Delta t \\
\vect s(t_2) &= \vect s(t) + \Delta t \, \bar {\dot{\vect s}}_{AM}
\end{split}
\end{equation}

\subsubsection{ABM4 Primer}
The simplest of the Adams-Bashforth-Moulton methods combines explicit
and implicit Euler integration. This simplest implementation does not
rely on historical data. All other Adams-Bashforth-Moulton methods,
including the fourth order method implemented in JEOD,
do rely on historical data; they are not self-starting.
Some other technique must be used to prime the method.
Once primed the Adams-Bashforth-Moulton methods are self-sustaining.
As both the fourth order Adams-Bashforth-Moulton method and the fourth order
Runge Kutta method have fourth order accuracy, JEOD uses the RK4 method
as a primer for the ABM4 method.

