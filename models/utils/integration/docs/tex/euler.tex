\chapter{Euler's Method}\label{app:euler}

\section{Overview}

Euler's method is the simplest of the integrators.
The technique addresses initial value problems for first order ODEs.
Euler's method propagates state to the end of an integration interval
in one step by assuming that the values of the derivatives over the
integration interval are equal to the values at the start of the interval.

\section{Classification}

Euler's method is a single step, single stage, single cycle integrator
for scalar-valued first order ODEs.
The technique exhibits a global truncation error that is
proportional to the step size.
As is the case for other solvers for scalar-valued first order ODEs, Euler's
method can easily be adapted to solving vector-valued initial value problems
of arbitrary order.

\section{Implemented and Provided Techniques}

Table \ref{tab:euler_method_problems} lists the types of initial value problems
that can be solved using Euler's method.
As noted in the table, Euler's method implementations exist
for all four types of initial value problems addressed by the
ER7 Utilities integration suite.

\begin{table}[htp]
\centering
\caption{Euler Method Initial Value Problems}
\label{tab:euler_method_problems}
\vspace{1ex}
\begin{tabular}{ll}\hline
\bf{Problem Type} & \bf{Support} \\
\hline\hline
\tt{FirstOrderODE} & Implemented \\
\tt{SimpleSecondOrderODE} & Implemented \\
\tt{GeneralizedDerivSecondOrderODE} & Implemented \\
\tt{GeneralizedStepSecondOrderODE} & Implemented \\
\hline
\end{tabular}
\end{table}

\section{Mathematical Description}

Euler's method is the simplest of techniques for numerically
solving a scalar first order ODE initial value problem given
by equation~(\ref{eqn:euler_scalar_1st_order_ode_problem}).

\begin{equation}
\label{eqn:euler_scalar_1st_order_ode_problem}
\begin{split}
\frac{d s(t)}{dt} &= f(t, s(t)) \\
s(t_0) &= s_0
\end{split}
\end{equation}

Euler's method propagates state from time $t$ to $t+\Delta t$
by assuming that the derivative is constant between $t$ and $t+\Delta t$
and is equal to the value at the start of the integration interval
\cite{butcher:2008}:

\begin{equation}
\label{eqn:euler_scalar_1st_order_ode_step}
s(t+\Delta t) = s(t) + f(t, s(t)) \Delta t
\end{equation}

Note that in numerical integration literature, it is the integration technique
that calls the derivative function $f(t, s(t))$. That is not the case
with Trick integration, and hence with the ER7 utilities integrators.
The derivatives are instead inputs to the integration function.

\subsection{First Order ODE}

The \erseven implementation of Euler's method applies
equation~(\ref{eqn:euler_scalar_1st_order_ode_step})
to each element of a vector-valued first order ODE initial value problem.

\subsection{Second Order ODE}

The \erseven implementation of Euler's method applies
equation~(\ref{eqn:euler_scalar_1st_order_ode_step})
to each element of the position vector and then
to each element of the velocity vector to solve a
vector-valued second order ODE initial value problem.

\subsection{Generalized Position / Generalized Velocity ODE}

The \erseven implementation of Euler's method applies
equation~(\ref{eqn:euler_scalar_1st_order_ode_step})
to each element of the position vector (using the generalized position
derivative function to compute the derivative of the position vector)
and then to each element of the velocity vector to solve a
vector-valued generalized position / generalized velocity initial value problem.


\subsection{Lie Group Second Order ODE}

Applying Euler's method to a second order Lie group problem involves
advancing generalized position using the Lie group exponential map step function
applied against the initial generalized position and generalized velocity.
Generalized velocity is once again advanced per the basic Euler method.

The exponential map step function
$\expmapStep(\Delta \vect{\theta}, \vect{x})$ advances
position via $\expmap(\Delta \vect{\theta})\cdot \vect x$. In this expression,
the exponential is the exponential map function that maps from the tangent
Lie algebra space at $x$ to the Lie group. The multiplication of this
quantity with the input generalized position is performed per the group operator
of the Lie group.

\begin{equation}
\label{eqn:euler_lie_group_position}
\begin{split}
\splitcenter{\Delta \vect{\theta}} &= \vect v(t) \Delta t \\
\vect x(t+\Delta t) &= \expmapStep(\Delta \vect{\theta}, \vect x(t)) \\
\vect v(t+\Delta t) &= \vect v(t) + \dot{\vect v}(t) \Delta t
\end{split}
\end{equation}

\section{Implementation}

TBS

\section{Usage Instructions}

To use Euler's method to integrate the dynamics of a monolithic JEOD simulation
(\emph{i.e.}, one in which the dynamics manager performs the integration),
set the {\tt{jeod\_integ\_opt}} element of the simulation's
{\tt{DynManagerInit}} object to {\tt{trick.Integration.Euler}}
in the appropriate Python input file.

The method is provided for backwards compatibility reasons and because
Euler's method forms the basis of almost all numerical integration techniques.

It is recommended that Euler's method not be used in any simulation.
Euler's method is highly inaccurate and rather unstable.
The only advantage of this technique is that it needs just one
evaluation of the derivative function per integration cycle.
More accurate techniques that similarly require just one derivative evaluation
per integration cycle exist.
Use one of these alternatives in lieu of using Euler's method,
or use a higher order technique.
Any other technique is better than Euler's method.
