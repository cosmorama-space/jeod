%%%%%%%%%%%%%%%%%%%%%%%%%%%%%%%%%%%%%%%%%%%%%%%%%%%%%%%%%%%%%%%%%%%%%%%%
%
% Purpose:
%
%  
%
%%%%%%%%%%%%%%%%%%%%%%%%%%%%%%%%%%%%%%%%%%%%%%%%%%%%%%%%%%%%%%%%%%%%%%%%%

\begin{abstract}

The \aerodynamicsDesc\ computes the forces and torques that model the effects of
aerodynamics on a vehicle.  These forces and torques are summed into
the forces and torques that model the overall vehicle dynamics.  These
models are used to characterize the affects of aerodynamics on a vehicle's
trajectory.  The \aerodynamicsDesc\ consists of two drag modeling options.
There is a simple drag model of using the coefficient of drag or the
ballistic coefficient (also known as the ballistic number) method.
A composite plate model with surfaces that include accommodation factors
for specular, diffuse or mixed reflections (mixed here means a
combination of specular and diffuse).

Additionally, an extensible framework for the addition of new methods of
calculating drag has been implemented, giving flexibility to the end
user of the \JEODid\ package.

\end{abstract}
