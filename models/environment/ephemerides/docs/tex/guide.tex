%----------------------------------
\chapter{User Guide}\hyperdef{part}{user}{}\label{ch:user}
%----------------------------------

\section{Instructions for Simulation Users}
\label{sec:guide_sim_users}

This section describes how to set up and use the \ModelDesc in an 
existing Trick 10 simulation.

\subsection{Controls}

The simulation developer typically establishes which ephemeris models are
employed in a given simulation.
In most simulations, the simulation user should leave this as-is.

The user may want to specify the JPL Development Ephemeris version which can be 
done via input file. The default configuration is set to DE405. To change this, 
simply add the following line in the input file.
 
\begin{codeblock}
env.de4xx.set_model_number(<integer value of model>)
## Ex.
env.de4xx.set_model_number(440)
\end{codeblock}

JEOD currently provides versions 405, 421, and 440.

The JEOD dynamics manager and dynamics integration provides the ability
to control whether ephemerides are updated as derivative class jobs.
This can be set from the input file.
\begin{codeblock}
dynamics.dyn_manager.deriv_ephem_update = <True or False>
\end{codeblock}
Note however that overriding this from the default for your simulation
may invalidate the simulation results.


\subsection{Accessing Ephemerides}
\label{sec:guide_sim_users_accessing_ephemerides}
In general, you do not access the ephemerides directly from the
ephemeris models. You instead those data from those planet reference frames.
The ephemeris models store the translational state of a planet
in the planet's inertial reference frame, the rotational state of
a planet in the planet's planet-fixed reference frame state.

The one exception to this are the translational states of various
barycentric frames. These frames are contained in the ephemeris models.
Assuming the environment simulation object is established as described
in section~\ref{sec:guide_sim_developers}, the Earth-Moon barycenter
frame is the simulation variable
\verb|env.de4xx.earth_moon_barycenter_frame|.
The solar system barycenter is also present in the JPL ephemeris model,
but its translational state is identically zero and its rotational
state is the identity rotation.

The recommended practice for logging these ephemerides data
in a Trick 10 simulation is to use the JEOD logger Python module.
The following assumes each planet is represented as a Trick10 simulation object
that contains a data member named \verb|planet| of type \verb|Planet|
and whose name is the name of the planet. (This is the recommended
practice for defining JEOD planetary objects in Trick10).
\begin{codeblock}
# Load the JEOD logger module.
import sys 
import os
sys.path.append ('/'.join([os.getenv("JEOD_HOME"), "lib/jeod/python"]))
import jeod_log

def log_trans_states (data_record, interval, file_suffix, \
                      planets, other_frames=()) :
  """
  Log the states of the specified reference frames.

  Arguments:
  data_record -- Trick data recorder sim object.
  interval    -- Simulation time between successive data blocks.
  file_suffix -- Suffix for the log file.
  planets     -- Comma separated list of names of sim objects that
  contain planets; the planet's translational states are logged.
  other_frames -- Comma separated list of names of additional reference
  frames whose translational states are to be logged.
  """

  # Create and initialize the logger.
  logger = jeod_log.Logger (data_record.drd)
  logger.open_group (interval, file_suffix)

  # Log the translational states for the specified planets
  for planet in planets :
     logger.log_ref_frame_trans_state (planet + ".planet.inertial")

  # Log the additional translational states
  for frame in other_frames :
    logger.log_ref_frame_trans_state (frame)

  # Close the logger.
  logger.close_group ()

  return
\end{codeblock}
Using the above Python function, the translational states of the
Sun, Earth, Moon, and Earth-Moon barycenter can be logged via
\begin{codeblock}
log_trans_states (data_record, interval, "state", \
                  ["sun", "earth", "moon"], \
                  ["env.de4xx.earth_moon_barycenter_frame"])
\end{codeblock}

Refer to the User Guide chapters of
\hypermodelrefinside{PLANET}{part}{user}
and \hypermodelrefinside{REFFRAMES}{part}{user} for details
on using the Planet and RefFrame classes.

\subsection{Accessing Additional Ephemerides}
\label{sec:guide_sim_users_accessing_additional_ephemerides}
Simulation users have on occasion express a desire to access ephemerides
for planets that are not represented in the simulation. One approach is
to add Trick10 simulation objects to the simulation that represents those
additional planetary bodies. Subscriptions can be created for these
extraneous bodies in the input file.

An alternative approach is to create those extra planets in the input file,
thereby avoiding altering the \Sdefine file just to access these auxiliary
data. The following illustrates this with the planets Uranus and Neptune.
\begin{codeblock}
# Create a planet and name it Uranus (note use of lower and upper case).
uranus = trick.Planet()
uranus.set_name("Uranus")

# Register the planet with JEOD and issue a subscription.
uranus.register_planet(dynamics.dyn_manager)
uranus.inertial.subscribe()

# Register the object with Trick.
trick.TMM_declare_ext_var_s(uranus, "Planet uranus")

# Now do the same for Neptune.
neptune = trick.Planet()
neptune.set_name("Neptune")
neptune.register_planet(dynamics.dyn_manager)
neptune.inertial.subscribe()
trick.TMM_declare_ext_var_s(neptune, "Planet neptune")
\end{codeblock}

The states of these additional parameters can be logged by changing
the call to the Python function \verb|log_trans_states| described
in section~\ref{sec:guide_sim_users_accessing_ephemerides} above.
\begin{codeblock}
log_trans_states (data_record, interval, "state", \
                  ["sun", "earth", "moon"], \
                  ["env.de4xx.earth_moon_barycenter_frame", \
                   "uranus.inertial", "neptune.inertial"])
\end{codeblock}


\section{Instructions for Simulation Developers}
\label{sec:guide_sim_developers}
This section describes how to create a Trick 10 simulation
that uses the \ModelDesc.


\subsection{The \texttt{env} Simulation Object}
JEOD provides a Trick10 \texttt{S\_module} file that defines the standard
JEOD \verb|env| simulation object. This Trick10 simulation object contains
an instance of the De4xxEphemeris class. Place the following in your
\texttt{\Sdefine} file, just after the dynamics manager simulation object:
\begin{codeblock}
#include "JEOD_S_modules/environment.sm"
\end{codeblock}
where \verb|JEOD_S_modules| is a symbolic link to
\verb|$JEOD_HOME/sims/shared/Trick10/S_modules|.


\subsection{The \texttt{S\_overrides.mk} File}
\label{sec:guide_s_overrides}

The Development Ephemeris data provided with JEOD is in the form
of C++ source files that are compiled into shared libraries. The DE4XX 
ephemeris model loads the symbols from the shared library as needed per DE4XX
version and by segments of time. These shared libraries are compiled outside of 
the standard model source code Trick build process. However, Trick provides a 
mechanism in the \texttt{S\_overrides.mk} file by which simulation developers
can augment the build process. JEOD provides the make rules and cmake commands 
necessary to build the shared libraries and link them to a convenient location 
within the standard Trick build directory. Developers need only include
\verb|$JEOD_HOME/bin/jeod/generic_S_overrides.mk| in their simulations' 
\texttt{S\_overrides.mk} file.
\begin{codeblock}
include $(JEOD_HOME)/bin/jeod/generic_S_overrides.mk
\end{codeblock}
In this version of JEOD, JPL Development Ephermis models DE405, DE421 and DE440
are all built for each versions' full range of dates. Only symbols associated 
with the requested date are loaded at runtime which keeps the memory usage low 
despite having the entire date range available.


\subsection{Using other DE4XX Models}
\label{sec:guide_de4xx}

The standard JEOD \verb|env| De4xxEphemeris model will default to DE405 version
inside its constructor. To use any other version (currently DE421 and DE440), 
simply set the model number in the input file for the simulation run.

\begin{codeblock}
env.de4xx.set_model_number(421)
env.de4xx.set_model_number(440)
\end{codeblock}



\subsection{Lunar Orientation}
\label{sec:guide_lunar_rnp}
The JPL Development Ephemeris module automatically updates lunar orientation
as a part of the ephemerides update process if the orientation of the Moon is
needed somewhere in the simulation. Nothing special needs to be done
with regard to lunar orientation if ephemerides are updated at the derivative
rate or if lunar orientation only needs to be updated along with all
other ephemerides.

If ephemerides are updated as scheduled jobs but you need lunar orientation at
the derivative rate (or at a faster scheduled rate) you will need to
add a derivative class or scheduled job to your \Sdefine file that calls
the DE4xx ephemeris model's \verb|propagate_lunar_rnp| function.
For example, adding the following job to the \verb|env| simulation object
will force lunar orientation to be propagated between updates.
\begin{codeblock}
   P0 ("derivative") env.de4xx.propagate_lunar_rnp ();
\end{codeblock}

\section{Instructions for Model Developers}
This section describes how to use the \ModelDesc
in \Cplusplus code.

\label{sec:guide_model_developers}
\subsection{Using the \ModelDesc}
The programmatic use of the \ModelDesc is for the most part limited
to those developers who are also extending some aspect of the model
or of some model that already uses the \ModelDesc.
The former usage is covered in the following section.
For the latter, refer the the instructions for model developers
in the model that is being extended.

\subsection{Extending the \ModelDesc}
The \verb|EphemerisInterface| class is the primary class that
external users may need to extend. Key methods that need to be
defined are described below.



\subsubsection{\texttt{initialize\_model}}
The \texttt{initialize\_model} function should be called early in the
initialization process.
The name \texttt{initialize\_model} is a suggested name
rather than a mandatory one;
this method is typically called directly from the
\Sdefine file. This method should\begin{itemize}
\item Perform model-specific initializations such as opening input files,
\item Register the model itself with the ephemeris manager,
\item Register with the ephemeris manager any reference frames that are
  contained directly in the model, and
\item Register with the ephemeris manager all ephemeris items that
  describe the data the model is capable of computing.
\end{itemize}
The order in which the various ephemeris models'
\texttt{initialize\_model} method can be significant.
Multiple models can register ephemeris items with the same name.
The first model to register such an item is the winner.
For example, suppose you are developing an ephemeris model of the Jovian
moons, with Jupiter's center of mass as the central frame for this model.
Your Jovian moon model should be registered after the JPL DE4xx model
is registered. If the DE4xx model is active, that model will be responsible for
updating Jupiter's planet-centered inertial frame. Your model will add
the Jovian moons to the overall ephemerides, and they will be correctly
connected to the rest of the solar system.

\subsubsection{\texttt{ephem\_initialize}}
The \texttt{ephem\_initialize} function is called by the
ephemeris manager's \texttt{initialize\_ephemerides} function,
presumably after all of the ephemeris models in the simulation have completed
their basic initialization and after all of the planets have registered
themselves and their reference frames with the ephemeris manager.
The subscription status of the ephemeris items is not yet known, but
the existence (or lack thereof) of the underlying reference frames is known.
For example, if your new ephemeris model can provide ephemerides for
Planet X, but there is not Planet X in the simulation, your
\texttt{initialize\_ephemerides} function should disable the
those capabilities related to modeling Planet X.

\subsubsection{\texttt{ephem\_activate}}
The \texttt{ephem\_activate} function is called by the
ephemeris manager's \texttt{activate\_ephemerides} function,
in reverse ephemeris model registration order.
This function in turn is called at initialization time and
when changes in subscribed frames indicate that the reference
frame tree needs to be rebuilt. The subscription status of the
ephemeris reference frames is known at this time. The primary
purpose of this function is to coordinate that known subscription status
with the activities that will occur later on in your model.

\subsubsection{\texttt{ephem\_build\_tree}}
The \texttt{ephem\_build\_tree} function is called by the
ephemeris manager's \texttt{activate\_ephemerides} function,
in normal ephemeris model registration order.
The function is called after all ephemeris models have been
activated.
This function should set the root frame of the reference frame tree
if the ephemeris model is responsible for that root node. All
non-root nodes should be then connected to the correct parent node.

\subsubsection{\texttt{ephem\_update}}
The \texttt{ephem\_update} function is called by the
ephemeris manager's \texttt{update\_ephemerides} function.
The ephemeris model should respond to this call by updating
each reference frame state for which it is responsible.
Each updated reference frame should be timestamped with the current
dynamic time.

\subsection{Adding a DE4XX model}
To add a new DE4XX version to JEOD, use the following steps:
\begin{enumerate}
  \item Set the JEOD\_HOME environment variable
  \item Download the DE4XX version from the JPL FTP server replacing XX with 
the desired version number. \newline
\href{ftp://ssd.jpl.nasa.gov/pub/eph/planets/ascii/de4XX}{ftp://ssd.jpl.nasa.gov/pub/eph/planets/ascii/de4XX} 
\begin{codeblock}
## When complete, the downloaded files will be 
## located at pub/eph/planets/ascii/de4XX
wget -m -nH ftp://ssd.jpl.nasa.gov//pub/eph/planets/ascii/de4XX
\end{codeblock}
  \item Move the ascii files to the jeod de4xx ascii data directory \newline
\verb|$JEOD_HOME/models/environment/ephemerides/de4xx_ephem/ascii_full_set|
\begin{codeblock}
mv pub/eph/planets/ascii/de4XX/* \
    $JEOD_HOME/models/environment/ephemerides/de4xx_ephem/ascii_full_set
\end{codeblock}
  \item Convert the ascii to C++ source files
\begin{codeblock}
cd $JEOD_HOME
make -f bin/jeod/makefile REGEN_DE4XX_DATA=1
\end{codeblock}
\end{enumerate}

This will parse the ascii files and generate the resulting \newline
\verb|$JEOD_HOME/models/environment/ephemerides/de4xx_ephem/data/data_src/de4XX.cc| file.

Once the \verb|de4XX.cc| file is generated, the shared library (\verb|libde4XX.so|) will be built as part 
of the simulation build process.

\section{Frequently Asked Questions}
\label{sec:guide_FAQ}
\begin{enumerate}
\item \emph{How do I access ephemeris model data?}\newline
In general, you don't. The primary job of any ephemeris model is to store
the requested ephemeris data as  states of the appropriate reference frame
objects.
As a user, you should access states from the reference frames in which
those data are stored rather than accessing them from some ephemeris model.

\item \emph{How do I control which ephemerides are computed?}\newline
You do this by subscribing to the reference frame that contains the desired
data. An ephemeris model updates the states of the active reference frames
that the model ``owns''. Reference frame activity is controlled by
subscription.

\item \emph{How do I access ephemeris model data for objects
that aren't in the simulation?}\newline
The easiest way to accomplish this is to via a Trick 10 simulation input file.
No modifications to the simulation are needed.
See section~\ref{sec:guide_sim_users_accessing_additional_ephemerides}
for details.

\item \emph{How do I use the DE421 or DE440 model?}\newline
\JEODid\ defaults to the DE405 model. Simply follow the instructions in
section~\ref{sec:guide_de4xx} for details.

\item \emph{How do I make the JPL ephemeris model cover a longer time span?}
\newline
If you are using the DE421 model, you can't. It covers July 29, 1899 to
October 9, 2053. If you are using the DE405 model,
override the default value of the \verb|EPHEM_YEARS| make variable
in your \verb|S_overrides.mk| file.
See section~\ref{sec:guide_s_overrides} for details.

\item \emph{How do I update the lunar orientation?}\newline
Lunar orientation is automatically updated by the JPL Development Ephemeris
module as a part of the ephemerides update process.
See section~\ref{sec:guide_lunar_rnp} for those cases where this
automated update is insufficient.
\end{enumerate}
