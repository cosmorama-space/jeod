
%%%%%%%%%%%%%%%%%%%%%%%%%%%%%%%%%%%%%%%%%%%%%%%%%%%%%%%%%%%%%%%%%%%%%%%%%%%%%%%%%
%
% Purpose:  Detailed part of Product Spec for the time model
%
% 
%
%%%%%%%%%%%%%%%%%%%%%%%%%%%%%%%%%%%%%%%%%%%%%%%%%%%%%%%%%%%%%%%%%%%%%%%%%%%%%%%%


\subsection{Process Architecture}
This section describes the flow from function to function.  The
operation of each function is described under its object in the section
\reftext{Functional Design}{ref:FunctionalDesign}.


\subsubsection{Overview of calling sequence (initialization)}

{\begin{enumerate}  %1.
\item \textbf{TimeManager::register\_time} or \textbf{TimeManager::register\_time\_named}\par
Stores the address of the manager in each time-type.  Adds the
type to the manager's list of time-types, which is
stored as a vector of pointers to each type.  Then stores the index
(position in the vector) of each type in each type. 

\item \textbf{TimeManager::register\_converter} \par
Adds the converter to the TimeManager's list of
converters.

\clearpage

\item \textbf{TimeManager::initialize}\par
Registers and initializes the dynamic time. Records the number
of time-types.
{\begin{enumerate}   %(a)

\item \textbf{TimeManagerInit::initialize\_manager}\par
Sequence of calls

{\begin{enumerate}  %i
\item \textbf{TimeManagerInit::initialize}\par
Records the dyn\_time index, and initializer index.

Allocates memory.

Initializes arrays.

{\begin{enumerate}  %A
\item \textbf{TimeManager::time\_lookup}\par
Look up the initializer type by name.
\item \textbf{TimeManagerInit::verify\_times\_setup} \par
Sanity checks.
\end{enumerate}}  %A 

\item \textbf{TimeManagerInit::populate\_converter\_registry}
{\footnote{{\bf Converter\_registry}\par
This is a registry that allows quick verification and lookup of
functions that convert from one time type to another.\par 
There are
2 converter registries, one for initialization, and one for runtime. 
Both take advantage of the C++ vector, but the initialization registry
is much larger than the runtime registry. \par 
The first
registry is used during the initialization process and contains 3
vectors that represent \textit{n }x \textit{n }arrays, where \textit{n}
is the number of time representations.  The converter from time
\textit{i }to type \textit{j }is associated with element \textit{(ni +
j) }of the vectors.\par 
{\begin{enumerate}
\item One vector \textit{(converter\_class\_ptr)} contains pointers to
the associated converter class.
\item A second vector \textit{(init\_converter\_dir\_table)} contains a
value -1, 0, or 1.  The \textit{init }in the name indicates that this
shows the direction in which the converter should be used during
\textit{initialization}.  The number indicates the direction (-1
reverse; 0 not available; 1 forward).
\item A third vector \textit{(update\_converter\_dir\_table) }does the
same for converters to be used during the simulation.
\end{enumerate}}
\par \par
The second registry is used at runtime and contains only 2
vectors, each with only \textit{n} elements:\par 
{\begin{enumerate}
\item update\_converter\_ptr.  Element\textit{ i }contains the pointer
to the converter class that is used to update time type \textit{i.}
\item update\_converter\_dir.  Element\textit{ i }contains -1, 0 or 1
indicating the direction in which the converter should be used.
\end{enumerate}}
}} \par

Stores pointers in the appropriate locations (a-to-b and b-to-a)
for each of the converters stored by the Time Manager.

Stores +1 / 0 / -1 in each location of the direction tables (+1
represents a-to-b, 0 indicates that the converter may not be used, -1
represents b-to-a).

\end{enumerate}}  %i

\item \textbf{TimeManagerInit::verify\_converter\_setup} \par

Comprises sanity checks for where potential conflicts could occur between
different converter classes.

{\begin{enumerate}  %i
\item \textbf{TimeManager::time\_lookup
}\textit{(optional)} \par
Look for the TAI, UTC, and UT1 time instances to verify correct setup.

\end{enumerate}}  %i

\clearpage

\item \textbf{TimeManagerInit::create\_init\_tree}
{\footnote{{\bf Initialization tree}\par 
Each time type has a record of how it gets initialized,
and one time type is designated as the
{\textquotedblleft}initializer{\textquotedblright}.  The
\textit{create\_init\_tree} function builds a tree starting with the
initializer, and working down to those types that are initialized based
on the value of the initializer, then those that are based on those
that are based on the initializer, etc.  If the user omits \textit{an
initialize\_from }specification for a particular type, this function
will attempt to identify a suitable initialization path from the
converter registry.  Structures that leave types isolated from the
initializer, or produce loops, will cause a termination.}} \par

Multi-pass function that repetitively calls
add\_type\_initialize on all time-types that are not in the tree until
everything is in the tree, or nothing else can be added.

{\begin{enumerate}   %i
\item \textbf{Time::add\_type\_initialize}\par
Uses the user-specified {\textquotedblleft}initialize\_from\_
name{\textquotedblright} values to place type. Where names are missing,
type is placed in the tree based on the availability of converter
functions.

{\begin{enumerate}   %-
\item \textbf{TimeManager::time\_lookup }\textit{(optional)} \par

Look up the time types by name.
\end{enumerate}}     % A

\end{enumerate}}  %i

\item \textbf{TimeManagerInit::initialize\_time\_types}
{\footnote{{\bf Initialize time types}\par
The function \textit{initialize\_time\_types} first
calls \textit{initialize\_initializer\_time} to set the values in the
time-type that resides at the top of the initialization tree, then
calls \textit{initialize\_from\_parent }on each un-initialized time
type.  \textit{initialize\_from\_parent }is a recursive function which
will call itself on the parent time-type if the parent is not yet
initialized.}}

Sequence of calls

{\begin{enumerate}   %i
\item \textbf{Time***::initialize\_initializer\_time}
{\footnote{{\bf Initialize initializer}\par
If dyn\_time is the only time class, it is the
initializer by default, and initializes to zero.  Otherwise, the
user-defined initializer time is assigned its initial
values.}} \textit{(optional)}

Initialize the time specified by the user as the
{\textquotedblleft}initializer{\textquotedblright} (this is at the head
of the initialization tree).

{\begin{enumerate}   %-
\item \textbf{Time***::convert\_to\_julian }\textit{(optional)}
\item[] If initialization data is in calendar format, this function converts it
to Truncated Julian format.
\end{enumerate}}     %-

\item \textbf{Time::initialize\_from\_parent}\par
Works down through the tree, initializing each time type from its parent
in the initialization tree.

{\begin{enumerate}   %-
\item \textbf{TimeConverter::initialize}
{\footnote{{\bf Initialize the converters}\par
To set the time from the parent, the converter
functions must be available.  Some of these may require some
initialization (e.g. the TAI to UTC converter requires finding the
appropriate values from data tables).  If the converter from parent to
child has not been initialized, that is completed before the
child's time is set as a call from
\textit{initialize\_from\_parent}.\par \ \ If the converter has not
been registered, the code will terminate.  This is redundant; if the
converter had not been registered, it could not have been used in
building the initialization tree.}}  \textit{(optional)}

Initializes any converter functions that are needed in the process.
\end{enumerate}}    %-
\end{enumerate}}  %i

\clearpage
\item \textbf{TimeManagerInit::create\_update\_tree}
{\footnote{{\bf Create the update tree}\par
The update tree is built in much the same way as
the initialization tree.  The top level of the update tree must be
TimeDyn, which is the time used for integration and is derived directly
from the simulation engine.  While the initialization process was
distributed over many functions, the assignment of the update process
is more compact, although the same checks are made that each time type
can {\textquotedblleft}see{\textquotedblright} TimeDyn, and that there
are no loops.}}

Sequence of calls
{\begin{enumerate}  %i
\item \textbf{Time::add\_type\_update}
{\footnote{{\bf Adding types to the
update list}\par
Recursive function.  Ensures that the parent type
is in the tree, or calls itself on the parent if not, then records the
appropriate converter functions in the runtime converter registry using
\textit{organize\_update\_list}}} \par

Repeats the tree-building exercise carried out in create\_init\_tree,
but uses the update side of the converter functionality and starts with
\textit{dyn-time} instead of {\textquotedblleft}initializer{\textquotedblright}.

{\begin{enumerate}  %-
\item \textbf{TimeManager::time\_lookup}  \textit{(optional)} \par
Look up the time types by name.

\item \textbf{TimeManager::organize\_update\_list}  \par
Generates the hierarchical list of time types in the order in which they
must be updated.  This is stored in TimeManager.

\item \textbf{TimeConverter::initialize} \textit{(optional)} \par
Called if the necessary time converter has not been called.

\end{enumerate}}  %-
\end{enumerate}}  %i
\end{enumerate}}  %(a)

\item \textbf{TimeManager::update}
{\footnote{{\bf Update all times}\par
Although all times have been set to their starting values, there
may be some small numerical differences that occur when comparing the
converters as used during initialization to those used during runtime. 
These potential differences will have negligible effect on an absolute
time reference, but may be noticeable with relative values.  Thus,
starting with one structure and switching to another should be avoided.
The initial times at the start of the simulation will be propagated
down from TimeDyn in exactly the same way that they will be during the
run.}} \par


Run a full update on all times with \textit{dyn-time} = 0.0.
\end{enumerate}}  %1

\clearpage

\subsubsection{Overview of calling sequence (runtime)}

{\begin{enumerate}  %1
\item \textbf{TimeManager::update}\par
Sequence of calls:

{\begin{enumerate}  %a
\item \textbf{TimeDyn::update}\par
Updates Dynamic Time first.

\item \textbf{Time***::update}\par
Call to the update function in each of the time classes.

{\begin{enumerate}  %i
\item \textbf{TimeConverter::convert\_a\_to\_b / convert\_b\_to\_a}
{\footnote{{\bf Converter function}\newline
The time class knows which converter class to use, and in which
direction to use it.}}   \par

Run the appropriate converter function.

\end{enumerate}}   %i

\item \textbf{TimeDyn::update\_offset}\par
If the scale factor,\textit{m}, has changed on dynamic time,
this function adjusts the offset, \textit{c}, of dynamic time,
\textit{d}, from simulation time, s, so that $d=ms+c$.
\end{enumerate}}  %a

\item \textbf{(TimeStandard::calendar\_update)}\ \ \textit{(optional)  }
Called on select derived times when a calendar-based output is desired.

\end{enumerate}}  %1


