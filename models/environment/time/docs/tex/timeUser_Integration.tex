%%%%%%%%%%%%%%%%%%%%%%%%%%%%%%%%%%%%%%%%%%%%%%%%%%%%%%%%%%%%%%%%%%%%%%%%%%%%%%%%%
%
% Purpose:  Integration part of User's Guide for the time model
%
% 
%
%%%%%%%%%%%%%%%%%%%%%%%%%%%%%%%%%%%%%%%%%%%%%%%%%%%%%%%%%%%%%%%%%%%%%%%%%%%%%%%%

 \section{Integration}\label{sec:Integration}

This section is intended for users who are incorporating the \timeDesc\ into a simulation.  It contains the basic requirements and step-by-step example of the construction of the \textit{Trick S\_define} file.  For more thorough examples, the verif directory in the standard release of \JEODid\ has a ``tutorial'' feel to it, stepping through from simple cases to more complex examples as it progresses from \textit{SIM\_1\_dyn\_only} through \textit{SIM\_5\_all\_inclusive}.


\subsection{S\_define requirements}
The time object in the S\_define is structured with the following
elements:


\begin{itemize}
\item Declare the following instances:

\begin{itemize}
\item TimeManager
\item TimeManagerInit
\item Desired time types (NOTE:  DynTime is automatically declared
within the TimeManager; declaring an instance of DynTime could cause a
failure when the code is confused about which version of DynTime to
use).
\item Desired time converters.  Note that converters may be
bidirectional, and that the code is sufficiently capable of
determining the appropriate direction.  This is important for the user
because the labels may be confusing; for example, the UTC to TAI
converter is labeled TAI\_UTC.  In the case that the converter is
desired in one direction at initialization, and the other during
simulation, IT IS NOT necessary to declare it twice.
\end{itemize}
\item Call the following initialization-class functions:

\begin{itemize}
\item The registration function for each of the declared time types to
register them with the Time Manager.  This function adds the time type
to the Manager's list, stores its location in that
list as a value in the time type itself, and stores the address of the
manager in each time type.
\item The registration functions for the declared time converters to
register them with the Time Manager.  This function adds the converter
to the Manager's list of available converters.
\item The Time Manager initialize function (NOTE -- this is a small
function that calls the main initialization functions in
TimeManagerInit).
\end{itemize}
\item Call the following run-time functions:

\begin{itemize}
\item The update function is called at any desired rate.  This will update
the decimal counter-type representation, \textbf{\textit{but not the
calendar representation}}, of all declared time-types.
\item \textit{(Optional) }the calendar-update function on any desired
time-types at any desired rate.  This function will convert the decimal
counter-type representation to a date::hr:min:sec representation.  Each
call to calendar\_update is type-specific and will update one,
\textbf{\textit{and only one,}} time-type.  Any call to the calendar-update will run the full decimal update method for all time-types if it has not already been done at that particular Simulator Time.  Consequently, the call to the regular update method is redundant if a calendar update is sequenced at the same rate.
\end{itemize}
\end{itemize}



\subsection{S\_define example}
\begin{enumerate}
\item First, declare the \textit{TimeManager} and \textit{TimeManagerInit} classes.  These
must be declared for all simulations.

\begin{verbatim}
 class TimeSimObject : public Trick::SimObject
 {
    jeod::TimeManager     manager;
    jeod::TimeManagerInit manager_init;

    ...

 };
\end{verbatim}



\item Next, in this example, I declare two additional time types, and two time converters
(there must be a converter from \textit{dyn}, and at least one per
declared time class.  In this case, the conversion path may be\textit{
dyn }to \textit{tai} to \textit{utc, }it will not be possible to go
from \textit{dyn }to \textit{utc }directly with these converters).

\begin{verbatim}
 class TimeSimObject : public Trick::SimObject
 {
    jeod::TimeManager     manager;
    jeod::TimeManagerInit manager_init;

    TimeTAI tai;
    TimeUTC utc;
    TimeConverter_Dyn_TAI  converter_dyn_tai;
    TimeConverter_TAI_UTC  converter_tai_utc;

    ...

 };
\end{verbatim}



\item Now I must register each of those types and their respective converters.
 The order in which this is done is not relevant; I chose to register
the time type, then immediately register the converter I plan to use for that time type; that order is just to keep track and ensure that I did not overlook
something.

\begin{verbatim}
 class TimeSimObject : public Trick::SimObject
 {
    jeod::TimeManager     manager;
    jeod::TimeManagerInit manager_init;

    TimeTAI tai;
    TimeUTC utc;
    TimeConverter_Dyn_TAI  converter_dyn_tai;
    TimeConverter_TAI_UTC  converter_tai_utc;

    TimeSimObject()
    {
        // Initialization jobs
        P_TIME ("initialization") manager.register_time( tai );
        P_TIME ("initialization") manager.register_converter( converter_dyn_tai );
        P_TIME ("initialization") manager.register_time( utc );
        P_TIME ("initialization") manager.register_converter( converter_tai_utc );

        ...

    }
 };
\end{verbatim}

\item Now initialize the manager.  This must be done for all configurations.

\begin{verbatim}
 class TimeSimObject : public Trick::SimObject
 {
    jeod::TimeManager     manager;
    jeod::TimeManagerInit manager_init;

    TimeTAI tai;
    TimeUTC utc;
    TimeConverter_Dyn_TAI  converter_dyn_tai;
    TimeConverter_TAI_UTC  converter_tai_utc;

    TimeSimObject()
    {
        // Initialization jobs
        P_TIME ("initialization") manager.register_time( tai );
        P_TIME ("initialization") manager.register_converter( converter_dyn_tai );
        P_TIME ("initialization") manager.register_time( utc );
        P_TIME ("initialization") manager.register_converter( converter_tai_utc );

        P_TIME ("initialization") manager.initialize( &manager_init );

        ...

    }
 };
\end{verbatim}

\item Next, schedule the regular updates.  This, also,  is required.  It will
update \textit{dyn}, \textit{tai} and \textit{utc}.

\begin{verbatim}
 class TimeSimObject : public Trick::SimObject
 {
    jeod::TimeManager     manager;
    jeod::TimeManagerInit manager_init;

    TimeTAI tai;
    TimeUTC utc;
    TimeConverter_Dyn_TAI  converter_dyn_tai;
    TimeConverter_TAI_UTC  converter_tai_utc;

    TimeSimObject()
    {
        // Initialization jobs
        P_TIME ("initialization") manager.register_time( tai );
        P_TIME ("initialization") manager.register_converter( converter_dyn_tai );
        P_TIME ("initialization") manager.register_time( utc );
        P_TIME ("initialization") manager.register_converter( converter_tai_utc );

        P_TIME ("initialization") manager.initialize( &manager_init );

        // Scheduled Jobs
        (DYNAMICS, "environment") manager.update( exec_get_sim_time() );
        ...
    }
 };
\end{verbatim}

\item Finally, I decided that the output from \textit{UTC} would be more
useful for my particular purpose if it were expressed as a calendar
format rather than a decimal format.  Here, I run the calendar update
at the same rate as the regular update, so that the calendar format is
always up-to-date with the decimal format.  Otherwise, the two values
may represent different times.

\begin{verbatim}
 class TimeSimObject : public Trick::SimObject
 {
    jeod::TimeManager     manager;
    jeod::TimeManagerInit manager_init;

    TimeTAI tai;
    TimeUTC utc;
    TimeConverter_Dyn_TAI  converter_dyn_tai;
    TimeConverter_TAI_UTC  converter_tai_utc;

    TimeSimObject()
    {
        // Initialization jobs
        P_TIME ("initialization") manager.register_time( tai );
        P_TIME ("initialization") manager.register_converter( converter_dyn_tai );
        P_TIME ("initialization") manager.register_time( utc );
        P_TIME ("initialization") manager.register_converter( converter_tai_utc );

        P_TIME ("initialization") manager.initialize( &manager_init );

        // Scheduled Jobs
        (DYNAMICS, "environment") manager.update( exec_get_sim_time() );
        (DYNAMICS, "environment") utc.calendar_update( exec_get_sim_time() );
    }
 };
\end{verbatim}

\end{enumerate}



\subsection{Updating the Data Tables}\label{ref:data_update}

Default data is provided for the offset between UTC and UT1, on a day-by-day basis,
from 1962 to the time just prior to to current JEOD version release date.  Data for the offset between TAI and UTC (the number of leap seconds) is also provided.  Because the UTC-UT1 offset changes continuously, this can quickly become outdated; the TAI-UTC data can also become outdated, but this is only updated once every 6 months so is not such a problem.
If the data available needs to be brought up-to-date, new data can be obtained from the International Earth Rotation and Reference Systems Service (IERS) webiste \href{http://www.iers.org}{http://www.iers.org}, which has links to Data/Products and Earth Orientation Data.

 If the user chooses to obtain their own data, we suggest the following steps:
 \begin{enumerate}

 \item Download the latest EOP 14 C04 (IAU2000) data file from the IERS at:

 \href{https://datacenter.iers.org/data/latestVersion/EOP_14_C04_IAU2000A_yearly_files.txt}
 {https://datacenter.iers.org/data/latestVersion/EOP\_14\_C04\_IAU2000A\_yearly\_files.txt}

 remove the file extension (\textit{.txt}) from the file name and save it to the data
 directory (\textit{models/environment/time/data}).

 \item OPTIONAL: Edit the data file by removing lines until data from only the desired
 time span remain in the file.  The initial 14 header lines must NOT be removed.

 \item Check the most recent leap second information, available in Bulletin C,
 (which is also linked from the IERS website), or from numerous other readily
 available sources.  Bulletin C is published every 6 months, and announces the
 current difference between UTC and TAI (i.e., the number of leap seconds).

 In the data directory (\textit{models/environment/time/data}) is a Perl script,
 \textit{parser.pl}.  This script contains a list of when all of the leap seconds
 were implemented.  Check the current leap second number against the final entry
 in that script, currently ending with the line
 \begin{verbatim}
    [ 2457754.5 , 37.0 ]  	 # 2017 JAN  1
 \end{verbatim}
 (In this case, the final value is 37 seconds, implemented at 0:00:00 hours on
 Jan 1, 2017, which has a Julian Date of 2457754.5)

 If Bulletin C has a UTC to TAI difference that does \textit{not} match the final entry
 in this table, the table should be updated by adding the Julian Date at last
 change and the new corresponding value, to the end of the table in the script.
 \item In the data directory (\textit{environment/time/data}), run the parser
 script with the command

 \begin {verbatim}
 perl parser.pl filename
 \end{verbatim}

 where filename is the previously saved DUT1 data file (i.e., eopc04\_14\_IAU2000.62-now).
 \item Verify that whichever of the two files \textit{tai\_to\_utc.cc}
 and \textit{tai\_to\_ut1.cc} are needed have been written appropriately.
 \end{enumerate}

\subsection{Overriding the Data Tables}\label{ref:data_override}
In some situations, the user may wish to specify a particular value in place of
using the data look-up.  This may be particularly relevant when developing
historical or futuristic missions for which data is not available, or for
carrying out a simulation using very recent data without going to the trouble
of updating the data tables.  However, this is generally not recommended because
it locks UT1 and/or UTC to TAI, and prevents the automatic updates that enhance
the clock precision.

First, both the TAI-UTC and TAI-UT1 converter classes have a flag
titled \textit{override\_data\_table} that defaults to \textit{false}.  This
flag must be reset to \textit{true} to enforce the override data.

Next, both classes have a variable to accommodate the desired value.  These are
titled \textit{leap\_sec\_override\_val} and \textit{tai\_to\_ut1\_override\_val}
for classes TimeConverter\_TAI\_UTC and TimeConverter\_TAI\_UT1 respectively.

All of these values can be set in an input file, with entries such as:
\begin{verbatim}
 time.tai_ut1.override_data_table = true
 time.tai_ut1.tai_to_ut1_override_val = -32.469
\end{verbatim}

Note that when overriding the data for generating UT1, the variable
is \textit{tai\_to\_ut1\_override\_val}, i.e the value for converting from TAI
to UT1.  This is NOT the same as DUT1, which is used for converting from UTC to
UT1.  To generate the necessary value, it is also necessary to subtract off from
DUT1 the number of leap seconds (the number of leap seconds provides the
conversion from UTC to TAI).

When overriding the data for generating UTC, the number of leap seconds is used,
and is (at least as of the publication of this document) a positive number.

The astute reader may have noticed, and be contemplating why, there is symmetry
in the setting of the flag (tai\_utc and tai\_ut1 have their respective flags
named identically), but not in the setting of the value.  The explanation, just
for the sake of curiosity, is as follows:  for UT1, the data value is simply
named for the converter class in which it resides, \textit{TimeConverter\_TAI\_UT1},
so the data represents the conversion from TAI to UT1; for UTC, the same
notation could have been confusing, because the value of the TAI to UTC
converter is the negative number of leap seconds.  We felt that having the 
negative sign would cause more issues than it solved, and so renamed it to
simply represent the number of leap seconds, or the UTC to TAI conversion.
