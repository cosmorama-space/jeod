%%%%%%%%%%%%%%%%%%%%%%%%%%%%%%%%%%%%%%%%%%%%%%%%%%%%%%%%%%%%%%%%%%%%%%%%%%%%%%%%%
%
% Purpose:  Introduction for the time model.
%
% 
%
%%%%%%%%%%%%%%%%%%%%%%%%%%%%%%%%%%%%%%%%%%%%%%%%%%%%%%%%%%%%%%%%%%%%%%%%%%%%%%%%
%\section{Purpose and Objectives of \timeDesc}
% Incorporate the intro paragraph that used to begin this Chapter here. 
% This is location of the true introduction where you explain what this model 
% does.
% This file was converted to LaTeX by Writer2LaTeX ver. 1.0 beta3
% see http://writer2latex.sourceforge.net for more info
The \timeDesc\ in \JEODid\ represents a
complete reformulation from the UTIME structure seen in JEOD1.x.  Like
its predecessor, it is responsible for tracking time as a simulation
advances; unlike its predecessor, it can do so in any number of time 
representations, using any number of clocks, that may be ticking at
different rates from one another.


While the SI second is a well-defined quantity, counted to high 
precision by atomic clocks, and incorporated into many physical
constants, it is not always the best clock to use for modeling
purposes.  For example, when modeling the rotation of Earth, it is
preferable to use a sidereal clock (1 day = 1 rotation), rather than
the synodic clock (1 day = 1 solar passage) associated with the SI
second.  The length of the SI second was chosen such that there would
be approximately 86,400 seconds in one synodic day, and the passage of
the SI second is represented by International Atomic Time (TAI).
 However, the length of the synodic day, now associated with Universal
Time (UT1), rarely lasts 86,400 SI seconds, and fluctuates with little
predictability.  Earth-based civil-time, or Universal Coordinated
Time (UTC), attempts to closely match the passage of UT1, while ticking 
at the same rate as TAI.  


Further complicating the management of time are other clocks with
various epochs (points in time where their value was 0). 
 Mission-Elapsed-Time (MET) is often used to track time since a 
mission began, and the Global Positioning System (GPS) has its own clock.
Users may wish to run only in local time, rather than UTC, or 
have clocks that can output their data in two or more local times for
international missions.

In JEOD 1.x, there was a tacit assumption that time always advanced
forward.  In this implementation, we can run time in reverse to find,
for example, a set of initial conditions that produce a desired
end-state.  As we look toward making JEOD applicable to
interplanetary missions, it may be desirable to incorporate a
solar-based timescale, with capability for relativistic corrections;
that capability has also been implemented.












