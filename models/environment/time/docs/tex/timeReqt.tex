%%%%%%%%%%%%%%%%%%%%%%%%%%%%%%%%%%%%%%%%%%%%%%%%%%%%%%%%%%%%%%%%%%%%%%%%%%%%%%%%%
%
% Purpose:  requirements for the time model
%
% 
%
%%%%%%%%%%%%%%%%%%%%%%%%%%%%%%%%%%%%%%%%%%%%%%%%%%%%%%%%%%%%%%%%%%%%%%%%%%%%%%%%

% add text here to describe general model requirements
% text is of the form:
%\requirement{REQUIREMENT DESCRIPTION}
%\label{reqt:REQT_DESC_ABBREVIATED}
%\begin{description}
%  \item[Requirement:]\ \newline
%     <Insert description of requirement> 
%  \item[Rationale:]\ \newline
%     <Insert description of rationale> 
%  \item[Verification:]\ \newline
%     <Insert description of verification (e.g. "Inspection")> 
%\end{description}

\requirement{Physical Time}
\label{reqt:physicaltime}
\begin{description}
\item[Requirement:]\ \newline
The \timeDesc\ shall include a representation of time that can be used for 
integration of the dynamic vehicle state.  It shall be capable of propagating 
forward, and in reverse.
\item[Rationale:]\ \newline
Not all time representations accurately represent the standard definition of 
``second'', which is required for the integration of the physical state.  There 
should be one clearly identified representation that is always used for 
integration purposes.  
To identify a suitable state that is capable of creating a particular outcome, 
it is necessary to be able to integrate from the outcome back in time to the 
original state.

\item[Verification:]\ \newline
Test \vref{test:1dynonly}, test \vref{test:sim2scalefactor}, and test 
\vref{test:timereversal}.
\end{description}

\requirement{Necessary Clocks}
\label{reqt:datatimerepresentation}
\begin{description}
  \item[Requirement:]\ \newline
    The \timeDesc\ shall be able to represent time in
    each of the following time systems:

    \subrequirement{TAI}\label{reqt:data_time_rep_TAI}
      \ \newline
      International Atomic Time,
      a very accurate and stable time scale calculated as a weighted 
      average of the time kept by about 200 cesium atomic clocks in 
      over 50 national laboratories worldwide.

    \subrequirement{UT1}\label{reqt:data_time_rep_UT1}
      \ \newline
      Universal Time, a measure of the rotation angle of the Earth 
      as observed astronomically.

    \subrequirement{UTC}\label{reqt:data_time_rep_UTC}
      \ \newline
      Coordinated Universal Time, the basis for the worldwide system 
      of civil time.

    \subrequirement{MET}\label{reqt:data_time_rep_MET}
      \ \newline
      Any number of Mission Elapsed Times, each with distinct user-defined 
      epochs.

    \subrequirement{GPS}\label{reqt:data_time_rep_GPS}
      \ \newline
      The time system used by the Global Positioning System.
		
    \subrequirement{TT}\label{reqt:data_time_rep_TT}
      \ \newline
      Terrestrial Time, the time system used by the Ephemeris models.
		
    \subrequirement{GMST}\label{reqt:data_time_rep_GMST}
      \ \newline
      Greenwich Mean Sidereal Time, the time system used to simulate the
			rotation of Earth.
		
    \subrequirement{TDB}\label{reqt:data_time_rep_TDB}
      \ \newline
      Barycentric Dynamic Time, potentially useful for higher fidelity ephemeris
			models, or for inner-solar-system mission.
		
    \subrequirement{UDE}\label{reqt:data_time_rep_UDE}
      \ \newline
      Any number of clocks each set to tick with any of the other clocks, but 
      each with a distinct user-defined epoch.

			
		

  \item[Rationale:]\ \newline
    The purpose of the time module is to represent 
    time in the multiplicity of time scales that are expected to be 
    needed by various simulation developers.

  \item[Verification:]\ \newline
    Test \vref{test:clockpresence}.
\end{description}

\requirement{Necessary Representation Formats}
\label{reqt:formattimerepresentation}
\begin{description}
  \item[Requirement:]\ \newline
    The \timeDesc\ shall be able to represent time in the 
    each of the following formats:

    \subrequirement{Julian Date}
    \subrequirement{Modified Julian}  
    \subrequirement{Truncated Julian Time}
    \subrequirement{Seconds since epoch}
    \subrequirement{Calendar / Clock}\ \newline
      Provide a calendar where a calendar is defined, including year, month, 
      day, hour, minute, second.
      
      Provide just a clock (comprising day, hour, minute, second) where a 
      calendar is not defined.
  \item[Rationale:]\ \newline
    Different models require different formats by which time is input to their 
    respective methods.  This is the compilation of the formats determined to 
    be the most useful for external models.  A request to add centuries since 
    epoch was deferred due to the ambiguous definition of `century' (being 100 
    years, or 36,525 days).  Instead, days since epoch was added, this can be 
    utilized to obtain centuries since epoch using whatever definition of 
    `century' the model requires.
  \item[Verification:]\ \newline
    Test \vref{test:STDinitbyval}, test \vref{test:STDinitbycal}, and test 
    \vref{test:UDEinitbyval}.
\end{description}


\requirement{Time Initialization by Representation}
\label{reqt:timeinitializationrep}
\begin{description}
  \item[Requirement:]\ \newline
    The \timeDesc\ shall be capable of initializing 
    time using any of the following time representations: 
		\begin{itemize}
		\item GPS
		\item MET
		\item TAI
		\item TDB
		\item TT
		\item UDE
		\item UT1
		\item UTC
		\end{itemize}

  \item[Rationale:]\ \newline
    The purpose of the time module is to represent 
    time in the multiplicity of time scales that are expected to be 
    needed by various simulation developers; this implies the internal ability
    to convert easily from one to another.

  \item[Verification:]\ \newline
    Test \vref{test:UDEinitbytdb}, test \vref{test:STDinitbytype} and \vref{test:UDEinitbyval}.
\end{description}


\requirement{Time Initialization by Format}
\label{reqt:timeinitializationbyformat}
\begin{description}
  \item[Requirement:]\ \newline
    The \timeDesc\ shall be capable of initializing 
    time using any one of the following formats:

		\subrequirement{Calendar} 
		\label{reqt:func_time_initialization_format_cal} \newline
		Using a calendar-based format yyyy/mm/dd::hh:mm:ss when such a 
		format is well defined in the appropriate time representation

    \subrequirement{Clock} 
		\label{reqt:func_time_initialization_format_clock} \newline
		Using a clock-based format dd::hh:mm:ss when a larger calendar 
		format is not well defined.

		\subrequirement{Truncated Julian Time} 
		\label{reqt:func_time_initialization_format_tjt} \newline
		Days since the NASA Truncated Julian Time epoch.

		\subrequirement{Modified Julian Time} 
		\label{reqt:func_time_initialization_format_mjt} \newline
		Days since the Modified Julian Time epoch.

		\subrequirement{Julian Time} 
		\label{reqt:func_time_initialization_format_jt} \newline
		Days since the Julian Time epoch.

		\subrequirement{Seconds since epoch} 
		\label{reqt:func_time_initialization_format_s2k} \newline
		Seconds since the appropriate epoch (e.g. J2000, or a 
		User-defined epoch).
		
		\subrequirement{Days since epoch} 
		\label{reqt:func_time_initialization_format_dUDE} \newline
		Days since the appropriate epoch (e.g. J2000, or a User-defined 
		epoch).
		
  \item[Rationale:]\ \newline
    While conversion between these types is algorithmic, it can be time 
    consuming and
		potentially prone to error.  The provision of these 
		capabilities allows the
		user to focus on getting the simulation dynamics correctly 
		configured without worrying about
		whether some known time has been correctly converted into a 
		usable input.

  \item[Verification:]\ \newline
    Test \vref{test:STDinitbyval}, test \vref{test:STDinitbycal}, and test 
    \vref{test:UDEinitbyval}.
\end{description}


\requirement{Time Initialization Bypass}
\label{reqt:timeinitializationbypass}
\begin{description}
  \item[Requirement:]\ \newline
    The \timeDesc\ must be capable of bypassing the initialization of
		time-types.
  \item[Rationale:]\ \newline
    There are applications where no fixed reference time is necessary, and
		arbitrarily inputting some time is, at best, redundant.
  \item[Verification:]\ \newline
	  Test \vref{test:1dynonly}
\end{description}


\requirement{Initialization of User-Defined-Epoch times}
\label{reqt:timeinitializationUDE}
\begin{description}
  \item[Requirement:]\ \newline
    The \timeDesc\ must be capable of initializing User-Defined-Epoch times
		(such as Mission Elapsed Time) in the following ways:
		\subrequirement{UDE value and fixed-epoch value known}
		\label{reqt:func_time_initialization_UDE_FE} \newline
		  The simulation is initialized relative to some fixed epoch, 
		  and the value
			of the UDE time is known at the start of the 
			simulation.  The UDE epoch
			can be determined.
		\subrequirement{UDE epoch and fixed-epoch value known}
		\label{reqt:func_time_initialization_UDEe_FE} \newline
		  The simulation is initialized relative to some fixed epoch, 
		  and the value
			of the UDE epoch is known relative to some (other) 
			fixed epoch time.  The
			UDE initial value	can be determined.
		\subrequirement{UDE value and UDE epoch known}
		\label{reqt:func_time_initialization_UDE_UDEe} \newline
		  The value of the UDE time is known at the start of the 
		  simulation, and the
			UDE epoch is known relative to some fixed-epoch time 
			representation (i.e. a Standard Time).  
			The values for some or all of the Standard Time 
			representations can then be determined.
			
  \item[Rationale:]\ \newline
    While conversion between these types is algorithmic, it can be time 
    consuming and potentially prone to error.  The provision of these 
    capabilities allows the user to focus on getting the simulation dynamics 
    correctly configured without worrying about whether some known time has 
    been converted into a usable input correctly.
  \item[Verification:]\ \newline
	  Test \vref{test:clockpresence} and test \vref{test:UDEinitbyval}.
\end{description}



\requirement{Ability to Overwrite Data Tables}
\label{reqt:backwardcompatibility}
\begin{description}
  \item[Requirement:]\ \newline
    The \timeDesc\ requires data tables to tabulate the occurrence of 
    leap seconds (UTC), and the time-dependent offsets (UT1).  These data must 
    be provided, along with the capability to overwrite the official data.
  \item[Rationale:]\ \newline
    Data can only be provided for periods in which it is available.  Neither 
    data table has future predictive capabilities; users wishing to run 
    simulations set in some future time may need the capability to assign a 
    best-estimate to these values.  Furthermore, any users wishing to test 
    compatibility with releases of JEOD pre-dating the JEOD 2.0.0 release will 
    need to turn off the update functionality of these tables, and overwrite 
    with some particular value.
  \item[Verification:]\ \newline
	  Test \vref{test:overrides}.
\end{description}


\requirement{Extensibility}
\label{reqt:extensibility}
\begin{description}
  \item[Requirement:]\ \newline
    The \timeDesc\ shall provide the capability of extension to permit the user
    to develop clocks of specific interest that will operate within the time
    structure.
  \item[Rationale:]\ \newline
   While every effort has been made to ensure that the time coverage is as
   complete as possible, we cannot imagine every potential application or
   situation in which the \timeDesc\ will be utilized. 
  \item[Verification:]\ \newline
	  Test \vref{test:sim6}.
\end{description}
