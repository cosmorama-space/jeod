%%%%%%%%%%%%%%%%%%%%%%%%%%%%%%%%%%%%%%%%%%%%%%%%%%%%%%%%%%%%%%%%%%%%%%%%%%%%%%%%
% reqt.tex
% Requirements on the Dynamics Manager Model
%
%
%%%%%%%%%%%%%%%%%%%%%%%%%%%%%%%%%%%%%%%%%%%%%%%%%%%%%%%%%%%%%%%%%%%%%%%%%%%%%%%%

%----------------------------------
\chapter{Product Requirements}\hyperdef{part}{reqt}{}
\label{ch:reqt}
%----------------------------------

\requirement{Top-level Requirement}
\label{reqt:toplevel}
\begin{description}
\item[Requirement:]\ \newline
  This model shall meet the JEOD project requirements specified in
  the \JEODid\
  \hyperref{file:\JEODHOME/docs/JEOD.pdf}{part1}{reqt}{ top-level
  document}.
\item[Rationale:]\ \newline
  This model shall, at a minimum,  meet all external and
  internal requirements 
  applied to the \JEODid\ release.
\item[Verification:]\ \newline
  Inspection 
\end{description}


\requirement{Registration and Lookup Services}
\label{reqt:lists}
\begin{description}
\item[Requirement:]\ \newline
  The model shall provide the ability to register and lookup by name
  the following kinds of objects:
  \subrequirement{Planets.}
  \label{reqt:planet_list}
  The model shall provide the ability to register and lookup objects
  of the class Planet
  (see \hypermodelref{PLANET}).
  \subrequirement{Ephemeris items.}
  \label{reqt:ephem_item_list}
  The model shall provide the ability to register and lookup objects
  of the class EphemerisItem
  (see \hypermodelref{EPHEMERIDES})
  and shall provide the ability to restrict the lookup to the
  JEOD-provided classes that derive from EphemerisItem. 
  \subrequirement{Reference frames.}
  \label{reqt:ref_frame_list}
  The model shall provide the ability to register and lookup objects
  of the type RefFrame
  (see \hypermodelref{REFFRAMES})
  and shall provide the ability to restrict the lookup to the
  reference frames registered as integration frames
  (inertial frames or non-rotating frames that are subject to gravitation
  forces only).
  \subrequirement{Mass bodies.}
  \label{reqt:mass_body_list}
  The model shall provide the ability to register and lookup objects
  of the type MassBody
  (see \hypermodelref{MASS})
  and shall provide the ability to restrict the lookup to the
  the class DynBody
  (see \hypermodelref{DYNBODY}).
\item[Rationale:]\ \newline
  The registration and lookup services are one of the driving
  requirements for the model.
\item[Verification:]\ \newline
  Inspection, Test
\end{description}


\requirement{Mode}
\label{reqt:mode}
\begin{description}
\item[Requirement:]\ \newline
  The model shall provide the ability to operate in one of three modes:
  \subrequirement{Empty space mode.}
  \label{reqt:empty_space_mode}
  Vehicles fly as if they are in empty space; the sun and planets are not
  present. Ephemerides models are inactive.
  \subrequirement{Single planet mode.}
  \label{reqt:single_planet_mode}
  Vehicles fly around a single planet. The sun and other planets are not
  present. Ephemerides models are inactive.
  \subrequirement{Ephemeris mode.}
  \label{reqt:ephemeris_mode}
  Vehicles fly in a system with multiple gravitational bodies.
  Ephemerides models are active and define the basis of the reference
  frame tree.
\item[Rationale:]\ \newline
  Simple problems should be simple to set up.
\item[Verification:]\ \newline
  Inspection, Test
\end{description}

\requirement[Ephemerides operations]{Ephemerides Initialization/Reinitialization}
\label{reqt:ephem}
\begin{description}
\item[Requirement:]\ \newline
  When operating in ephemeris mode,
  the model shall initialize the registered ephemerides models
  and shall provide these models with the ability to construct
  the base of the reference frame tree.
\item[Rationale:]\ \newline
  The task of creating the base of the tree belongs to the ephemerides models.
  This model's task is to provide a bridge between the ephemerides and
  reference frame models.
\item[Verification:]\ \newline
  Inspection, Test
\end{description}

\requirement[Ephemerides-free operations]{Ephemerides-Free Operations}
\label{reqt:ephem_free}
\begin{description}
\item[Requirement:]\ \newline
  When operating in empty space or single planet mode, this
  model shall define the base of the reference frame tree.
\item[Rationale:]\ \newline
  The ephemerides models are disabled (and may not be present)
  in these simple modes.
  The base of the reference frame tree still needs construction.
\item[Verification:]\ \newline
  Inspection, Test
\end{description}

\requirement{Body Actions}
\label{reqt:body_actions}
\begin{description}
\item[Requirement:]\ \newline
  In conjunction with the \hypermodelref{BODYACTION},
  this model shall provide the ability to initialize and asynchronously
  modify aspects of a mass or dynamic body.
  \subrequirement{Action Registration.}
  \label{reqt:add_body_action}
  The model shall provide the ability to register a new body actions that is
  to be performed at the appropriate time.
  \subrequirement{Action Initialization.}
  \label{reqt:initialize_body_action}
  The model shall initialize each registered body actions at the
  appropriate time and shall initialize each registered action once.
  \subrequirement{Action Application.}
  \label{reqt:apply_body_action}
  The model shall apply registered body actions at the appropriate time.
  Once applied the action shall be removed from the registered set of actions.
\item[Rationale:]\ \newline
  The Body Action Model and its interactions with other models,
  including this one, is one of the key innovations of JEOD 2.0.
\item[Verification:]\ \newline
  Inspection, Test
\end{description}

\requirement{Initialization Body Actions}
\label{reqt:init_time_body_actions}
\begin{description}
\item[Requirement:]\ \newline
  \subrequirement{Initialization Classes.}
  \label{reqt:init_time_body_action_classes}
  The model shall initialize and apply registered BodyAction instances that
  derive from the MassBodyInit, MassBodyAttach, and DynBodyInit classes
  at simulation initialization time.
  \subrequirement{Processing Order.}
  \label{reqt:init_time_body_actions_order}
  The model shall be able to properly initialize and apply a set of
  non-conflicting initialization actions regardless of the order in
  which the actions were registered.
\item[Rationale:]\ \newline
  The Body Action Model, along with its interactions with other models,
  was introduced primarily to solve the initialization problem.
\item[Verification:]\ \newline
  Inspection, Test
\end{description}

\requirement{Asynchronous Body Actions}
\label{reqt:asynch_body_actions}
\begin{description}
\item[Requirement:]\ \newline
  The model shall provide the ability to initialize and apply registered
  BodyAction instances during the course of a simulation run.
\item[Rationale:]\ \newline
  This is a nice side benefit of the model.
\item[Verification:]\ \newline
  Inspection, Test
\end{description}

\requirement{Common Integration Scheme} 
\label{reqt:common_integ}
\begin{description}
\item[Requirement:]\ \newline
  The model shall provide the ability to use the same integration scheme
  for all registered dynamic bodies.
\item[Rationale:]\ \newline
  Unintentionally using different integration techniques introduces
  subtle errors into a simulation.
\item[Note:]\ \newline
  Some users intentionally want to use different integration techniques or
  integration rates for different vehicles.
  The JEOD 2.0 implementation does not support this.
  Support for this capability is planned for JEOD 2.1.
\item[Verification:]\ \newline
  Inspection, Test
\end{description}

\requirement{State and Time Propagation} 
\label{reqt:integration}
\begin{description}
\item[Requirement:]\ \newline
  The model shall manage the propagation of the states of the registered
  dynamic bodies and of time from one dynamic time step to the next.
\item[Rationale:]\ \newline
  Coordinating the integration of state and time eliminates several
  subtle simulation errors.
\item[Verification:]\ \newline
  Inspection, Test
\end{description}


\requirement{Gravitation}
\label{reqt:gravitation}
\begin{description}
\item[Requirement:]\ \newline
  The model shall provide the ability to compute the gravitational
  accelerations acting on all registered dynamic bodies.
\item[Rationale:]\ \newline
  Computing gravitational acceleration is an
  essential part of state propagation.
\item[Verification:]\ \newline
  Inspection, Test
\end{description}
