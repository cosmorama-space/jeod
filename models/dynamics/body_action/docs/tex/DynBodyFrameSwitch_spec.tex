%%%%%%%%%%%%%%%%%%%%%%%%%%%%%%%%%%%%%%%%%%%%%%%%%%%%%%%%%%%%%%%%%%%%%%%%%%%%%%%%
% DynBodyFrameSwitch_spec.tex
% Specification of the DynBodyFrameSwitch class
%
%%%%%%%%%%%%%%%%%%%%%%%%%%%%%%%%%%%%%%%%%%%%%%%%%%%%%%%%%%%%%%%%%%%%%%%%%%%%%%%%

\chapter{Product Specification}\label{ch:\modelpartid:spec}

\section{Conceptual Design}
The DynBodyFrameSwitch class switches the frame in which the subject DynBody
object's state is represented and propagated. This class extends the
base BodyAction class. Frame switching occurs when the subject DynBody
transitions between spheres of influence.

The class provides two modes of operation:
\begin{description}
\item[SwitchOnApproach] The subject body's integration frame is switched when
  the body enters the sphere of influence of the new integration frame.
\item[SwitchOnDeparture] The subject body's integration frame is switched when
  the body leaves the sphere of influence of the current integration frame.
\end{description}

The class uses a user-provided switch distance to determine whether the
body has entered the sphere of influence of the new integration frame
or left the sphere of influence of the current integration frame.

\section{Mathematical Formulations}
Two commonly used concepts to describe the gravitational sphere of influence
of a smaller gravitational body orbiting a larger gravitational body are
the Hill sphere\cite{BodyAction:Valtonen} and
the Laplace sphere of influence\cite{BodyAction:Brown}.

The Hill sphere is

\begin{equation}
r_{\text{Hill}} \approx a  \left(\frac{m}{3M}\right)^{1/3}
\end{equation}

The Laplace sphere of influence is

\begin{equation}
r_{\text{SOI}} \approx a  \left(\frac{m}{M}\right)^{2/5}
\end{equation}

Table~\ref{tab:DynBodyFrameSwitch:SOI} lists the radii of the
Hill and Lagrange spheres of influence for a few of the pairs of
solar system bodies. These values are supplied for the use by the user.
The DynBodyFrameSwitch class uses a user-defined value to test transitioning
between spheres of influence.

\begin{table}[htp]
\centering
\caption{Spheres of Influence}
\label{tab:DynBodyFrameSwitch:SOI}
\vspace{1ex}
\begin{tabular}{||c||rl|rl|} \hline
{\bf Bodies} &
\multicolumn{2}{|c|}{\bf Hill} &
\multicolumn{2}{|c|}{\bf Laplace} \\ \hline \hline
Earth/Moon   &    $61,500$ & km &  $66,200$ & km  \\
Sun/Earth    & $1,496,700$ & km & $924,700$ & km  \\
Sun/Mars     & $1,084,110$ & km & $577,300$ & km  \\ \hline
\end{tabular}
\end{table}

\section{Detailed Design}
This is a very simple sub-model.
The {\tt initialize()} method simply checks for errors and subscribes to the
target integration frame. The {\tt is\_ready()} method checks for the
body entered the sphere of influence of the new integration frame or leaving
the sphere of influence of the current integration frame based on the
user-provided mode and switch distance. The {\tt apply()} method invokes
the DynBody {\tt switch\_integration\_frames()} method to perform
the frame switch.
