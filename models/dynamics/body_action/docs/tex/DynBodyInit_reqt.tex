%%%%%%%%%%%%%%%%%%%%%%%%%%%%%%%%%%%%%%%%%%%%%%%%%%%%%%%%%%%%%%%%%%%%%%%%%%%%%%%%
% DynBodyInit_reqt.tex
% Requirements on/levied by the DynBodyInit submodel
%
%%%%%%%%%%%%%%%%%%%%%%%%%%%%%%%%%%%%%%%%%%%%%%%%%%%%%%%%%%%%%%%%%%%%%%%%%%%%%%%%

\chapter{Product Requirements}\label{ch:\modelpartid:reqt}


\requirement[Translational state]{Translational State Specification in Terms of a Known Frame}
\label{reqt:DynBodyInit:cartesian_trans}
\begin{description}
  \item[Requirement:]\ \newline
    This model shall provide the ability to initialize
    the position and/or velocity vectors of a subject DynBody object
    given the subject body's position and/or velocity vectors
    with respect to a reference frame
    whose state with respect to the vehicle's integration frame is known.
  \item[Rationale:]\ \newline
    This requirement derives from
    requirement~\ref{reqt:overview:dyn_body_init}.
  \item[Verification:]\ \newline
    Inspection, Test
\end{description}


\requirement[Rotational state]{Rotational State Specification in Terms of a Known Frame}
\label{reqt:DynBodyInit:cartesian_rot}
\begin{description}
  \item[Requirement:]\ \newline
    This model shall provide the ability to initialize
    the attitude and/or attitude rate of a subject DynBody object
    given the object's attitude and/or attitude rate
    with respect to a reference frame
    whose state with respect to the vehicle's integration frame is known.
  \item[Rationale:]\ \newline
    This requirement derives from
    requirement~\ref{reqt:overview:dyn_body_init}.
  \item[Verification:]\ \newline
    Inspection, Test
\end{description}

\requirement{Orbital Elements}
\label{reqt:DynBodyInit:orbelem}
\begin{description}
  \item[Requirement:]\ \newline
    This model shall provide the ability to initialize
    the position and velocity of a subject DynBody object
    given a set of osculating orbital elements as represented in
    some non-rotating reference frame about some planet.
   Table~\ref{tab:reqt_orb_elem_checklist}
    lists the orbital elements sets
    that pertain to this requirement.
  \item[Rationale:]\ \newline
    The multiplicity of options enables the use of
    known vehicle translational states as provided by external sources.
  \item[Verification:]\ \newline
    Inspection, Test
\end{description}

\begin{table}[htp]
\begin{minipage}{\textwidth}
\centering
\caption{Orbital Elements}
\label{tab:reqt_orb_elem_checklist}
\vspace{1ex}
\begin{tabular}{||c||cccccccccccccc||c|}
\hline
   &
  \multicolumn{14}{||c||}{
    {\bf Orbital Elements}
    \footnote{
      Six orbital elements are needed to uniquely specify the orbital state.
      Six orbital elements are checked in each line of the table.
      The \ModelDesc shall provide the ability to transform the indicated
      sets of six orbital elements to position and velocity vectors.}
    } & \\ \hline\hline
  \tilt{\bf{Subrequirement}} &
  \tilt{\bf{Inclination}} &
  \tilt{\bf{Longitude of ascending node}} &
  \tilt{\bf{Semi-major axis}} &
  \tilt{\bf{Semi-latus rectum}} &
  \tilt{\bf{Eccentricity}} &
  \tilt{\bf{Altitude at periapsis}} &
  \tilt{\bf{Altitude at apoapsis}} &
  \tilt{\bf{Argument of periapsis}} &
  \tilt{\bf{Argument of latitude}} &
  \tilt{\bf{Time since periapsis passage}} &
  \tilt{\bf{Mean anomaly}} &
  \tilt{\bf{True anomaly}} &
  \tilt{\bf{Orbital radius}} &
  \tilt{\bf{Radial velocity}} &
  \tilt{\bf{Index Number}
    \footnote{
      The options are indexed in this fashion to replicate the
      numbering scheme used in JEOD 1.4/1.5.}
    } \\ \hline\hline
%                     SET #1  Classical orbital elements:
%                             Inclination,
%                             Longitude of Ascending Node,
%                             Eccentricity,
%                             Argument of Periapsis,
%                             Semi Major Axis,
%                             Time since Periapsis passage
\tabsubrequirement &
     $\surd$ & $\surd$ &
     $\surd$ & & $\surd$ & & & $\surd$ & & $\surd$ & & & & & 1 \\
%                     SET #2: Inclination,
%                             Longitude of Ascending Node,
%                             Eccentricity,
%                             Argument of Periapsis,
%                             Semi Major Axis,
%                             Mean Anomaly
\tabsubrequirement &
     $\surd$ & $\surd$ &
     $\surd$ & & $\surd$ & & & $\surd$ & & & $\surd$ & & & & 2\\
%                     SET #3: Inclination,
%                             Longitude of Ascending Node,
%                             Eccentricity,
%                             Argument of Periapsis,
%                             True Anomaly,
%                             Semi Latus Rectum,
\tabsubrequirement &
     $\surd$ & $\surd$ &
     & $\surd$ & $\surd$ & & & $\surd$ & & & & $\surd$ & & & 3 \\
%                     SET #4: Inclination,
%                             Longitude of Ascending Node,
%                             Argument of Periapsis,
%                             True Anomaly,
%                             Altitude at Periapsis,
%                             Altitude at Apoapsis,
\tabsubrequirement &
     $\surd$ & $\surd$ &
     & & & $\surd$ & $\surd$ & $\surd$ & & & & $\surd$ & & & 4 \\
%                     SET #5: Inclination,
%                             Longitude of Ascending Node,
%                             Argument of Periapsis,
%                             Time since Periapsis passage,
%                             Altitude at Periapsis,
%                             Altitude at Apoapsis,
\tabsubrequirement &
     $\surd$ & $\surd$ &
     & & & $\surd$ & $\surd$ & $\surd$ & & $\surd$ & & & & & 5 \\
%                     SET #6: Inclination,
%                             Longitude of Ascending Node,
%                             Semi Major Axis,
%                             Argument of Latitude,
%                             Orbital Radius,
%                             Orbit Radial Velocity,
\tabsubrequirement &
     $\surd$ & $\surd$ &
     $\surd$ & & & & & & $\surd$ & & & & $\surd$ & $\surd$ & 6 \\
%                     SET #10 AEITA
%                             Inclination,
%                             Right Ascension of Ascending Node,
%                             Eccentricity,
%                             Argument of Periapsis,
%                             Semi Major Axis,
%                             True Anomaly,
\tabsubrequirement &
     $\surd$ & $\surd$ &
     $\surd$ & & $\surd$ & & & $\surd$ & & & & $\surd$ & & & 10 \\
%                     SET #11 HAPRWP
%                             Altitude at Periapsis,
%                             Altitude at Apoapsis,
%                             Inclination,
%                             Right Ascension of Ascending Node,
%                             Argument of Periapsis,
%                             True Anomaly */
\tabsubrequirement &
     $\surd$ & $\surd$ &
     & & & $\surd$ & $\surd$ & $\surd$ & & & & $\surd$ & & & 11 \\
\hline
\end{tabular}
\end{minipage}
\end{table}
%\end{landscape}

\requirement[LVLH frame]{State Specification in Terms of an LVLH Frame}
\label{reqt:DynBodyInit:LVLH}
\begin{description}
  \item[Requirement:]\ \newline
    If the position and velocity of a reference DynBody object's body frame
    with respect to a subject DynBody object's integration frame are known,
    this model shall provide the ability to initialize
    aspects of the state of the subject body
    given elements of the subject body's state
    with respect to the reference body's
    rectilinear or curvilinear Local Vertical, Local Horizontal frame
    with respect to some planet.
  \item[Rationale:]\ \newline
    This requirement derives from
    requirement~\ref{reqt:overview:dyn_body_init}.
  \item[Verification:]\ \newline
    Inspection, Test
\end{description}

\requirement[NED frame]{State Specification in Terms of a North-East-Down Frame}
\label{reqt:DynBodyInit:NED}
\begin{description}
  \item[Requirement:]\ \newline
  \subrequirement{Point-relative.}
  \label{reqt:DynBodyInit:reference_NED}
    This model shall provide the ability to initialize
    aspects of the state of a subject body
    given elements of the subject body's state
    with respect to a planet-based reference point.
  \subrequirement{Vehicle-relative.}
  \label{reqt:DynBodyInit:relative_NED}
    If the position and velocity of a reference DynBody object's body frame
    with respect to a subject DynBody object's integration frame are known,
    this model shall provide the ability to initialize
    aspects of the state of the subject body
    given elements of the subject body's state
    with respect to the reference body's
    North-East-Down frame
    with respect to some planet.
  \item[Rationale:]\ \newline
    This requirement derives from
    requirement~\ref{reqt:overview:dyn_body_init}.
  \item[Verification:]\ \newline
    Inspection, Test
\end{description}

\requirement{Readiness Detection}
\label{reqt:DynBodyInit:is_ready}
\begin{description}
  \item[Requirement:]\ \newline
    This model shall provide the ability to detect when a state
    initializer is ready to be executed (and when it is not ready)
    based on the data needed by the initialization object.
  \item[Rationale:]\ \newline
    This requirement derives from
    requirement~\ref{reqt:overview:dyn_body_init}.

    One of the key goals of this model is to remove dependencies upon the order
    in which objects are declared in the S\_define file or the order in which
    objects are added to the Dynamic Manager's list of BodyAction objects.
  \item[Verification:]\ \newline
    Inspection, Test
\end{description}
