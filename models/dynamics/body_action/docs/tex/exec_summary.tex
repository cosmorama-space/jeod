%%%%%%%%%%%%%%%%%%%%%%%%%%%%%%%%%%%%%%%%%%%%%%%%%%%%%%%%%%%%%%%%%%%%%%%%%%%%%%%%
% exec_summary.tex
% Body Action model executive summary
%
%%%%%%%%%%%%%%%%%%%%%%%%%%%%%%%%%%%%%%%%%%%%%%%%%%%%%%%%%%%%%%%%%%%%%%%%%%%%%%%%

\chapter*{Executive Summary}

The \ModelDesc forms a component of the dynamics suite of
models within \JEODid. It is located at
models/dynamics/body\_action.

The model comprises several classes that provide various mechanisms
for setting aspects of instances of MassBody and DynBody classes as well as
their derived classes.
The capabilities provided by the model include:
\begin{itemize}
\item Setting the mass properties of a MassBody object,
\item Attaching and detaching DynBody and MassBody objects to/from one another,
\item Setting the state of a DynBody object, and
\item Changing the frame in which a DynBody object's state is propagated.
\end{itemize}

Note that while the first function refers to a MassBody, the latter two
capabilities pertain to DynBody. Meanwhile, attachment can involve
interations between a DynBody and MassBody object.
 The subject can be specified by MassBody or DynBody, and the correct behavior
 implied (if possible).

\section*{Interations With Other Models}
The \ModelDesc interacts with several JEOD models. Chief among these
are interactions with
the \hypermodelref{MASS},
which defines the MassBody class;
the \hypermodelref{DYNBODY},
which defines the DynBody class; and
the \hypermodelref{DYNMANAGER},
which defines the DynManager class.

A JEOD simulation typically contains one or more instances of a
class that derives from the DynBody class.
Each DynBody object represents some vehicle or module in space.
The DynBody-type vehicle may also have other DynBody or MassBody submodules
attached. The purpose of this model is to alter some aspect of the subject body.
Some of the model classes operate directly on MassBody objects
via the use of public interfaces. Some operate directly on DynBody's using
public interfaces. Others, such as attach actions, operate using either
MassBody or DynBody methods depending on the type of subject.

In addition to the DynBody objects that represent vehicles,
a JEOD simulation also contains a DynManager object that manages the
dynamic objects in a simulation.
The DynManager object uses the \ModelDesc via a Standard Template Library
list of pointers to queue \ModelDesc objects. Simulation users can add
elements to this list. As the list comprises pointers to objects, the
pointed-to objects can be instances of different classes so long as
they derive from the same base class. (All \ModelDesc objects derive
from a common base class.)
The simulation DynManager object uses this list during simulation
initialization time to initialize the simulation's MassBody/DynBody objects
and uses this list during simulation run time to perform
asynchronous actions.

In addition to the above three models, the \ModelDesc interacts with
several other JEOD models. A driving concern for the \ModelDesc is to
do as little as possible in the model.
The model source files instead leverage capabilities that exist in other
models.
In addition to the models listed above, this model uses:
\begin{itemize}
\item The \hypermodelref{DERIVEDSTATE}
to construct Local Vertical, Local Horizontal frames,
\item The \hypermodelref{MATH}
to determine planetary coefficients,
\item The \hypermodelref{PLANETFIXED}
to construct North-East-Down frames,
\item The \hypermodelref{REFFRAMES}
to compute relative state.
\end{itemize}


\section*{\ModelDesc Base Class}
All of the instantiable \ModelDesc classes ultimately derive from the abstract BodyAction
base class. This abstract base class by itself does not change a single aspect of
a MassBody or DynBody object. That is the
responsibility of the various specialized classes that derive from the base BodyAction
class. The base class provides a common virtual framework for describing actions to be
performed on a body-type object.
This framework includes:
\begin{itemize}
\item Descent from a common base class. This common basis enables forming a
collection of body actions in the form of base class pointers. The DynManager
class does exactly this. The \ModelDesc was explicitly architected to enable
this treatment.
\item Base class member data. This member data includes a pointer to the
subject of the BodyAction object in the form of either of DynBody subject or a
MassBody subject. Populating both subject pointers is unnecessary, as the
DynBody corresponding to a MassBody is implied (and vice versa).
\item Base class member functions. The member functions specify how to operate
on the BodyAction object. In particular, a trio of virtual methods described
below form the standard basis by which derived classes can extend
this base class.
\end{itemize}

The base class defines three virtual methods that together form the
mechanism by which an external user of the \ModelDesc functionally
interacts with BodyAction objects:
\begin{description}
\item[initialize]
Each \ModelDesc object must be initialized.
This initialization is the task of the overridable {\tt initialize()}
method. This method initializes a BodyAction object, and should verify that the BodyAction is
properly configured.
It does not alter the subject object.
\item[is\_ready]
The {\tt is\_ready()} method indicates whether the BodyAction object is
ready to perform its action on the subject.
For example, a DynBodyInitLvlhState object sets parts of the subject
DynBody object's state based on the Local Vertical, Local Horizontal (LVLH)
frame defined by some reference DynBody object.
The DynBodyInitLvlhState object remains in a not-ready state
until the reference body's position and velocity vector have been set.
\item[apply]
The {\tt apply()} method performs the action on the subject.
After the action is performed, it is removed from the DynManager's queue.
\end{description}

\section*{\ModelDesc Derived Classes}

The \ModelDesc provides several classes that derive from the base BodyAction
class. These are
\begin{description}
\item[MassBodyInit] Derives from BodyAction. \\
Instances of this class set the subject MassBody object's
mass properties and specify the object's mass points.
In the case where the subject MassBody belongs to a DynBody,
specifying the mass points also specifies the DynBody object's
vehicle points.
\item[BodyAttach] Derives from BodyAction. \\
Classes derived from this class attach one body object to another.
This abstract class does not accomplish this objective;
it is a non-instantiable class. Attaching bodies is a task to be
performed by derived classes specifying the attach method.
\item[BodyAttachAligned] Derives from BodyAttach. \\
Instances of this class attach the subject body to a parent body.
The attachment is specified in terms of a pair of points, one on the subject body and the other on the parent body. 
The attachment makes the two
points coincident and makes the transformation between the two
points' reference frames be a 180 degree yaw.
\item[BodyDetach and BodyDetachSpecific] Derive from BodyAction. \\
Instances of these classes detach a body object
from its immediate parent.
The class BodyDetach detaches the subject from its
immediate parent. The class BodyDetachSpecific searches the attachment tree
for the specified other body (starting from the subject body), then detaches the
linkage at the parent body. That is, if the subject is the progeny, the detach
takes place at the other body; if the subject is the progenitor, the detach
takes place at the subject.
The found object is detached from the specified parent.
\item[DynBodyInit] Derives from BodyAction. \\
The purpose of this class is to initialize the state (or parts
of the state) of a DynBody object.
This specific class does not accomplish this goal;
this is a non-instantiable class.
The classes that derive from DynBodyInit provide multiple
means of initializing aspects of a DynBody object's state.
\item[DynBodyInitRotState] Derives from DynBodyInit. \\
Instances of this class initialize a vehicle's attitude
and/or attitude rate.
\item[DynBodyInitTransState] Derives from DynBodyInit. \\
Instances of this class initialize a vehicle's position
and/or velocity.
\item[DynBodyInitOrbit] Derives from DynBodyInitTransState. \\
Instances of this class initialize a vehicle's position
and velocity by means of some orbit specification.
\item[DynBodyInitPlanetDerived] Derives from DynBodyInitWrtPlanet,
which in turn derives from DynBodyInit. \\
These classes form the basis for subsequent derived classes that
initialize state with respect to a planetary frame.
\item[DynBodyInitLvlhState] Derives from DynBodyInitPlanetDerived. \\
Instances of this class initialize state with respect to the
rectilinear LVLH frame defined by some reference vehicle's orbit
about a specified planet.
Two derived classes from this class,
DynBodyInitLvlhRotState and DynBodyInitLvlhTransState,
limit the initialization to the rotational and translational
states of the subject DynBody.
\item[DynBodyInitNedState] Derives from DynBodyInitPlanetDerived. \\
Instances of this class initialize state with respect to the
North-East-Down frame defined by some reference vehicle or
reference point.
Two derived classes from this class,
DynBodyInitNedRotState and DynBodyInitNedTransState,
limit the initialization to the rotational and translational
states of the subject DynBody.
\end{description}

\section*{Integrating the \ModelDesc in a Simulation}

The following approaches present an obvious, but wrong, approach to using the \ModelDesc
in a Trick simulation:
\begin{itemize}
\item Defining data elements of the appropriate model classes
for each vehicle simulation object in the simulation S\_define file and
\item Explicitly calling the {\tt initialize()} and {\tt apply()} methods
for those model objects in the simulation S\_define file.
\end{itemize}

{\em Do not follow the above practice.} This practice circumvents the
flexibility built into the model and will almost certainly lead to
difficulties in the future. It is best add the body action to the DynManager's
queue, and allow DynManager to apply.

\subsection*{The BodyAction List}

The recommended practice regarding the use of the \ModelDesc in a
Trick simulation is to use the model in conjunction with the
simulation's DynManager object. The DynManager maintains a list
of \ModelDesc objects. Calling the DynManager's {\tt add\_body\_action}
adds an item to the list. The DynManager draws objects from this list
at the appropriate time and only does so when the objects indicate
that they are ready to be executed. This mechanism disconnects the
execution order of the enqueued model objects from the order in
which the simulation objects are declared and from the order
in which the objects are added to the list.

This separation between initialization and declaration order
was first addressed in JEOD 1.4/1.5, but the C-based solution
involved some rather convoluted and very inflexible code.
The C++ based solution introduced in JEOD 2.0 makes for
a very extensible and simpler set of code. Users are strongly
recommended to take advantage of this flexibility.

\subsection*{A Simple Simulation}
A simple way to make use of the DynManager's BodyAction list
limit the use of the \ModelDesc to a pre-defined,
but presumably well-vetted, set.
A simulation S\_define file that implements this option will:
\begin{itemize}
\item Declare a very specific set of model objects as simulation
object elements for each vehicle in the simulation,
\item Specifically invoke the DynManager {\tt add\_body\_action()}
method on each of these model objects as initialization class jobs,
\item Invoke the DynManager {\tt initialize\_simulation()} method
as an initialize class job that runs after the {\tt add\_body\_action()}
jobs, and
\item Optionally invoke the DynManager {\tt perform\_actions()}
as a scheduled job to enable asynchronous use of the model during
the course of the simulation run.
\end{itemize}

This approach makes user input fairly simple. All
the simulation user needs to do is populate the pre-defined set of
model objects with the appropriate values.

The lack of flexibility is the key disadvantage of this option.
To illustrate this, consider a mission involving a chaser and a target vehicle.
Different aspects of the mission that need to be investigated
include the following scenarios:
\begin{itemize}
\item The launch of the chaser vehicle. The chaser starts at rest with respect
to the launch pad. The target vehicle does not even need to
be involved in this scenario.
Specifying the chaser's initial state in some inertial
frame would be awkward at best.
The latitude, longitude, and altitude of the launch pad are known
quantities, as is the geometry of the chaser with respect to the pad.
The most transparent option would be to specify the chaser initial state
in terms of these known quantities.
\item The transfer of the chaser from orbit insertion to far-field rendezvous.
In this situation, the chaser needs to start in an orbit that is related
to the target vehicle's orbit. The large initial separation between the
two vehicles rules out a relative cartesian specification for the chaser.
\item Various proximity operations between the chaser and target.
Initializing the chaser's state relative to that
of the target is exactly what is called for in this scenario.
\item Chaser de-orbit and entry.
As with the launch scenario, the target vehicle can be omitted
from this scenario. The chaser vehicle needs to be specified in an orbit
relative to the rotating Earth so that it will enter the atmosphere at the
correct location with respect to the rotating Earth.
\end{itemize}

\subsection*{Multiple Initialization Capabilities}
To handle phase transitions in multi-phase simulations (e.g., orbit to re-entry), patched approaches are recommended.
In this methodology, synchronous BodyAction's re-initialize states to "tweak" as appropriate
This imperfect approach allows the developer to implement multiple vehicle configurations as needed.
The best way to initialize the chaser vehicle's state in these different
scenarios is to use the \ModelDesc state initialization classes best suited
for the scenario. For example:
\begin{itemize}
\item The DynBodyInitNed class to initialization the chaser on the launch pad,
\item The DynBodyInitTransState or DynBodyInitOrbit classes to initialize
the chaser at orbit insertion and prior to entry, and
\item The DynBodyInitLvlh class to initialize the chaser in close proximity
to the target.
\end{itemize}
Each of these different initialization capabilities needs to be allocated
in the S\_define file, trick/C++ source, or python input layer to enable their use in the simulation
(NOTE: as of Trick 19.2.3 the DynManager queue for BodyActions is checkpointable/restartable with some exceptions).
One common way to accomplish this end is to declare as simulation object
elements all of the state initializers needed by the various scenarios.

This option continues that used in the previous example.
The simulation S\_define file specifically
invokes the DynManager {\tt add\_body\_action()} method for
each initializer. This solves the issue of different scenarios
requiring different initializers, but introduces a new issue.
The set of state initializers now over-specifies the state.
The input file for a particular scenario will need to deactivate
those state initializers that are not to be used in that scenario
and/or use contextual triggers to activate them at the correct time.

\subsection*{Deferring {\tt add\_body\_action()} Calls}
An alternative approach to solving this over-specification problem is to
never let it happen. The previous option required registering
in the S\_define file a call to {\tt add\_body\_action()}
for each state initializers declared in the S\_define file.
This option only registers one such call in the S\_define file.
A simulation S\_define file that implements this option will:
\begin{itemize}
\item Declare a potentially broad set of \ModelDesc objects as
simulation object elements for each vehicle in the simulation.
\item Declare a pointer to a BodyAction object in the same sim object
that contains the DynManager object.
\item Define {\em as a zero-rate scheduled job} a call to the DynManager's
{\tt add\_body\_action()} method that adds the contents of this pointer
to the DynManager's BodyAction list.
\end{itemize}

The task of calling {\tt add\_body\_action()} to place an initializer
in the DynManager's body action list is deferred to the input file.
An input file that follows this paradigm will
perform the following:
\begin{itemize}
\item Set the contents of a particular initializer,
\item Point the BodyAction pointer declared in
the S\_define file point to this initializer,
\item Issue a {\tt call} to the zero-rate
{\tt add\_body\_action()} job to add this initializer, and
\item Repeat the above for each initializer to be used in
the simulation run.
\end{itemize}

\subsection*{Dynamic allocation of initializers}
The S\_define file may become bloated with multiple initializers of the
same type in a simulation that involves several vehicles.
Along with deferring the calls to {\tt add\_body\_action()},
this final option defers the allocation of the initializers
to the input file via the Trick input processor's {\tt new} capability.

The key advantage of this final approach is that it maximizes flexibility.
This is also a potential disadvantage; enhanced flexibility invites errors
on the part of the user. Another disadvantage is that the allocated
objects cannot (currently) be logged.

\section*{Using the \ModelDesc in a Simulation}
How the \ModelDesc is to be used depends on which of the above options
was employed by the simulation integrator. This discussion assumes a
setup along the lines of the penultimate option described above
("Deferring {\tt add\_body\_action()} calls").
The simulation user selects those capabilities from
the S\_define file that best represent the needs of the scenario
to be simulated.

For example, consider the case of investigating the behavior of a chaser
vehicle in close proximity to a target vehicle. The S\_define file
provides multiple mechanisms for initializing the chaser and target.
The simulation user selects a set of initializers that enables specifying:
\begin{itemize}
\item The target vehicle's mass properties and points of interest,
\item The target vehicle's translational state with some given inertial values,
\item The target vehicle's rotational state relative to the target
vehicle's LVLH frame,
\item The chaser vehicle's mass properties and points of interest, and
\item The chaser vehicle's state in terms of the relative state between
a pair of points of interest on the chaser and target.
\end{itemize}

For each selected item, the user must place statements in the
simulation input that:
\begin{itemize}
\item Populate the data items for the object. \\
\item Point the simulation body action pointer to this model object. \\
\item Call the {\tt add\_body\_action()} simulation job to add the
object to the Dynamic Manager's list of model objects.
\end{itemize}

Four mistakes that the user can make in this process are:
\begin{itemize}
\item Failing to specify the subject body for each model object.
The JEOD methods do not know that the simulation integrator has made the
subject body and the model object parts of the same simulation object.
\item Attempting to reuse model objects. The {\tt add\_body\_action()}
method must not (and cannot) create a copy of the object passed to it.
\item Creating a circular set of dependencies. For example,
specifying the state of vehicle A with respect to vehicle B and
specifying the state of vehicle B with respect to vehicle A.
\item Failing to specify all needed states.
\end{itemize}
JEOD detects and reports these errors with messages.

\section*{Requirements on Derived Classes}
The \ModelDesc places several functional constraints on the classes
that derive from the BodyAction base class.
These constraints apply to the classes that are a part of the model
and to any extensions of the class written by model extenders.
These constraints are on the trio of virtual methods as overridden
by the derived class:
\begin{enumerate}
\item The {\tt initialize()} method, if overridden,
must forward the {\tt initialize()} call to the immediate parent object.
The typical approach is to check for error conditions,
forward the {\tt initialize()} call  upward, and finally perform
class-specific initializations. This removes the need to write redundant
routines in derived classes.
\item The {\tt is\_ready()} method, if overridden,
must query the readiness of the parent class in addition to
performing class-specific readiness checks. Claiming to be ready
against the advice of the parent is forbidden.
\item The {\tt apply()} method, if overridden,
must forward the {\tt apply()} call to the immediate parent object.
The typical approach is to reformulate the user inputs in terms
that will be understood by the parent class,
forward the {\tt apply()} call  upward, and finally perform
any cleanup actions.
\end{enumerate}

\section*{Requirements on Functional Users of the \ModelDesc}
The \ModelDesc places several functional constraints on the use of the model.
These constraints apply to the Dynamics Manager Model and to any
use of the model written by a user of JEOD.
These constraints are
\begin{enumerate}
\item A BodyAction object must be initialized.
The object's {\tt initialize()} method must be invoked prior to invoking
either the {\tt is\_ready()} or {\tt apply()} methods for that object.
\item A BodyAction object must be initialized once only.
Invoking a single object's {\tt initialize()} method multiple times may
result in undetected errors such as memory leaks.
\item A BodyAction object must be initialized at the appropriate time.
\begin{enumerate}
\item BodyAttach actions must be initialized
after the subject MassBody objects have had their initial mass properties
set and have had their mass points defined.
\item  DynBodyInit actions furthermore must be initialized after
the initialization-time attachments have been performed.
\item All other actions must be initialized after the states of the
simulation's DynBody objects have been set.
\end{enumerate}
\item A BodyAction object's {\tt apply()} method must not be invoked
before the object indicates that it is ready to be applied. In other words,
the object's {\tt is\_ready()} method must return {\tt true} before
calling the object's {\tt apply()} method.
\item A BodyAction object's {\tt apply()} method must not be invoked
multiple times. The \ModelDesc follows the fire-and-forget paradigm.
\end{enumerate}
