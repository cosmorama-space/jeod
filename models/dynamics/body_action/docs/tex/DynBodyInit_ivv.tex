%%%%%%%%%%%%%%%%%%%%%%%%%%%%%%%%%%%%%%%%%%%%%%%%%%%%%%%%%%%%%%%%%%%%%%%%%%%%%%%%
% DynBodyInit_ivv.tex
% Verification and validation for the DynBodyInit and derived classes.
%
%%%%%%%%%%%%%%%%%%%%%%%%%%%%%%%%%%%%%%%%%%%%%%%%%%%%%%%%%%%%%%%%%%%%%%%%%%%%%%%%

\chapter{Inspection, Verification, and Validation}\label{ch:\modelpartid:ivv}
\section{Inspection}\label{sec:DynBodyInit:inspect}

\inspection[DynBodyInit design]{Design Inspection}
\label{inspect:DynBodyInit:design}
Table~\ref{tab:DynBodyInit:virtual_overrides}
portrays the class hierarchy for this sub-model
and identifies which classes provide implementations of the
{\tt initialize()}, {\tt is\_ready()}, and {\tt apply()} methods.
As shown in that table,\begin{itemize}
\item The DynBodyInit class derives directly from the BodyAction class.
\item All other \partxname classes derive from DynBodyInit, some directly,
others indirectly.
\item For each provided implementation of the key virtual methods
an implementation of that method also exists in the parent class.
\end{itemize}
Code inspection reveals that each implementation of the key virtual methods
invokes the overridden parent class implementation and that
all of the {\tt is\_ready()} methods properly incorporate the result from
the invocation of the overridden {\tt is\_ready()} into the readiness
assessment.

By inspection, the \partxname satisfies
requirements~\ref{reqt:BodyAction:virtual_methods}
and~\ref{reqt:BodyAction:base_class_mandate}.

\begin{table}[htp]
\centering
\caption{DynBody State Initialization Sub-Model Virtual Methods}
\label{tab:DynBodyInit:virtual_overrides}
\vspace{1ex}
\begin{minipage}{0.8\textwidth}
\centering
\begin{tabular}{||l|l|c|c|c|} \hline
{\bf Class} & {\bf Parent Class} &
\multicolumn{3}{|c|} {\bf Class Defines $\cdots$} \\
{\bf Name
  \footnote{To reduce the table width, the leading DynBodyUnit on
  classes that derive from DynBodyInit is replaced with center dots.
  For example, the class DynBodyInitWrtPlanet is abbreviated as
  $\cdots$WrtPlanet.
  \label{fn:DynBodyInit:virt_meth_table_abbrev}}} &
{\bf Name \footref{fn:DynBodyInit:virt_meth_table_abbrev}} &
{\bf initialize()} & {\bf is\_ready()} & {\bf apply()} \\
\hline \hline
DynBodyInit & BodyAction &
  \checkmark & \checkmark & \checkmark \\
$\cdots$WrtPlanet & DynBodyInit &
  \checkmark & \checkmark & \checkmark \\
$\cdots$PlanetDerived & $\cdots$WrtPlanet &
  \checkmark & \checkmark & \checkmark \\
$\cdots$LvlhState & $\cdots$PlanetDerived &
  \checkmark & & \checkmark \\
$\cdots$LvlhRotState & $\cdots$LvlhState &
  \checkmark & & \\
$\cdots$LvlhTransState & $\cdots$LvlhState &
  \checkmark & & \\
$\cdots$NedState & $\cdots$PlanetDerived &
  \checkmark & & \checkmark \\
$\cdots$NedRotState & $\cdots$NedState &
  \checkmark & & \\
$\cdots$NedTransState & $\cdots$NedState &
  \checkmark & & \\
$\cdots$TransState & DynBodyInit &
  \checkmark & \checkmark & \checkmark \\
$\cdots$Orbit & $\cdots$TransState &
  \checkmark & & \checkmark \\
$\cdots$RotState & DynBodyInit &
  \checkmark & \checkmark & \checkmark \\
\hline
\end{tabular}
\end{minipage}
\end{table}


\section{Validation}
This section describes various tests conducted to verify and validate
that the \partxname satisfies the requirements levied against it.
All verification and validation test source code, simulations and procedures
for this sub-model are in either the {\tt SIM\_orbinit}, or {\tt SIM\_lvlh\_init},
subdirectory of the JEOD directory {\tt models/dynamics/body\_action/verif}.

Each of the simulation directories comprise
\begin{itemize}
\item A Trick \Sdefine file. This file defines
two dynamic bodies, each having a variety of state initializers;
a dynamics manager, which is needed to run the initializers,
and a minimal set of other objects needed by the dynamics manager.
\item Run directories in each of the {\tt SET\_test} and
{\tt SET\_test\_val} directories,
The run directories within the {\tt SET\_test} directory contain
simulation input files which the correspondingly-named run directories
in the {\tt SET\_test\_val} directory contain log files used for
regression testing.
\item Several {\tt Modified\_data} files and one {\tt Log\_data} file,
which are used by the run input files to configure a simulation run.
\item In {\tt SIM\_orbinit}, one test script, {\tt scripts/run\_tests.pl}.
The output of the shell script is organized to correspond with
the validation tests described in this section.
Numbers that pass the various test criteria are marked in green while
failing values are marked in red.
\end{itemize}

\newpage
\test{State Initializations}
\label{test:DynBodyInit:state_init}
\begin{description}
\item[Background]
The purpose of the \partxname is to initialize
reference frame states contained in a DynBody object.
A reference frame state comprises two main parts, the translational and
rotational aspects of the frame. A reference frame's translational state
describes the position and velocity of the origin of the frame with respect
to the parent frame; the rotational state describes the orientation and
angular velocity of the frame's axes with respect to the parent frame.

The DynBodyInit subclasses set some but not necessarily all aspects of some
reference frame associated with a DynBody object. The set reference frame
might or might not be the DynBody object's integrated frame, and the
reference reference frame for the DynBodyInit object might or might not
be the DynBody object's integration frame.

\item[Test description]
These tests verify the errors that result from initializing state
via the various DynBodyInit subclasses fall within bounds.

\item[Test cases]
Table~\ref{tab:DynBodyInit:Cartesian}
lists the simulation run directories associated with this test
and the results of the test.

\item[Success criteria]
These tests pertain to a vehicle initialized in low Earth orbit
using either specifications relative to the Earth or to another
spacecraft in low Earth orbit. The errors should be very small,
and thus the success criteria are quite stringent for this test.
Each component of the position vector
must have an error no larger than 1 millimeter
and each component of the velocity vector
must have an error no larger than 0.01 millimeters/second.
\item[Test results]
All tests pass.
\item[Applicable requirements]
This test collectively demonstrates the satisfaction of
requirements~\ref{reqt:DynBodyInit:cartesian_trans}
to~\ref{reqt:DynBodyInit:is_ready}.
\end{description}

\begin{table}[htp]
\centering
\caption{Cartesian Position/Velocity Test Cases}
\label{tab:DynBodyInit:Cartesian}
\vspace{1ex}
\begin{minipage}{\textwidth}
\centering
\begin{tabular}{||l|l|l|l|l|l|l|l|} \hline
{\bf Run} & {\bf Class} & {\bf Source} & {\bf Subj.} & {\bf Subj.} &
{\bf Pos. Err.} & {\bf Vel. Err.} & {\bf Status} \\
{\bf Directory} & {\bf Name
\footnote{The class names in the table are listed in abbreviated form.
The true class name is the listed class name prefixed with ``DynBodyInit''
and suffixed with ``State''.
For example, Trans is short for DynBodyInitTransState.}} &
{\bf Frame} & {\bf Frame} & {\bf Vehicle} &
{(m)} & {(m/s)} & \\
\hline \hline
RUN\_0400 & Trans      & \Inertial & \Body   & ISS     &
  \green{$0.0$}          & \green{$0.0$}          & \passed \\
RUN\_0401 & Trans      & \Inertial & \Body   & STS 114 &
  \green{$0.0$}          & \green{$0.0$}          & \passed \\
RUN\_0410 & Trans      & \Pfix     & \Body   & ISS     &
  \green{$3.0\eneg{9}$}  & \green{$1.9\eneg{12}$} & \passed \\
RUN\_0411 & Trans      & \Pfix     & \Body   & STS 114 &
  \green{$3.0\eneg{9}$}  & \green{$1.8\eneg{12}$} & \passed \\
RUN\_0441 & Trans      & \TBody    & \Body   & STS 114 &
  \green{$1.3\eneg{10}$} & \green{$3.5\eneg{13}$} & \passed \\
RUN\_0571 & LvlhTrans  & \TLvlh    & \Body   & STS 114 &
  \green{$1.3\eneg{10}$} & \green{$3.3\eneg{13}$} & \passed \\
RUN\_0681 & NedTrans   & \TNed     & \Body   & STS 114 &
  \green{$4.1\eneg{9}$}  & \green{$2.2\eneg{12}$} & \passed \\
RUN\_3771 & Lvlh       & \TLvlh    & \Body   & STS 114 &
  \green{$1.3\eneg{10}$} & \green{$3.3\eneg{13}$} & \passed \\
RUN\_3822 & Ned        & \Ned      & \Struct & Pad 39A &
  \green{$2.1\eneg{9}$}  & \green{$7.1\eneg{14}$} & \passed \\
RUN\_4451 & Trans      & \TStruct  & \Struct & STS 114 &
  \green{$3.3\eneg{10}$} & \green{$2.8\eneg{13}$} & \passed \\
RUN\_4681 & NedTrans   & \TNed     & \Struct & STS 114 &
  \green{$1.3\eneg{9}$}  & \green{$1.3\eneg{13}$} & \passed \\
RUN\_5461 & Trans      & \TPoint   & \Point  & STS 114 &
  \green{$7.7\eneg{10}$} & \green{$1.0\eneg{12}$} & \passed \\
\hline
\end{tabular}
\end{minipage}
\end{table}

\begin{table}[htp]
\centering
\caption{Orbital Elements Test Cases}
\label{tab:DynBodyInit:Orbit}
\vspace{1ex}
\begin{tabular}{||l|l|l|l|l|l|l|} \hline
{\bf Run} & {\bf Source} & {\bf Subject} & {\bf Set} &
{\bf Pos. Err} & {\bf Vel. Err.} & {\bf Status} \\
{\bf Directory} & {\bf Frame} & {\bf Vehicle} & {\bf \#} &
{(m)} & {(m/s)} & \\
\hline \hline
RUN\_0001 & \Inertial & ISS     &  1 &
  \green{$4.9\eneg{5}$}  & \green{$6.6\eneg{8}$}  & \passed \\
RUN\_0002 & \Inertial & ISS     &  2 &
  \green{$2.7\eneg{5}$}  & \green{$5.4\eneg{8}$}  & \passed \\
RUN\_0003 & \Inertial & ISS     &  3 &
  \green{$1.9\eneg{5}$}  & \green{$5.7\eneg{8}$}  & \passed \\
RUN\_0004 & \Inertial & ISS     &  4 &
  \green{$1.7\eneg{5}$}  & \green{$5.3\eneg{8}$}  & \passed \\
RUN\_0005 & \Inertial & ISS     &  5 &
  \green{$2.9\eneg{5}$}  & \green{$5.9\eneg{8}$}  & \passed \\
RUN\_0006 & \Inertial & ISS     &  6 &
  \green{$1.7\eneg{5}$}  & \green{$5.3\eneg{8}$}  & \passed \\
RUN\_0010 & \Inertial & ISS     & 10 &
  \green{$1.7\eneg{5}$}  & \green{$5.7\eneg{8}$}  & \passed \\
RUN\_0011 & \Inertial & ISS     & 11 &
  \green{$1.7\eneg{5}$}  & \green{$5.3\eneg{8}$}  & \passed \\
RUN\_0101 & \Inertial & STS 114 &  1 &
  \green{$1.7\eneg{5}$}  & \green{$3.2\eneg{8}$}  & \passed \\
RUN\_0102 & \Inertial & STS 114 &  2 &
  \green{$4.2\eneg{5}$}  & \green{$4.4\eneg{8}$}  & \passed \\
RUN\_0103 & \Inertial & STS 114 &  3 &
  \green{$8.3\eneg{5}$}  & \green{$1.1\eneg{7}$}  & \passed \\
RUN\_0104 & \Inertial & STS 114 &  4 &
  \green{$8.3\eneg{5}$}  & \green{$9.6\eneg{8}$}  & \passed \\
RUN\_0105 & \Inertial & STS 114 &  5 &
  \green{$2.0\eneg{5}$}  & \green{$3.4\eneg{8}$}  & \passed \\
RUN\_0106 & \Inertial & STS 114 &  6 &
  \green{$3.9\eneg{5}$}  & \green{$5.0\eneg{8}$}  & \passed \\
RUN\_0110 & \Inertial & STS 114 & 10 &
  \green{$8.4\eneg{5}$}  & \green{$1.1\eneg{7}$}  & \passed \\
RUN\_0111 & \Inertial & STS 114 & 11 &
  \green{$8.3\eneg{5}$}  & \green{$9.6\eneg{8}$}  & \passed \\
RUN\_0201 & \Pfix     & ISS     &  1 &
  \green{$8.0\eneg{5}$}  & \green{$6.7\eneg{8}$}  & \passed \\
RUN\_0202 & \Pfix     & ISS     &  2 &
  \green{$7.1\eneg{5}$}  & \green{$3.8\eneg{8}$}  & \passed \\
RUN\_0203 & \Pfix     & ISS     &  3 &
  \green{$6.4\eneg{5}$}  & \green{$3.7\eneg{8}$}  & \passed \\
RUN\_0204 & \Pfix     & ISS     &  4 &
  \green{$6.3\eneg{5}$}  & \green{$6.4\eneg{8}$}  & \passed \\
RUN\_0205 & \Pfix     & ISS     &  5 &
  \green{$7.0\eneg{5}$}  & \green{$5.1\eneg{9}$}  & \passed \\
RUN\_0206 & \Pfix     & ISS     &  6 &
  \green{$6.3\eneg{5}$}  & \green{$6.2\eneg{8}$}  & \passed \\
RUN\_0210 & \Pfix     & ISS     & 10 &
  \green{$6.4\eneg{5}$}  & \green{$3.7\eneg{8}$}  & \passed \\
RUN\_0211 & \Pfix     & ISS     & 11 &
  \green{$6.3\eneg{5}$}  & \green{$6.4\eneg{8}$}  & \passed \\
RUN\_0301 & \Pfix     & STS 114 &  1 &
  \green{$4.2\eneg{5}$}  & \green{$6.2\eneg{8}$}  & \passed \\
RUN\_0302 & \Pfix     & STS 114 &  2 &
  \green{$8.9\eneg{5}$}  & \green{$8.9\eneg{8}$}  & \passed \\
RUN\_0303 & \Pfix     & STS 114 &  3 &
  \green{$2.3\eneg{5}$}  & \green{$6.6\eneg{8}$}  & \passed \\
RUN\_0304 & \Pfix     & STS 114 &  4 &
  \green{$1.9\eneg{5}$}  & \green{$5.7\eneg{8}$}  & \passed \\
RUN\_0305 & \Pfix     & STS 114 &  5 &
  \green{$2.4\eneg{5}$}  & \green{$6.0\eneg{8}$}  & \passed \\
RUN\_0306 & \Pfix     & STS 114 &  6 &
  \green{$1.9\eneg{5}$}  & \green{$5.6\eneg{8}$}  & \passed \\
RUN\_0310 & \Pfix     & STS 114 & 10 &
  \green{$2.5\eneg{5}$}  & \green{$6.7\eneg{8}$}  & \passed \\
RUN\_0311 & \Pfix     & STS 114 & 11 &
  \green{$1.9\eneg{5}$}  & \green{$5.7\eneg{8}$}  & \passed \\
\hline
\end{tabular}
\end{table}

\begin{table}[htp]
\centering
\caption{Attitude Test Cases}
\label{tab:DynBodyInit:Attitude}
\vspace{1ex}
\begin{minipage}{\textwidth}
\centering
\begin{tabular}{||l|l|l|l|l|l|l|} \hline
{\bf Run} & {\bf Class} & {\bf Source} & {\bf Subj.} & {\bf Subj.} &
{\bf Att. Err.} & {\bf Status} \\
{\bf Directory} & {\bf Name
\footnote{See table~\ref{tab:DynBodyInit:Cartesian}.}} &
{\bf Frame} & {\bf Frame} & {\bf Vehicle} &
{(d)} & \\
\hline \hline
RUN\_1230 & LvlhRot    & \TLvlh    & \Body   & ISS     &
  \green{$0.0$}          & \passed \\
RUN\_2100 & Rot        & \Inertial & \Body   & ISS     &
  \green{$3.0\eneg{14}$} & \passed \\
RUN\_3771 & Lvlh       & \TLvlh    & \Body   & STS 114 &
  \green{$2.1\eneg{14}$} & \passed \\
RUN\_3822 & Ned        & \Ned      & \Struct & Pad 39A &
  \green{$1.8\eneg{14}$} & \passed \\
RUN\_4451 & Rot        & \TStruct  & \Struct & STS 114 &
  \green{$2.7\eneg{14}$} & \passed \\
RUN\_4681 & NedRot     & \TNed     & \Struct & STS 114 &
  \green{$1.4\eneg{14}$} & \passed \\
RUN\_5461 & Rot        & \TPoint   & \Point  & STS 114 &
  \green{$1.3\eneg{14}$} & \passed \\
\hline
\end{tabular}
\end{minipage}
\end{table}


\begin{table}[htp]
\centering
\caption{Body Rate Test Cases}
\label{tab:DynBodyInit:Att_rate}
\vspace{1ex}
\begin{minipage}{\textwidth}
\centering
\begin{tabular}{||l|l|l|l|l|l|l|} \hline
{\bf Run} & {\bf Class} & {\bf Source} & {\bf Subj.} & {\bf Subj.} &
{\bf Rate Err.} & {\bf Status} \\
{\bf Directory} & {\bf Name
\footnote{See table~\ref{tab:DynBodyInit:Cartesian}.}} &
{\bf Frame} & {\bf Frame} & {\bf Vehicle} &
{(d/s)} & \\
\hline \hline
RUN\_1230 & LvlhRot    & \TLvlh    & \Body   & ISS     &
  \green{$7.8\eneg{19}$} & \passed \\
RUN\_2100 & Rot        & \Inertial & \Body   & ISS     &
  \green{$0.0$}          & \passed \\
RUN\_3771 & Lvlh       & \TLvlh    & \Body   & STS 114 &
  \green{$1.2\eneg{17}$} & \passed \\
RUN\_3822 & Ned        & \Ned      & \Struct & Pad 39A &
  \green{$9.8\eneg{19}$} & \passed \\
RUN\_4451 & Rot        & \TStruct  & \Struct & STS 114 &
  \green{$2.2\eneg{17}$} & \passed \\
RUN\_4681 & NedRot     & \TNed     & \Struct & STS 114 &
  \green{$4.5\eneg{19}$} & \passed \\
RUN\_5461 & LvlhRot    & \TLvlh    & \Body   & STS 114 &
  \green{$0.0$}          & \passed \\
\hline
\end{tabular}
\end{minipage}
\end{table}

\clearpage
\section{Requirements Traceability}
Table~\ref{tab:DynBodyInit:reqt_traceability}
summarizes the inspections and tests that demonstrate the satisfaction of the
requirements levied on the \partxname.

\begin{table}[htp]
\centering
\caption{DynBodyInit Requirements Traceability}
\label{tab:DynBodyInit:reqt_traceability}
\vspace{1ex}
\begin{tabular}{||l @{\hspace{4pt}} l|l @{\hspace{2pt}} l @{\hspace{4pt}} l|}
\hline
\multicolumn{2}{||l|}{\bf Requirement} &
\multicolumn{3}{l|}{\bf Artifact} \\ \hline\hline
\ref{reqt:DynBodyInit:cartesian_trans} & Translational state &
   Test & \ref{test:DynBodyInit:state_init} &
   State initializations\\[4pt]
\ref{reqt:DynBodyInit:cartesian_rot} & Rotational state &
   Test & \ref{test:DynBodyInit:state_init} &
   State initializations\\[4pt]
\ref{reqt:DynBodyInit:orbelem} & Orbital elements &
   Test & \ref{test:DynBodyInit:state_init} &
   State initializations\\[4pt]
\ref{reqt:DynBodyInit:LVLH} & LVLH frame &
   Test & \ref{test:DynBodyInit:state_init} &
   State initializations\\[4pt]
\ref{reqt:DynBodyInit:NED} & NED frame &
   Test & \ref{test:DynBodyInit:state_init} &
   State initializations\\[4pt]
\ref{reqt:DynBodyInit:is_ready} & Readiness detection &
   Test & \ref{test:DynBodyInit:state_init} &
   State initializations\\
\hline
\end{tabular}
\end{table}
