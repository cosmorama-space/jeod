%%%%%%%%%%%%%%%%%%%%%%%%%%%%%%%%%%%%%%%%%%%%%%%%%%%%%%%%%%%%%%%%%%%%%%%%%%%%%%%%
% BodyAttach_Detach_ivv.tex
% Verification and validation for the MassBody Attach/Detach Sub-Model
%
%%%%%%%%%%%%%%%%%%%%%%%%%%%%%%%%%%%%%%%%%%%%%%%%%%%%%%%%%%%%%%%%%%%%%%%%%%%%%%%%

\chapter{Inspection, Verification, and Validation}\label{ch:\modelpartid:ivv}
\section{Inspection}
\inspection[Attach/detach design]{Design Inspection}
\label{inspect:BodyAttach_Detach:base_class}
Table~\ref{tab:BodyAttach_Detach:virtual_overrides}
portrays the class hierarchy for this sub-model
and identifies which classes provide implementations of the
{\tt initialize()}, {\tt is\_ready()}, and {\tt apply()} methods.
From that table, all \partxname classes derive from BodyAction, some directly,
others indirectly.
Each implementation of the specified methods invokes the overridden parent class
implementation. The two {\tt is\_ready()} methods properly incorporate the result
from the invocation of the overridden {\tt is\_ready()} into the readiness
assessment.

By inspection, the \partxname satisfies
requirements~\ref{reqt:BodyAction:virtual_methods}
and~\ref{reqt:BodyAction:base_class_mandate}.

\begin{table}[htp]
\centering
\caption{Mass Body Attach/Detach Sub-Model Virtual Methods}
\label{tab:BodyAttach_Detach:virtual_overrides}
\vspace{1ex}
\begin{minipage}{0.8\textwidth}
\centering
\begin{tabular}{||l|l|c|c|c|} \hline
{\bf Class} & {\bf Parent Class} &
\multicolumn{3}{|c|} {\bf Class Defines $\cdots$} \\
{\bf Name
  \footnote{To reduce the table width, the leading MassBody on the
  class names is replaced with center dots.
  For example, the class BodyAttachAligned is abbreviated as
  $\cdots$AttachAligned.
  \label{fn:BodyAttach_Detach:virt_meth_table_abbrev}}} &
{\bf Name \footref{fn:BodyAttach_Detach:virt_meth_table_abbrev}} &
{\bf initialize()} & {\bf is\_ready()} & {\bf apply()} \\
\hline \hline
$\cdots$Attach & BodyAction &
  \checkmark &            & \checkmark \\
$\cdots$AttachAligned & $\cdots$Attach &
  \checkmark &            & \checkmark \\
$\cdots$AttachMatrix & $\cdots$Attach &
             &            & \checkmark \\
$\cdots$Detach & BodyAction &
  \checkmark & \checkmark & \checkmark \\
$\cdots$DetachSpecific & BodyAction &
  \checkmark & \checkmark & \checkmark \\
$\cdots$Reattach & BodyAction &
             &            & \checkmark \\
\hline
\end{tabular}
\end{minipage}
\end{table}
\clearpage


\section{Validation}\label{sec:\modelpartid:validation}
This section describes various tests conducted to verify and validate
that the \partxname satisfies the requirements levied against it.
All verification and validation test source code, simulations and procedures
for this sub-model are in the {\tt SIM\_verif\_attach\_mass}, or {\tt SIM\_ref\_attach}, subdirectory of
the JEOD directory {\tt models/dynamics/body\_action/verif}.

\test[Attach/detach]{MassBody Attach/Detach}
\label{test:BodyAttach_Detach:attach_detach}
\begin{description}
\item[Background]
This test exercises the capabilities for this sub-model and
for the  MassBodyInit Class Sub-Model.
The MassBodyInit class initializes the mass properties and mass points
of a MassBody object. The classes defined in this sub-model
provide the ability to attach and detach MassBody objects to/from one another.
\item[Test description]
This test defines 21 run directories.
Test RUN\_05 tests mass properties initialization only;
it does not exercise the requirements of this particular sub-model.
Tests RUN\_01 to RUN\_04 and RUN\_06 to RUN\_11 use the matrix/offset
attachment mechanism while RUN\_101 to RUN\_104 and RUN\_106 to RUN\_111 invoke
identical behavior using the point-to-point attachment mechanism.

\item[Test results]
These tests are described in detail in the
\hypermodelrefinside{MASS}{part}{ivv}.
Per the criteria in that document,
all test cases pass.

\item[Applicable requirements]
This series of tests collectively exercises the capabilities of this sub-model
and the MassBodyInit Class Sub-Model. These tests demonstrate
requirements~\ref{reqt:MassBodyInit:mass_ppty_init}
to~\ref{reqt:MassBodyInit:mass_pnts_init}
and~\ref{reqt:BodyAttach_Detach:attach_support}
to~\ref{reqt:BodyAttach_Detach:detach_specific}.
\end{description}

\test[Kinematic Attach/detach]{Kinematic Attach}
\label{test:BodyAttach_Detach:ref_attach}
\begin{description}
\item[Background]
This test exercises the capabilities for this sub-model's kinematic RefFrame attach capability. For the RefFrame
attachments, we must demonstrate that the resulting attachment zeroes any previous velocities w.r.t. the parent frame
and that mass tree is not impacted by this type of attachment.
\item[Test description]
Test RUN\_ref\_attach\_pt2pt attaches a Dynbody to a planet reference frame using the point-to-point attachment.
Test RUN\_ref\_attach\_matrix attaches a DynBody to a planet reference frame using the matrix/offset
attachment mechanism.

\item[Test results]
All test cases pass.

\item[Applicable requirements]
This series of tests collectively exercises the capabilities of this sub-model. These tests demonstrate
requirements~\ref{reqt:BodyAttach_Detach:ref_attach}
\end{description}


\section{Requirements Traceability}
Table~\ref{tab:BodyAttach_Detach:reqt_traceability}
summarizes the inspections and tests that demonstrate the satisfaction of the
requirements levied on the \partxname.

\begin{table}[htp]
\centering
\caption{BodyAttach\_Detach Requirements Traceability}
\label{tab:BodyAttach_Detach:reqt_traceability}
\vspace{1ex}
\begin{tabular}{||l @{\hspace{4pt}} l|l @{\hspace{2pt}} l @{\hspace{4pt}} l|}
\hline
\multicolumn{2}{||l|}{\bf Requirement} &
\multicolumn{3}{l|}{\bf Artifact} \\ \hline\hline
\ref{reqt:BodyAttach_Detach:attach_support} & Attach framework &
   Test & \ref{test:BodyAttach_Detach:attach_detach} &
   Attach/detach\\[4pt]
\ref{reqt:BodyAttach_Detach:attach} & Attach MassBody/DynBody objects &
   Test & \ref{test:BodyAttach_Detach:attach_detach} &
   Attach/detach\\[4pt]
\ref{reqt:BodyAttach_Detach:detach_immediate} & Detach from parent &
   Test & \ref{test:BodyAttach_Detach:attach_detach} &
   Attach/detach\\[4pt]
\ref{reqt:BodyAttach_Detach:detach_specific} & Specific detach &
   Test & \ref{test:BodyAttach_Detach:attach_detach} &
   Attach/detach\\[4pt]
\ref{reqt:BodyAttach_Detach:ref_attach} & Specific detach &
   Test & \ref{test:BodyAttach_Detach:attach_detach} &
   Attach/detach\\
\hline
\end{tabular}
\end{table}
