%%%%%%%%%%%%%%%%%%%%%%%%%%%%%%%%%%%%%%%%%%%%%%%%%%%%%%%%%%%%%%%%%%%%%%%%%%%%%%%%
% overview_intro.tex
% Intro chapter of the overview part of the Body Action Model
%
%%%%%%%%%%%%%%%%%%%%%%%%%%%%%%%%%%%%%%%%%%%%%%%%%%%%%%%%%%%%%%%%%%%%%%%%%%%%%%%%
\chapter{Introduction}\hyperdef{part}{intro}{}
\label{ch:overview:intro}

\section{Purpose and Objectives of the \ModelDesc}
\label{sec:overview:purp}
The \ModelDesc comprises several classes that provide various mechanisms
for setting aspects of instances of the MassBody and DynBody class and
 their derived classes.
The provided capabilities include:
\begin{itemize}
\item Setting the mass properties of a MassBody object,
\item Attaching and detaching MassBody or DynBody objects to/from one another,
\item Setting the state of a DynBody object, and
\item Changing the frame in which a DynBody object's state is propagated.
\end{itemize}

\section{Context within JEOD}
The following document is parent to this document:
\begin{itemize}
\item \hyperJEOD
\end{itemize}

The \ModelDesc forms a component of the dynamics suite of
models within \JEODid. It is located at
models/dynamics/body\_action.


%%%%%%%%%%%%%%%%%%%%%%%%%%%%%%%%%%%%%%%%%%%%%%%%%%%%%%%%%%%%%%%%%%%%%%%%%%%%%%%%
% change_history.tex
% History of dyn_manager.pdf
% Add a new line to the table for each update. 
%%%%%%%%%%%%%%%%%%%%%%%%%%%%%%%%%%%%%%%%%%%%%%%%%%%%%%%%%%%%%%%%%%%%%%%%%%%%%%%%

\section{Documentation History}
\begin{tabular}{||l|l|l|l|} \hline
{\bf Author } & {\bf Date} & {\bf Revision} & {\bf Description} \\ \hline \hline
 David Hammen & November, 2009 & 1.0 & Initial Version \\
 David Hammen & April, 2010 & 1.1 & User guide and IVV chapters updated \\
 Andrew Spencer & October, 2010 & 1.2 & Reflected new inheritence tree \\
\hline
\end{tabular}



\section{Documentation Organization}
\label{sec:overview:docorg}
This document is formatted in accordance with the
NASA Software Engineering Requirements Standard~\cite{NASA:SWE}.

The document comprises several parts:
\begin{description}
\item[Overview]
This first part of the document provides an
overview of the \ModelDesc.
\item[BodyAction]
Part \ref{part:BodyAction} describes the class BodyAction, the
base class for the \ModelDesc.
Each subsequent part describes a collection of related classes
that derive from this base class.
\item[MassBodyInit]
Part \ref{part:MassBodyInit} describes the class MassBodyInit, which
initializes the mass properties and mass points of a MassBody object.
\item[Attach/Detach]
Part \ref{part:BodyAttach_Detach} describes the classes that cause
MassBody and DynBody objects to attach to and detach from one another.
\item[DynBodyInit]
Part \ref{part:DynBodyInit} describes the class DynBodyInit and its
derived classes, which initializes the state of a DynBody object.
\item[FrameSwitch]
Part \ref{part:DynBodyFrameSwitch} describes the class DynBodyFrameSwitch,
which switches the frame in which a DynBody object's state is propagated.
\end{description}

Each part is organized into a similar structure, comprising the following
chapters in order:

\begin{description}
\item[Introduction] -
This introduction describes the objective and purpose of the \ModelDesc. The
introductions for subsequent parts describe the objective and purpose of
the classes that are the subject of that part. To avoid undo repetition, the
introductory chapters of the subsequent parts are of an abbreviated nature.

\item[Product Requirements] -
The requirements chapter in this overview part of the document describes
the requirements on the \ModelDesc as a whole.
The requirements chapters of subsequent parts describe the requirements
that pertain to the classes that are the subject of that specific part.

\item[Product Specification] -
The specification chapter in this overview part of the document describes
the architecture and design of the \ModelDesc as whole.
Where applicable,
The specification chapter in the subsequent parts describes the underlying
theory, architecture, and design of the subject classes of the part in detail.
Where applicable,
the product specification chapters are organized into sections
as follows:
\begin{itemize}
 \item Conceptual Design.
 \item Mathematical Formulations (where applicable).
 \item Detailed Design.
 \item Version Inventory (Part I - Overview only).
\end{itemize}


\item[User Guide] -
Describes how to use the \ModelDesc as a whole (this part) /
sub-models (subsequent parts).
Where applicable,
the User Guide chapters are organized into sections
that represent the following JEOD-defined user types:
\begin{itemize}
 \item Analysts or users of simulations (Analysis).
 \item Integrators or developers of simulations (Integration).
 \item Model Extenders (Extension).
\end{itemize}

\item[Inspection, Verification, and Validation] -
The inspection, verification, and validation (IV\&V) chapter
in this overview part of the document
shows the traceability of the high-level requirements to the sub-model
detailed requirements.
The IV\&V chapters in the subsequent parts describes
the verification and validation procedures and results for the subject
classes of the part.
\end{description}
