%%%%%%%%%%%%%%%%%%%%%%%%%%%%%%%%%%%%%%%%%%%%%%%%%%%%%%%%%%%%%%%%%%%%%%%%%%%%%%%%
% BodyAction_spec.tex
% Specification of the BodyAction class
%
%%%%%%%%%%%%%%%%%%%%%%%%%%%%%%%%%%%%%%%%%%%%%%%%%%%%%%%%%%%%%%%%%%%%%%%%%%%%%%%%

\chapter{Product Specification}\label{ch:\modelpartid:spec}

\section{Conceptual Design}
All of the instantiable \ModelDesc classes ultimately derive from the BodyAction
base class. This base class by itself does not change a single aspect of
a subject MassBody, DynBody, or derived class object. Making changes to a body is the
responsibility of the various classes that derive from the base BodyAction
class. The base class provides a common framework for describing actions to be
performed. This framework includes
\begin{itemize}
\item Descent from a common base class. This common basis enables forming a
collection of body actions in the form of base class pointers. The DynManager
class does exactly this. The \ModelDesc was explicitly architected to enable
this treatment.
\item Base class member data, including a pointer to the {\tt mass\_subject} MassBody object or
{\tt dyn\_subject} DynBody object that is the subject of the BodyAction object.
\item Base class member functions that specify how to operate on the BodyAction
object. In particular, a trio of virtual methods described below
form a standard the basis by which derived classes can extend this base class.
\end{itemize}

\section{Mathematical Formulations}
N/A
\section{Detailed Design}
The base class defines three virtual methods that together form the
function interface by which an external user of the \ModelDesc functionally
interacts with BodyAction objects. These methods are
\begin{description}
\item[initialize]
Each \ModelDesc object must be initialized.
This initialization is the task of the overridable {\tt initialize()}
method. This method initializes a BodyAction object.
It does not alter the subject object.
\item[is\_ready]
The {\tt is\_ready()} method indicates whether the BodyAction object is
ready to perform its action on the subject object, but does not alter the object.
For example, a DynBodyInitLvlhState object sets parts of the subject
DynBody object's state based on the Local Vertical, Local Horizontal (LVLH)
frame defined by some reference DynBody object.
The DynBodyInitLvlhState object remains in a not-ready state
until the reference body's position and velocity vector have been set.
\item[apply]
The {\tt apply()} method performs the action on the subject object.
\end{description}

The BodyActionMessages class simply encapsulates labels for use with the
JEOD Message Handler. The model reports the following types of messages:
\begin{description}
\item[fatal\_error]
      Issued when performing an action results in an error return from the
      method performing the action.
\item[illegal\_value]
      Issued when a simple type (e.g. an enum) has an illegal value.
\item[invalid\_name]
      Issued when a name is invalid (NULL, empty, or does not name an object
      of the specified type).
\item[invalid\_object]
      Issued when a pointer points to an object of the wrong type.
\item[null\_pointer]
      Error issued when a pointer is required but was not provided.
\item[not\_performed]
      Issued when a BodyAction cannot be run.
\item[trace]
      Debug message issued to trace BodyAction actions.
\end{description}
