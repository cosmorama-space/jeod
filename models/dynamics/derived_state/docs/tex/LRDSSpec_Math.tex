%%%%%%%%%%%%%%%%%%%%%%%%%%%%%%%%%%%%%%%%%%%%%%%%%%%%%%%%%%%%%%%%%%%%%%%%%%%%%%%%%
%
% Purpose:  Mathematical Formulation part of Product Spec for the LVLH
%             relative derived state model.
%
% 
%
%%%%%%%%%%%%%%%%%%%%%%%%%%%%%%%%%%%%%%%%%%%%%%%%%%%%%%%%%%%%%%%%%%%%%%%%%%%%%%%%

the \LRDSDesc\ depends on the \textit{LVLH Frame model} to supply a
rectangular LVLH reference frame.
In this discussion, we will consider two reference frames, $Q$ (subject) and
$R$ (target) with origin respectively located at $\vec Q$ and $\vec R$
where both $Q$ and $R$ are children of planet-centered inertial. Suppose
that $R$ is the LVLH frame associated with $\vec R$ with respect to a
reference planet.  Note that in order to construct a canonical LVLH frame at
$\vec R$, the angular momentum $\vec H = \vec R \times \dot{\vec{R}}$
must be non-zero. Let $\hat{\vec{H}}$ be the
unit vector in the direction of $\vec H$, thus $\hat{\vec{H}} =
\frac{\vec H}{|\vec H|}$. From the documentation of the LVLH Frame Model,
found in models/utils/lvlh\_frame/docs/lvlh\_frame.pdf, the axis unit vectors
of $R$ $\hat{\vec{X}}$, $\hat{\vec{Y}}$ and $\hat{\vec{Z}}$ are given by
\begin{eqnarray}
\hat{\vec{Z}} & = & -\frac{\vec R}{|\vec R|} \\ \nonumber
\hat{\vec{Y}} & = & -\hat{\vec{H}} \\ \nonumber
\hat{\vec{X}} & = & \hat{\vec{Y}} \times \hat{\vec{Z}}
\label{eq:lvlh_axes}
\end{eqnarray}
In the rectangular case, we obtain the relative state by simply invoking the
\textit{~compute\_relative\_state} method of the subject frame with respect to the
rectilinear LVLH frame ($R$) which is generated by the \textit{LvlhFrame} class and is
available from the dynamics manager in the reference frame tree. Specifically,
the function call
~$Q$.\textit{compute\_relative\_state}($R$,\textit{recti\_state}) provides
the state of $Q$ expressed in $R$.

In general, tracking the action of the \textit{compute\_relative\_state}
method of the \textit{RefFrame} class can be challenging, but since $Q$ and
$R$ have a common parent $I$, the planetary inertial frame, all the work is
done by a single call to the \textit{decr\_left} method of the
\textit{RefFrameStateClass}. For rectilinear coordinates, it is not necessary
to understand the inner workings of the transformations because they are
handled automatically by \textit{compute\_relative\_state}. However, as we
shall see in ~\ref{subsec:curvi_coords},
we will replace the frame $R$ by the \textit{RefFrameState} $R_q$ that differs
from $R$ by a $Y$-rotation through the \emph{phase angle} $\theta$. For technical
reasons, $R_q$ will not be a reference frame in the JEOD sense, so we will
need to manually implement the calculations that would otherwise be done by
\textit{RefFrameState::decr\_left}. Here are the equations as applied to
generic \textit{RefFrameStates} $A$, $B$ and $C$.
\begin{eqnarray}
T_{B:C} & = & T_{A:C}T_{A:B}^T \\ \nonumber
\omega_{B:C} & = & \omega_{A:C} - T_{B:C}\omega_{A:B} \\ \nonumber
x_{B:C} & = & T_{A:B}(x_{A:C} - x_{A:B}) \\ \nonumber
v_{B:C} & = & T_{A:B}(v_{A:C} - v_{A:B}) - \omega_{A:B} \times x_{B:C}
\label{eq:decr_left}
\end{eqnarray}

\subsection{Curvilinear Coordinates}
\label{subsec:curvi_coords}
The curvilinear system replaces the LVLH $X$-axis by the circle centered at the
planet center and tangent to the LVLH $X$-axis at the origin $\vec R$ of $R$,
and the $x'$ coordinate of the curvilinear system represents distance
measured on this circle.  The curvilinear $Z$ coordinate is measured radially
from the circle with the positive $Z$ direction being toward the planet.
The $Y$ coordinate axis of the curvilinear system is the same as that of $R$.

The \LRDSDesc\ obtains a curvilinear representation of
$Q$ with respect to $R$ by first computing the rectilinear LVLH state and then
invoking \textit{~convert\_rect\_to\_circ} on the rectilinear LVLH state
(\textit{recti\_state}).

\subsection{Translational State}
\label{subseq:trans}
The translational part of the curvilinear state is defined by common practice,
and as such, is not an exact analog of the translational component
of a rectilinear state. Where analogs exist, they will be identified and
explained.

A key parameter associated with the curvilinear representation is the
\emph{phase angle} $\theta$ which
describes the angular separation of $\vec Q$ and $\vec R$ as measured in
the orbital plane of $\vec R$. A second key
quantity is the vector $\vec Q'$ which is the projection of $\vec Q$ on the
orbital plane.

The incoming parameter \textit{~recti\_state} contains some important fields
which are extracted as follows:
\begin{eqnarray}
x & = & \mbox{recti\_state.trans.position[0]} \\ \nonumber
y & = & \mbox{recti\_state.trans.position[1]} \\ \nonumber
z & = & \mbox{recti\_state.trans.position[2]} \\ \nonumber
\dot x & = & \mbox{recti\_state.trans.velocity[0]} \\ \nonumber
\dot y & = & \mbox{recti\_state.trans.velocity[1]} \\ \nonumber
\dot z & = & \mbox{recti\_state.trans.velocity[2]}
\label{eq:recti_state}
\end{eqnarray}

\subsubsection{Projecting $\vec Q$ onto the $X$-$Z$ Plane of $R$}
We first obtain the projection $\vec Q'$ of $\vec Q$ onto the $X$-$Z$ plane
of $R$. 
\begin{equation}
\vec Q' = \vec R + x\hat{\vec{X}} + z\hat{\vec{Z}}
\label{eq:QPrime}
\end{equation}
\noindent where $\hat{\vec{X}}$ and $\hat{\vec{Z}}$ are as in ~\ref{eq:lvlh_axes}.

\subsubsection {The Phase Angle $\theta$}
The points $0$, $\vec Q'$ and $\vec R - z\hat{\vec{Z}}$ form a right triangle in
the $X$-$Z$ plane of $R$, and the angle $\theta$ at the origin is said to be
the \emph{phase angle} of $\vec Q$ with respect to $\vec R$.
If $r=|\vec R|$, we then have the following equations for $x'$, $y'$, $z'$, $q'$ and
$\theta$, the position coordinates in the curvilinear system, $|\vec Q'|$ and
the phase angle.
\begin{eqnarray}
q' & = & \sqrt {x^2 + (r-z)^2} \\ \nonumber
\theta & = & \tan^{-1} \frac{x}{r-z} \\ \nonumber
x' & = & r\theta \\ \nonumber
y' & = & y \\ \nonumber
z' & = & r-q'
\label{eq:curvi_position}
\end{eqnarray}

The translational velocity in the curvilinear system is the same as the
rectilinear LVLH velocity except that it is expressed in a frame rotated by
$\theta$. We effectively express the curvilinear velocity 
with respect to a frame co-moving with $R$ but rotated to the LVLH of the
point $\vec Q'$. Of course, this is physical fiction because $\vec Q'$ does
not co-move with $\vec R$, and this introduces some small physical
inconsistencies which are described in ~\ref{subsec:rot}. Let us refer to the
co-moving frame as $R_q$. We then introduce the $Y$-rotation $M_\theta$ which,
in the parlance of ~\ref{eq:decr_left} can be identified as $T_{R:R_q}$.
\begin{equation}
M_\theta = \left (\begin{array}{ccc}
\cos \theta & 0 & \sin \theta \\
0 & 1 & 0 \\
-\sin \theta & 0 & \cos \theta \end{array} \right )
\label{eq:m_theta}
\end{equation}
Then the velocity field of \textit{~rel\_state} is computed as follows:
\begin{equation}
\mbox{~rel\_state.trans.velocity} = M_\theta \cdot \mbox{~recti\_state.trans.velocity}
\label{eq:curvi_vel}
\end{equation}

\subsection{Rotational State}\label{subsec:rot}
For attitude and attitude rate (angular velocity) to be meaningful,
they must be relative to some coordinate system. In the rectangular case, the
rectangular LVLH frame $R$ served as the reference. In the curvilinear case,
$R_q$ will serve as the reference. Note that we have specified
a velocity (same as $R$) and an orientation (R rotated by $\theta$), but not an
origin. If we place the origin of $R_q$ at $\vec Q'$, then we are inconsistent
because $\vec Q'$ and $\vec R$ have non-zero relative motion. If instead, we
place the origin of $R_q$ at $\vec R$, then any rotation of $R_q$ relative to
$R$ will introduce an $\vec\omega \times \vec X$ term into ~\ref{eq:curvi_vel}.
Practically speaking, this issue can be safely ignored. We certainly want
$R_q$ to co-move with $R$ so that the curvilinear velocity will "see" the
translational motion of $Q$ with respect to $R$. While we do account for a
rotational motion of $R_q$ with respect to $R$, in virtually all
operational scenarios such as rendezvous maneuvers, the differential rotation
rate is on the order of $10^{-6}$ radians per second. The default setting is
to ignore it.

now it is time to use the equations of ~\ref{eq:decr_left} to understand how
using $R_q$ rather than $R$ as a reference will
change attitude and angular velocity calculations. Note that
the incoming parameter \textit{recti\_state} already contains the $R$-relative
quantities, so we only need account for the differences introduced by using a
different reference.

The easiest place to start is with the attitude calculation, the first
identity of ~\ref{eq:decr_left}. If we let $A = I$, $B = R$, and
$C = Q$, then
\begin{eqnarray}
\mbox{recti\_state.rot.T\_parent\_this} = T_{R:Q} = T_{I:Q}T_{I:R}^t, where \\ \nonumber
T_{I:Q}T_{I:R}^t = Q\mbox{.rot.T\_parent\_this} \cdot R\mbox{.rot.T\_parent\_this}^t
\label{eq:att1}
\end{eqnarray}

Further, $T_{R:R_q} = M_\theta$, 
so $T_{I:R_q} = M_\theta T_{I:R}$. Now applying ~\ref{eq:decr_left} with
$A = I$, $B = R_q$ and $C = Q$, then
\begin{equation}
T_{R_q:Q} = T_{I:Q}T_{I:R_q}^t = (T_{I:Q}T_{I:R}^t)M_\theta^t = 
\mbox{recti\_state.rot.T\_parent\_this}M_\theta^t
\label{eq:att2}
\end{equation}

The next part to tackle is the angular velocity as described by the second
identity of ~\ref{eq:decr_left}. Again, let $A=I$, $B=R$ and $C=Q$. This gives
\begin{equation}
\omega_{R:Q} = \omega_{I:Q} - T_{R:Q}\omega_{I:R}
\label{eq:att3}
\end{equation}
If instead, we let the middle frame $B = R_q$ we have
\begin{equation}
\omega_{R_q:Q} = \omega_{I:Q} - T_{R_q:Q}\omega_{I:R_q}
\label{eq:att4}
\end{equation}
If we subtract ~\ref{eq:att3} from ~\ref{eq:att4} we derive a formula for the
difference between the rectilinear angular velocity \textit{recti\_state.rot.ang\_vel\_this}
and the curvilinear value \textit{rel\_state.rot.ang\_vel\_this}. Specifically,
\begin{eqnarray}
\mbox{rel\_state.rot.ang\_vel\_this} = \mbox{recti\_state.rot.ang\_vel\_this} + T_{R:Q}\omega_{I:R} - T_{R_q:Q}\omega_{I:R_q} \\ \nonumber
= T_{R:Q}\omega_{I:R} - T_{I:Q}M_\theta^t\omega_{I:R_q} \\ \nonumber
= T_{R:Q}(\omega_{I:R} - M_\theta^t\omega_{I:R_q}) \\ \nonumber
= T_{R:Q}(\omega_{I:R}-\omega_{I:R_q})
\label{eq:att5}
\end{eqnarray}
\noindent where we take advantage of the fact the $M_\theta^t$ is a
$Y$-rotation, and thus leaves $\omega_{I:R_q}$ unchanged because $\omega_{I:R_q}$
and, for that matter, $\omega_{I:R}$ are of the form
$\left ( \begin{array}{c} 0 \\ \omega \\ 0 \end{array} \right )$. Moreover, 
$\omega_{I:R} - \omega_{I:R_q} =
\left ( \begin{array}{c} 0 \\ \dot\theta \\ 0 \end{array} \right )$.
Therefore
\begin{equation}
\mbox{rel\_state.rot.ang\_vel\_this} = \mbox{recti\_state.rot.ang\_vel\_this} +
\mbox{recti\_state.T\_parent\_this}
\left ( \begin{array}{c} 0 \\ \dot\theta \\ 0 \end{array} \right )
\label{eq:att6}
\end{equation}

It is straightforward to differentiate the expression for $\theta$ in
~\ref{eq:curvi_position} with respect to time if the derivatives of $r$, $x$
and $z$ are available. The derivatives of $x$ and $z$ are part of the
translational portion of \textit{~recti\_state},
and $\dot r = \frac{\vec{R} \cdot \dot{\vec{R}}}{r}$, therefore
\begin{equation}
\dot\theta =
\frac{\dot x(r-z)-x(\frac{\vec{R} \cdot \dot{\vec{R}}}{r}-\dot z)}{x^2 + (r-z)^2}
\label{eq:thetadot}
\end{equation}
\noindent which, with ~\ref{eq:att6}, completes the calculation of
\textit{~rel\_state.rot.ang\_vel\_this}.

As already noted, accounting for $\dot\theta$ is not necessary for many
applications and is not strictly consistent with physics. For these reasons,
the $\dot\theta$ correction is turned off by default in JEOD, and in that case, the
rectilinear and curvilinear angular velocities are identical. The calculations
are only included for completeness.

\subsection{\textit{convert\_circ\_to\_rect}}
Now that the equations of \textit{convert\_rect\_to\_circ} have been explained,
their inversions will be presented with minimal comment.  The
\textit{convert\_circ\_to\_rect} method
takes a \textit{RefFrameState} argument, say
\textit{~curvi\_state}, and fills in the \textit{~rel\_state}
field of \textit{RelativeDerivedState} with the rectilinear
equivalents. It is assumed that 
\textit{~curvi\_state} shares the same conventions for curvilinear coordinates
described in ~\ref{subsec:curvi_coords}. We also have access to the target frame $R$,
thus $r$ and $\vec R$ are known. Since we know the curvilinear position
$\left (\begin{array}{c} x' \\ y' \\ z' \end{array} \right ) =
\mbox{~curvi\_state.trans.position}$,
we can invert the equations of ~\ref{subsec:curvi_coords} as follows:
\begin{eqnarray}
\theta & = \frac{x'}{r} \\ \nonumber
q' & = & r - z' \\ \nonumber
x & = & q' \sin\theta \\ \nonumber
y & = & y' \\ \nonumber
z & = & r - q' \cos\theta
\label{eq:recti_position}
\end{eqnarray}

We can invert~\ref{eq:curvi_vel} to obtain the rectilinear LVLH velocity.
\begin{equation}
\mbox{~rel\_state.trans.velocity} =
M^t_\theta\mbox{~curvi\_state.trans.velocity}
\label{eq:recti_vel}
\end{equation}
Similarly, we can invert~\ref{eq:att2} to obtain the orientation of the
rectilinear LVLH frame.
\begin{equation}
\mbox{~rel\_state.rot.T\_parent\_this} =
\mbox{~curvi\_state.t\_parent\_this}M_\theta
\label{eq:recti_rot}
\end{equation}

Now that the rectilinear translational state and orientation of $Q$ with
respect to $R$ are known, it is trivial to invert~\ref{eq:att6} for the
rectilinear angular velocity.
\begin{equation}
\mbox{rel\_state.rot.ang\_vel\_this} = \mbox{curvi\_state.rot.ang\_vel\_this} -
\mbox{rel\_state.rot.T\_parent\_this} \left ( \begin {array}{c}
0 \\ \dot\theta \\ 0 \end{array}\right )
\label{eq:recti_ang_vel}
\end{equation}
