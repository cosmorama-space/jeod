%%%%%%%%%%%%%%%%%%%%%%%%%%%%%%%%%%%%%%%%%%%%%%%%%%%%%%%%%%%%%%%%%%%%%%%%%%%%%%%%%
%
% Purpose:  Analysis part of User's Guide for the Euler model
%
% 
%
%%%%%%%%%%%%%%%%%%%%%%%%%%%%%%%%%%%%%%%%%%%%%%%%%%%%%%%%%%%%%%%%%%%%%%%%%%%%%%%%

% \section{Analysis}

\label{sec:euleruseranalysis}
\subsection{Identifying the \EulerDescT}
If the Euler angles have been included in the simulation, there will be an instance of \textit{EulerDerivedState} located in the S\_define file.  This would typically be found in the vehicle object.  There should be an accompanying call to an initialization routine, which takes a reference to the \textit{subject\_body} as one if its inputs, and may also take a reference to a reference-frame, although this is not necessary.  There will also be an accompanying call to an update function.  The essential variable \textit{sequence}, the order in which the angles should be assigned to each axis, is defined elsewhere, often in the input file.

Example:
\begin{verbatim}
sim_object{
dynamics/derived_state:    EulerDerivedState example_of_euler_angles;

(initialization) dynamics/derived_state:
example_of_rel_state_object.example_of_euler_angles.initialize (
    Inout DynBody &      subject_body = vehicle_1.dyn_body,
    Inout DynManager &   dyn_manager  = manager_object.dyn_manager);
    
{environment} dynamics/derived_state:
example_of_rel_state_object.example_of_euler_angles.update ( )

} example_of_rel_state_object;
\end{verbatim}

The input file may contain an entry comparable to:
\begin{verbatim}
example_of_rel_state_object.example_of_euler_angles.sequence = Roll_Pitch_Yaw;
\end{verbatim}
This variable can take one of the following values:
\begin{itemize}
\item{} Roll\_ Pitch\_ Yaw
\item{} Roll\_ Yaw\_ Pitch
\item{} Pitch\_ Yaw\_ Roll
\item{} Pitch\_ Roll\_ Yaw
\item{} Yaw\_ Roll\_ Pitch
\item{} Yaw\_ Pitch\_ Roll
\end{itemize}

\subsection{Editing the \EulerDescT}
The sequence is open to edit by the analyst.

\subsection{Output Data}
The direct output from the \EulerDesc\ comprises only the two vectors of angles,
\textit{ref\_body\_angles} (the Euler angles from the reference frame to the subject frame), and
\textit{body\_ref\_angles} (the Euler angles to the reference frame from the subject frame).
\begin{verbatim}
example_of_rel_state_object.example_of_euler_angles.ref_body_angles[0-2]
example_of_rel_state_object.example_of_euler_angles.body_ref_angles[0-2]
\end{verbatim}



