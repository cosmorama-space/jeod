%%%%%%%%%%%%%%%%%%%%%%%%%%%%%%%%%%%%%%%%%%%%%%%%%%%%%%%%%%%%%%%%%%%%%%%%%%%%%%%%%
%
% Purpose:  Verification part of V&V for the DerivedState model
%
% 
%
%%%%%%%%%%%%%%%%%%%%%%%%%%%%%%%%%%%%%%%%%%%%%%%%%%%%%%%%%%%%%%%%%%%%%%%%%%%%%%%%

% \section{Verification}

%%% code imported from old template structure
\inspection{Top-level Requirement}\label{inspect:top_level}
 This document structure, the code, and associated files have been inspected, and together satisfy requirement~\ref{reqt:top_level}.

\inspection{Minimum Functionality}\label{inspect:DerivedState_min_func}
 This model provides:
\begin{enumerate}

 \item The following reference frames:
\begin{enumerate}
 \item Local-Vertical-Local-Horizontal (LVLH)
\item North-East-Down(NED)
\end{enumerate}

\item The translational states in the following formats, using reference frames defined elsewhere in \JEODid:
\begin{enumerate}
 \item Classical Orbital Elements
 \item Planetary altitude-latitude-longitude
\end{enumerate}

\item The rotational states utilizing a sequence of 3 rotation angles about coordinate axes, 

\item The Solar Beta angle associated with any given state, 

\item The ability to express the state of one reference frame in terms of another.

\end{enumerate}  

Together, these capabilities satisfy requirement~\ref{reqt:DerivedState_min_func}.