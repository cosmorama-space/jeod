%%%%%%%%%%%%%%%%%%%%%%%%%%%%%%%%%%%%%%%%%%%%%%%%%%%%%%%%%%%%%%%%%%%%%%%%%%%%%%%%%
%
% Purpose:  Conceptual part of Product Spec for the LVLH model
%
% 
%
%%%%%%%%%%%%%%%%%%%%%%%%%%%%%%%%%%%%%%%%%%%%%%%%%%%%%%%%%%%%%%%%%%%%%%%%%%%%%%%%


%\section{Conceptual Design}
The \LVLHDesc\ is used to express the state of some vehicle with respect to a reference frame that is aligned with the local vertical and horizontal axes at some other point (referred to below as the defining point), which is often another vehicle.  A planetary object (referred to below as the specified planet) is used to define ``vertical'', and also to orient the axes in the horizontal plane.

The axes are defined as:

\begin{align*}
 \hat i &= \hat j \times \hat k \\
 \hat j &= - \hat h \\
 \hat k &= - \hat r
\end{align*}

$\hat h$ represents the unit vector oriented with the angular momentum vector, associated with the translational motion of the defining point with respect to the specified planet.

$\hat r$ represents the unit vector oriented with the radial vector, from the specified planet to the defining point.

The LVLH reference frame fits into the reference frame tree as a child of the inertial reference frame associated with the specified planet.

The LVLH reference frame provided by this derived state model is a recti-linear LVLH representation.
