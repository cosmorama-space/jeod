
%%%%%%%%%%%%%%%%%%%%%%%%%%%%%%%%%%%%%%%%%%%%%%%%%%%%%%%%%%%%%%%%%%%%%%%%%%%%%%%%%
%
% Purpose:  Detailed part of Product Spec for the SolarBeta model
%
% 
%
%%%%%%%%%%%%%%%%%%%%%%%%%%%%%%%%%%%%%%%%%%%%%%%%%%%%%%%%%%%%%%%%%%%%%%%%%%%%%%%%

\section{Detailed Design}
See the \href{file:refman.pdf}{Reference Manual}\cite{derivedstatebib:ReferenceManual} for a summary of member data and member methods for all classes.  

\subsection{Process Architecture}
The process architecture for the \SolarBetaDesc\ is trivial, comprising operationally independent methods.

\subsection{Functional Design}
This section describes the functional operation of the methods in each class.

The \SolarBetaDesc\ contains only one class:
\begin{itemize}
\classitem{SolarBetaDerivedState}
\textref{DerivedState}{ref:DerivedState}

This contains only the methods \textit{initialize} and \textit{update}:
\begin{enumerate}
\funcitem{initialize}
The initialization process comprises the following steps:
\begin{enumerate}
\item{} The generic DerivedState initialization routine is called to establish the naming convention of the reference object (i.e., the planet about which the vehicle is orbiting) and state identifier.
\item{} The registry of active bodies, maintained by the Dynamics Manager (see \href{file:\JEODHOME/models/dynamics/dyn_manager/docs/dyn_manager.pdf}{\em Dynamics Manager Documentation}~\cite{dynenv:DYNMANAGER}) is updated to ensure that the ephemeris of the sun and the reference body are both being updated.
\end{enumerate}

\funcitem{update}
The process at update is described entirely in the \reftext{Mathematical Formulations}{sec:solarbetamath} section.


\end{enumerate}
\end{itemize}



