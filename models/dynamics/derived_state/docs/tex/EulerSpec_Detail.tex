
%%%%%%%%%%%%%%%%%%%%%%%%%%%%%%%%%%%%%%%%%%%%%%%%%%%%%%%%%%%%%%%%%%%%%%%%%%%%%%%%%
%
% Purpose:  Detailed part of Product Spec for the Euler model
%
% 
%
%%%%%%%%%%%%%%%%%%%%%%%%%%%%%%%%%%%%%%%%%%%%%%%%%%%%%%%%%%%%%%%%%%%%%%%%%%%%%%%%

\section{Detailed Design}

See the \href{file:refman.pdf}{Reference Manual}\cite{derivedstatebib:ReferenceManual} for a summary of member data and member methods for all classes.  

\subsection{Process Architecture}
The process architecture for the \EulerDesc\ is trivial, comprising entirely independent methods.

\subsection{Functional Design}
This section describes the functional operation of the methods in each class.

The \EulerDesc\ contains only one class:
\begin{itemize}
\classitem{EulerDerivedState}
\textref{DerivedState}{ref:DerivedState}

This contains the methods \textit{initialize}, and \textit{update}:
\begin{enumerate}

\funcitem{initialize}
There are two processes for initialization, depending on the relationship in the Reference Frame Tree between the subject-body reference frame, and the exterior reference frame.  If the former is not a child of the latter, then it becomes necessary to specify which frame is to be used as the external reference frame.  In this case, a separate initialization process allows for the specification of that frame.  Otherwise, the initialization process simply calls the generic DerivedState initialization routine to establish the naming convention of the reference object (i.e., the planet about which the vehicle is orbiting) and state identifier.

\funcitem{update}
If the Euler angles are to be computed relative to a frame other than the parent frame, that relative state is first calculated.  Then the Trick method \textit{euler\_matrix} is called for both reference-to-subject and subject-to-reference frame transformations.

\end{enumerate}
\end{itemize}
