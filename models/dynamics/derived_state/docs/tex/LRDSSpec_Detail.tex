
%%%%%%%%%%%%%%%%%%%%%%%%%%%%%%%%%%%%%%%%%%%%%%%%%%%%%%%%%%%%%%%%%%%%%%%%%%%%%%%%%
%
% Purpose:  Detailed part of Product Spec for the LVLH relative derived state model
%
% 
%
%%%%%%%%%%%%%%%%%%%%%%%%%%%%%%%%%%%%%%%%%%%%%%%%%%%%%%%%%%%%%%%%%%%%%%%%%%%%%%%%

%\section{Detailed Design}
See the \href{file:refman.pdf}{Reference Manual}\cite{derivedstatebib:ReferenceManual} for a summary of member data and member methods for all classes.  

\subsection{Process Architecture}
The process architecture for the \LRDSDesc\ is trivial; the \LRDSDesc\
comprises the usual \textit{initialize} and \textit{update} methods
and two public methods for conversion between rectilinear and circular
curvilinear coordinates.

\subsection{Functional Design}
This section describes the functional operation of the methods in each class.

The \LRDSDesc\ contains only one class:
\begin{itemize}
\classitem{LvlhRelativeDerivedState}
\textref{RelativeDerivedState}{ref:RelativeDerivedState}

This contains the public methods \textit{convert\_rect\_to\_circ},
\textit{convert\_circ\_to\_rect}, \textit{initialize}, and \textit{update}:
\begin{enumerate}

\funcitem{convert\_rect\_to\_circ}
This method follows the process outlined in the
\textref{Mathematical Formulations section}{sec:Lrdsmath} for
converting a state from rectilinear coordinates to the circular curvilinear
system also described in ~\textref{math section}{sec:Lrdsmath}.

\funcitem{convert\_circ\_to\_rect}
This method follows the process outlined in the
\textref{Mathematical Formulations section}{sec:Lrdsmath} for
converting a state from the circular curvilinear coordinates to
standard rectilinear LVLH.

\funcitem{initialize}
This method is a simple passthrough to the \textit{initialize} method of
\textit{RelativeDerivedState}.

\funcitem{update}
The \textit{update} method first does a standard conversion to rectilinear
LVLH. If \textit{lvlh\_type} is set to \textit{CircularCurvilinear},
then the method \textit{convert\_rect\_to\_circ} is called to perform the
conversion.

\end{enumerate}
\end{itemize}
