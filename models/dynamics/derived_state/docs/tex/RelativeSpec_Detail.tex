
%%%%%%%%%%%%%%%%%%%%%%%%%%%%%%%%%%%%%%%%%%%%%%%%%%%%%%%%%%%%%%%%%%%%%%%%%%%%%%%%%
%
% Purpose:  Detailed part of Product Spec for the Relative model
%
% 
%
%%%%%%%%%%%%%%%%%%%%%%%%%%%%%%%%%%%%%%%%%%%%%%%%%%%%%%%%%%%%%%%%%%%%%%%%%%%%%%%%

\section{Detailed Design}
\label{sec:relativedetail}
See the \href{file:refman.pdf}{Reference Manual}\cite{derivedstatebib:ReferenceManual} for a summary of member data and member methods for all classes.  

\subsection{Process Architecture}
The architecture for the \RelativeDesc\ is trivial, comprising methods that are operationally independent.

\subsection{Functional Design}
This section describes the functional operation of the methods in each class.


The \RelativeDesc\ contains only one class:

\begin{itemize}
\classitem{RelativeDerivedState}
\label{ref:RelativeDerivedState}
\textref{DerivedState}{ref:DerivedState}

This contains only the methods \textit{initialize} and \textit{update}:
\begin{enumerate}
\funcitem{initialize}
The initialization process comprises the following steps:
\begin{enumerate}
\item{} The generic DerivedState initialization routine is called to assign the \textit{subject} value, and establish the naming convention for the state identifier.  Note - this is based on the \textit{subject} name, not on the \textit{target} name.  The user may wish to consider this in determining which object to associate as target, and which to associate as subject.
\item{} Ensures that the \textit{subject\_frame} and \textit{target\_frame} are well-defined.
\item{} The registry of active bodies, maintained by the Dynamics Manager (see \href{file:\JEODHOME/models/dynamics/dyn_manager/docs/dyn_manager.pdf}{\em Dynamics Manager Documentation}~\cite{dynenv:DYNMANAGER}) is updated to ensure that the \textit{target\_frame} is being updated, if necessary.
\end{enumerate}

\funcitem{update}
The variable \textit{direction\_sense} can be set to one of two values:
\begin{enumerate}
\item{ComputeSubjectStateinTarget}
\item{ComputeTargetStateinSubject}
\end{enumerate}

The appropriate call is made to the appropriate reference frame's \textit{calculate\_relative\_state} method, using the other reference frame as an argument (e.g., for ComputeSubjectStateinTarget, we are looking for the state of \textit{subject} with respect to \textit{target}, so the subject-frame's method would be called, using the target frame as an argument).


\end{enumerate}
\end{itemize}



