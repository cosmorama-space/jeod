%%%%%%%%%%%%%%%%%%%%%%%%%%%%%%%%%%%%%%%%%%%%%%%%%%%%%%%%%%%%%%%%%%%%%%%%%%%%%%%%%
%
% Purpose:  Analysis part of User's Guide for the LVLH model
%
% 
%
%%%%%%%%%%%%%%%%%%%%%%%%%%%%%%%%%%%%%%%%%%%%%%%%%%%%%%%%%%%%%%%%%%%%%%%%%%%%%%%%

% \section{Analysis}
\label{sec:lvlhuseranalysis}

It must be reiterated that the \LVLHDesc\ does not provide a state for the vehicle with which it is associated, it provides a \textit{reference frame}.  The LvlhDerivedState does contain a member called \textit{lvlh\_frame}, which contains a member called \textit{state}, but this is the state of the reference frame with respect to the parent planet-centered inertial reference frame.  This is identical to the inertial state of the vehicle with respect to the same planet; it is \textbf{not} the LVLH state of the vehicle with respect to the same planet. 

\subsection{Identifying the \LVLHDesc}
If the LVLH reference frame has been included in the simulation, there will be an instance of \textit{LvlhDerivedState} located in the S\_define file.  This would typically be found in either the vehicle object, or a separate relative-state object.  There should be an accompanying call to an initialization routine, which takes a reference to the \textit{subject\_body} as one of its inputs, and an accompanying call to an update function.  The essential variable \textit{reference\_name} is defined elsewhere, often in the input file.

The LVLH reference frame, generated by the \LVLHDesc\ model, may also be used to represent the state of another vehicle.  In this case, there will be further inclusion of a RelativeDerivedState instance; see the \textref{RelativeDerivedState User's Guide}{sec:relativeuseranalysis} for details on the implementation of RelativeDerivedState.

Example:
\begin{verbatim}
sim_object{
dynamics/derived_state:    LvlhDerivedState example_of_lvlh_state;

(initialization) dynamics/derived_state:
example_of_rel_state_object.example_of_lvlh_state.initialize (
    Inout DynBody &      subject_body = vehicle_1.dyn_body,
    Inout DynManager &   dyn_manager  = manager_object.dyn_manager);
    
{environment} dynamics/derived_state:
example_of_rel_state_object.example_of_lvlh_state.update ( )

} example_of_rel_state_object;
\end{verbatim}

Then the input file may have an entry comparable to:
\begin{verbatim}
example_of_rel_state_object.example_of_lvlh_state.reference_name = "Earth";
\end{verbatim}


\subsection{Editing the \LVLHDesc}
The planetary identification (\textit{reference\_name}) is open to edit by the analyst.

\subsection{Output Data}
The direct output from the \LVLHDesc\ comprises only the inertial state of the origin of the LVLH reference frame.  This is typically not useful.  The strength of the \LVLHDesc\ comes in its applicability to a RelativeDerivedState; see the \textref{RelativeDerivedState User's Guide}{sec:relativeuseranalysis} for details on the implementation of RelativeDerivedState.

