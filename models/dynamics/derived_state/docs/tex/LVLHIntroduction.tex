%%%%%%%%%%%%%%%%%%%%%%%%%%%%%%%%%%%%%%%%%%%%%%%%%%%%%%%%%%%%%%%%%%%%%%%%%%%%%%%%%
%
% Purpose:  Introduction for the LVLH model.
%
% 
%
%%%%%%%%%%%%%%%%%%%%%%%%%%%%%%%%%%%%%%%%%%%%%%%%%%%%%%%%%%%%%%%%%%%%%%%%%%%%%%%%


%\section{Purpose and Objectives of \LVLHDesc}
% Incorporate the intro paragraph that used to begin this Chapter here. 
% This is location of the true introduction where you explain what this model 
% does.

The \LVLHDesc\ is used to express the state of a vehicle in terms of the local vertical (opposing the unit radial vector from the defined planetary reference frame origin - usually the planet about which it is orbiting), and two vectors in the local horizontal plane; the first is defined as the unit orbital angular momentum vector (so must be perpendicular to the local vertical), the second completes the right-hand coordinate system.

While several of the Derived State values simply describe the state of a vehicle in some other defined reference frame, the LVLH Derived State defines a new reference frame itself.  Furthermore, since the reference frame is defined \textit{by} the vehicle, it has no state in this frame.  Instead, the LVLH frame, defined by the LVLH Derived State, can be used to express the state of another vehicle.

The LVLH reference frame, created by this model, is a recti-linear frame, so for two vehicles at the same altitude, the state of one in the LVLH frame of another will exhibit a positive vertical (i.e., down) component. See Figure~\vref{fig:nedrectilinear} for an illustration.













