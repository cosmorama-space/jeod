%%%%%%%%%%%%%%%%%%%%%%%%%%%%%%%%%%%%%%%%%%%%%%%%%%%%%%%%%%%%%%%%%%%%%%%%%%%%%%%%%
%
% Purpose:  Introduction for the OrbElem model.
%
% 
%
%%%%%%%%%%%%%%%%%%%%%%%%%%%%%%%%%%%%%%%%%%%%%%%%%%%%%%%%%%%%%%%%%%%%%%%%%%%%%%%%


%\section{Purpose and Objectives of \OrbElemDesc}
% Incorporate the intro paragraph that used to begin this Chapter here. 
% This is location of the true introduction where you explain what this model 
% does.

The \OrbElemDesc\ is used to express the state of a vehicle with respect to the planet about which it is orbiting in terms of its orbital elements:
\begin{itemize}
 \item {Semi-major axis}
\item {Semiparameter, or semi-latus rectum}
\item {Eccentricity}
\item {Inclination}
\item {Argument of periapsis}
\item {Longitude of ascending node}
\item {Anomaly (true, mean, and either eccentric, hyperbolic, or parabolic)}
\item {Mean motion}
\item {Specific orbital energy}
\item {Specific orbital angular momentum}
\end{itemize}













