\boilerplatechapterone{

The \ModelDesc\ maintains the mass properties of all non-planetary massive
objects in the simulation.  Mass-Body objects that also have a dynamic state are
represented as Dyn-Body ~\cite{dynenv:DYNBODY} objects, 
Dyn-Body being an inheritance from Mass-Body (i.e. a DynBody object is a
MassBody object, but with additional descriptors).
Consequently, the most frequently and obviously encountered Mass-Body objects
are the Dyn-Body-based simulation vehicle(s); this model accurately maintains 
the mass
properties of such entities throughout processes such as mass depletion due to
fuel usage, the attaching and detaching of one Mass Body to/from another, and
the relative motion of individual massive components of a composite body
comprising multiple Mass Body elements (e.g. rotation of a solar panel on a 
vehicle).

When the dynamic state of a composite vehicle (one comprising multiple
components) is integrated, only one integration is performed for the entire
composite unit; the component parts are not integrated directly.
It is the \ModelDesc that is responsible for computing the overall mass and 
inertia
of the composite body that will be used in determining the dynamic response to
applied forces and torques.

While the most obviously encountered Mass Body objects happen to be Dyn Body
objects, the separation of mass properties from dynamic properties allows for a
natural delineation between those objects for which dynamic state is
important, and those for which it is not.  When only the mass is important, a
Mass Body is used; a Dyn Body is used when the object must also have a dynamic
state.

For example, a user may choose to add
a massive fuel tank to a vehicle with no intention of ever wanting to know how
anything about the tank's position or velocity.
Indeed, unless the tank is ejected at some point,
then even if its dynamic state is desired, it can be obtained from the vehicle
itself; maintaining its own state would be redundant.

It is therefore quite a natural development to build a dynamic body from a
collection of non-dynamic massive bodies, each of which has its own unique (and
possibly time-varying) mass properties.  Keeping track of varying mass
properties within a compound body is a non-trivial undertaking, so this model
performs those tasks behind the scenes.
}{\ModelHistory}