\subsection{The Basic MassBody}
The \ModelDesc\ must maintain an up-to-date record of the mass and inertia 
tensor (the mass properties) of massive objects.  The inertia tensor 
is referenced to a coordinate system with origin at 
the center of mass and set of axes, the \textit{body-axes}, oriented as 
desired.
Because a stand-alone massive block
is neither very interesting nor useful, additional characteristics are 
incorporated to enhance the Mass Body, while maintaining its fundamental
purpose.

\subsection{Adding Dynamic State Information}
First, it is often desirable, though not essential, to monitor and propagate 
the
dynamic state of a massive object.  This feature is provided through
inheritance,
whereby a MassBody object can also be a DynBody object~\cite{dynenv:DYNBODY},
with the dynamic state features incorporated through the sub-class additions.
Once again though, having a dynamically active massive object that just moves 
subject to central forces is not particularly interesting or useful.

JEOD must be able to realistically propagate the response of a vehicle to 
external forces and torques. The rotational response to an applied torque, and 
the translational response to an applied force are both easy, but the 
rotational response to an applied force requires knowledge of the relative 
position vector between the center of mass of the object and the point at 
which 
the force is applied.  Conventionally, locations of objects within a vehicle 
are defined in the \textit{structural} reference frame.  Consequently, a 
MassBody also defines a point
(a MassBasicPoint, see \textref{Points of Interest}{sec:MassPoint}) called the 
\textit{structure\_point}, relative to which all 
positions can be defined or determined.  A new coordinate system, with origin 
at the structure-point and with a set of axes called the 
\textit{structural-axes}, is used 
for enumeration of the position vectors.  The structural-axes and body-axes 
are oriented and positioned relative to one another.

\paragraph {Aside on Structure-Body Orientation}
Typically, this relative orientation is straightforward, such as by 
having the axes aligned, or with a 180-degree yaw, or 90-degree pitch, or 
similar. See \textref{Composite and Core Properties}{sec:core_comp_prop} for 
more details.

\paragraph {Aside on Axes and Reference Frames}
The two coordinate systems now defined are both further developed into 
full-up reference frames -- with states all of their own -- in the 
DynBody~\cite{dynenv:DYNBODY} class.
There, the structure-axes and body-axes become known as the structural and 
body reference frames (respectively), and retain their orientations and their 
origins at the structure-point and center of mass (respectively).

\subsection{Mass-Mass Attachments}
Even with a dynamic state, the object is still very limited in its 
versatility. 
A very significant additional feature, which by itself adds sufficient 
complexity to justify the need for a separate model just to maintain the mass 
properties, is the ability to connect 
two (or more) massive objects to make compound objects.  When each of these 
massive objects can be attaching or detaching to/from the compound object, or 
changing its mass properties, or 
moving with respect to the other mass entities in the compound object, then 
the 
maintenance of the compound mass properties becomes non-trivial and 
necessitates a dedicated model for proper treatment.

To handle multiple massive objects connected together, we use a tree 
structure, refered to as the \textit{mass-tree}.
Every MassBody is in one -- and only one -- tree; some trees are ``atomic'', 
having only one member, 
and some are more complex.  Each tree represents the elemental massive objects 
that are physically connected together to make a compound massive object.  
There may be more than one tree in any simulation, it is not necessary that 
all MassBody objects be connected.
 
At the base (or top, depending on how it is visualized) of the tree is the 
\textit{root} body, 
which (unless it is atomic) has one or more \textit{children}.  Each child has 
one (exactly one) \textit{parent}, and possibly one or more children of its 
own.  There are two important restrictions on how these objects are positioned 
in the tree:
\begin{itemize}
 \item Because the dynamic state is propagated from parent to child, a dynamic 
 MassBody (i.e. a DynBody) can never be a child of a non-dynamic MassBody 
 (i.e. 
 a base-class MassBody).
 \item There can be no circular attachments (e.g. a situation in which A is a 
 child of B, which is a child of C, which is a child of A, is forbidden.)
\end{itemize}
An important consequence is that the mass tree cannot always 
reflect reality.  It is entirely feasible that two objects directly connected 
on the mass tree have no physical connection; the mass tree is purely a 
mathematical construct and should never be relied upon as an illustration of 
physical connectivity.

Recall that each MassBody has a coordinate system associated with its 
structure-point; when two MassBody objects are joined, those coordinate 
systems should be linked.  Therefore, each 
structure-point is given an orientation and position 
specification that identifies its relative state with respect to its parent's 
structure-point axes.  The orientation is represented as a quaternion and as a 
transformation matrix.

Note that the root MassBody objects have no parent, thus it would correctly be 
inferred that they have no defined orientation or position.  However, realize 
that if a base-class 
MassBody (as opposed to a DynBody) were at the root of the tree, then the 
entire tree 
must comprise base-class MassBody objects 
(as opposed to DynBody objects) since DynBody 
objects cannot be children of MassBody objects.  Thus, the entire tree has 
no state, thus absolute orientation and position are not defined, and all that 
is necessary 
is relative orientation and position of the elemental MassBody objects with 
respect to one 
another.  When (as is typical) the root is actually a DynBody, the structural 
reference frame that is developed from the structure-point is given a parent, 
thus the orientation and position of the structural reference frame, and thus 
of the structure-point, are defined.

\subsection{Core and Composite Properties}\label{sec:core_comp_prop}
Consider a compound object, comprising multiple MassBody objects.  The 
\ModelDesc must provide the mass properties of this single entity, but it must 
also retain the mass properties of the individual components, and the mass 
properties of all sub-trees:
\begin{itemize}
 \item Mass-loss would be applied to elemental components, and then 
 incorporated 
 into the compound object, so elemental properties must be retained so that 
 they can be manipulated.
 \item Motion or detachment of elemental components -- or of a section of the 
 mass tree -- 
 requires that the mass properties of that moving entity be known so that 
 the mass and inertia can be subtracted out (and added back in the case of 
 motion).
\end{itemize}
Thus, every parent object in the tree must, at all times, keep information on 
itself, and on that part of the tree of which it is the head (i.e. itself, and 
all of its children, children of children, etc.).  Thus, we keep two sets of 
mass properties associated with each MassBody:
\begin{enumerate}
 \item \textbf{Core Properties} are those properties associated with the 
 center of mass of the elemental body.
 \item \textbf{Composite Properties} are those properties associated with the 
 center of mass of that part of the mass tree of which it is the head.
\end{enumerate}

To simplify the determination of whether a MassBody needs a set of composite 
properties, we just provide this capacity to all MassBody objects; in the case 
of a body without children, the core properties and composite properties are 
equivalent.

With each set of properties comes a different center of mass, and consequently 
a different set of body-axes.  A Mass Point is automatically created for each 
center of mass, thereby setting 
the relative position of each set of body-axes relative to the structural 
axes, and the relative orientation of each of 
the body-axes sets with respect to the structural-axes.  Note that both sets 
of body-axes have the same orientation.

 \paragraph{Aside on locations of variables}
The relative position is stored as the respective properties' position (e.g. 
\textit{composite\_properties.position}). The relative orientation of each of 
the body-axes sets with respect to the structural-axes is specified as a 
transformation matrix or quaternion-set (e.g. 
\textit{core\_properties.T\_parent\_this}).  The rationale for this choice is 
expounded in \textref{Detailed Design}{sec:detailed_design}.

\paragraph{Aside for DynBody extension}

In the DynBody extension, there are three distinct reference frames - the 
core-body frame, the composite-body frame, and the structural frame.  The 
structural frame has its origin at the structure-point, while the body frames 
have their origin at the respective centers of mass.  Therefore, the position 
of a body frame with respect to the structural frame in a DynBody is equal to 
the respective properties' position value (e.g. 
\textit{core\_properties.position}).  By analogy, the orientation of the body 
frames with respect to the structural frame is the same as that of the 
body-axes with respect to the structural-axes. For example, the transformation 
from the structural frame to the composite-body frame is 
\textit{composite\_properties.T\_parent\_this}.

\subsection{Points of Interest}\label{sec:MassPoint}
There are numerous instances where some point on a body is of particular 
interest, and its position needs to be well defined - the location of an 
antenna, or a sensor, the point at which another object is attached, etc.  In 
many applications, it is also desirable to specify an axes-set at the point so 
that other positions can be defined with respect to the point.  Consequently, 
we include an orientation in the defining data of each point.

We have two types of points - the basic \textit{MassBasicPoint}, and the more 
commonly used \textit{MassPoint} (which is, essentially, a MassBasicPoint with 
a name; the name provides the user with a way to identify the mass point).

Every MassBody starts with three MassBasicPoints, and we have already 
considered them:
\begin{enumerate}
  \item The structure-point; axes are the structure-axes.
  \item The composite-properties point; axes are the body-axes with origin at 
  the composite center of mass.
  \item The core-properties point; axes are the body-axes with origin at the 
  core center of mass.
\end{enumerate}

Any MassBody can then be given any number of additional, user-specified, 
MassPoint instances.  Every MassPoint gets added to the mass-tree as a child 
of the structure-point of the same MassBody (obviously, excepting the 
structure-point itself, which is a child of the structure-point of the parent 
MassBody, and thereby a sibling to other MassPoint instances of the parent 
MassBody).


\subsection {Summary}
A MassBody comprises three components:
\begin{enumerate}
 \item A MassBasicPoint, called the \textit{structure\_point} that provides:
 \begin{enumerate}
  \item A position and orientation relative to the parent body (if it exists)
  \item A reference point and axes (\textit{structural-axes}), from which 
  further measurements may be made (such as to define the location of the 
  center of mass).
 \end{enumerate}
 \item A MassProperties object, \textit{core\_properties}, that provides:
   \begin{enumerate}
   \item Mass
   \item Inertia tensor, referenced to a specified axes-set, the 
   \textit{body-axes}.
   \item Position of the center of mass with respect to its structure-point, 
   expressed in the structural-axes.
   \item The orientation of the body-axes with respect to the 
   structural-axes.
   \end{enumerate}
   for the stand-alone MassBody.
  \item An additional MassProperties object, \textit{composite\_properties}, 
  that provides the same set of properties for the compound object comprising 
  the MassBody and everything attached \textbf{to} it in the mass tree (not 
  including objects that it is attached to, the hierarchy in the mass tree is 
  very important).
\end{enumerate}