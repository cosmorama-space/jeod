\subsection{Defining the Mass Body}
The following values for each MassBody are typically specified in the input
file:
 \begin{enumerate}
   \item Mass
   \item Position of the center of mass with respect to its structural-axes
   (often also known as the structural frame or structure frame)
   \item The orientation of the body-axes with respect to the
   structural-axes
   \item Inertia tensor.
   \end{enumerate}

This specification is usually performed through the use of the MassBodyInit
Body Action (this is strongly recommended).  The Body Action
document~\cite{dynenv:BODYACTION} has its own User Guide specifically for the
MassBodyInit action, duplicating some of the examples presented here.

In the following example we have a simulation object called
\textit{sim\_object\_A}, which contains a MassBody called \textit{body}, and
an instance of the Body Action MassBodyInit called \textit{mass\_init}.

Example code is given for Trick10 users; non-Trick users should
follow a similar pattern using their appropriate syntax.  We start with the
trivial declarations that tell \textit{mass\_init} that it is initializing
body, that the body has a mass of 1.0 $kg$, and specifies the position of the
center of mass of the body with respect to its structure-point.

\paragraph{Trick 10}
\begin{verbatim}
sim_object_A.mass_init.set_subject_body(sim_object_A.body);
sim_object_A.mass_init.properties.position = trick.attach_units("M",[1.0, 0.0,
0.0])
sim_object_A.mass_init.properties.mass     = trick.attach_units("kg",1.0)
\end{verbatim}

The orientation of the body-axes with respect to the structure-axes must also
be specified.  Note that this orientation will be applied equally to the
composite\_properties body-axes and to the core\_properties body-axes.

There are several methods by which the orientation can be specified, as
detailed in the Orientation Model document~\cite{dynenv:ORIENTATION}.  There
are also two ways to interpret the specified rotation -- either as body to
structure, or as structure to body (if none is specified, the default value is
structure-to-body).

The following values are all equivalent, and all mean structure-to-body:
\begin{itemize}
 \item 0
 \item StructToBody
 \item StructToCase
 \item StructToPoint
 \item StructToChild
\end{itemize}

The following values are all equivalent, and all mean body-to-structure:
\begin{itemize}
 \item 1
 \item BodyToStruct
 \item CaseToStruct
 \item PointToStruct
 \item ChildToStruct
\end{itemize}

\paragraph{Aside on Naming}
While the names are functionally equivalent, they have specific meanings,
hence the number of equivalent options:
\begin{enumerate}
\item \textit{Struct} refers to the structure-point.
\item \textit{Body} typically refers to the body-axes associated with the
MassBasicPoints \textit{core\_properties} and \textit{composite\_properties}.
\item \textit{Case} is typically used to represent the case frame of a sensor,
or similar, located at a MassPoint.
\item \textit{Point} is a generic term, used for any mass point.
\item \textit{Child} is also a generic reference that recognizes the mass-tree
structure, wherein all mass points for a MassBody are children of the
structure-point (obviously, excepting the structure-point itself).
\end{enumerate}


In the example below, we specify structure-to-body, and use eigen-vector
rotation of 90 degrees about the z-axis (this would be the z-axis in the
structure-axes)

\paragraph{Trick 07}

  \begin{verbatim}
sim_object_A.mass_init.properties.pt_frame_spec =
MassPointInit::StructToBody;
sim_object_A.mass_init.properties.pt_orientation.data_source =
                                        Orientation::InputEigenRotation;
sim_object_A.mass_init.properties.pt_orientation.eigen_angle {d} = 90.0;
sim_object_A.mass_init.properties.pt_orientation.eigen_axis[0]   = 0.0, 0.0,
1.0;
\end{verbatim}

This example, in Trick10 gives a transformation matrix from body-axes to
structure-axes, with a 180-degree rotation about the y-axis.
\paragraph{Trick 10}

  \begin{verbatim}
sim_object_A.mass_init.properties.pt_frame_spec =
                                         trick.MassPointInit.BodyToStruct;
sim_object_A.mass_init.properties.pt_orientation.data_source =
                                                 trick.Orientation.InputMatrix;
sim_object_A.mass_init.properties.pt_orientation.trans[0] = [ -1.0,  0.0,  0.0]
sim_object_A.mass_init.properties.pt_orientation.trans[1] = [  0.0,  1.0,  0.0]
sim_object_A.mass_init.properties.pt_orientation.trans[2] = [  0.0,  0.0, -1.0]
\end{verbatim}

Note that the orientation of the body with respect to any other object has not
been specified.  That is only relevant under one of the following situations:
\begin{enumerate}
 \item The body is attached to another body; in this case the relative
 orientation between the two bodies will be specified as part of the
 attachment definition.
 \item The absolute orientation of the body is useful in some sense; in this
 case, the body needs a state, and so should be a DynBody.  The initialization
 of a DynBody does include orientation of its reference frames (which a
 MassBaody does not have) with respect to their parent (which is the inertial
 reference frame).  By orienting the DynBody reference frames, the MassBody
 axes (which are aligned with the respective DynBody reference frames) also
 get aligned.
\end{enumerate}

The final initialization item is the inertia tensor, and there are several
methods for specifying this value.  The variable
\textit{mass\_init.properties.inertia\_spec} can take one of the following
values:
\begin{itemize}
 \item NoSpec         The inertia tensor is not specified.  Any attempt to
 specify it in the \textit{mass\_init} object will be ignored.
 \item Body (default) The inertia tensor is specified using the body-axes.
 \item StructCG       The inertia tensor is specified using axes oriented with
 the structure-axes, but based at the center-of-mass.
 \item Struct  The inertia tensor is specified using the structure-axes.
 \item SpecCG  The inertia tensor is specified using axes oriented with some
 specified orientation, based at the center-of-mass.
 \item Spec    The inertia tensor is specified using axes oriented with some
 specified orientation, based at a specified origin.
\end{itemize}

For options \textit{Spec} and \textit{SpecCG} ONLY, the value
\textit{mass\_init.properties.inertia\_orientation} must be specified.  This
is an instance of the Orientation class, see the Orientation document for full
details~\cite{dynenv:ORIENTATION}.  The value specified is the process by which
the orientation of the axes in which the inertia tensor is specified may be
transformed into the orientation of the body-axes (e.g. $T_{spec \rightarrow
body}$.  It is analogous to the specification of the body-axes orientation
relative to the structural-axes orientation.

For option \textit{Spec} ONLY, the value \textit{mass\_init.inertia\_offset}
must also be specified.  This is a simple 3-vector, expressed in the axes
associated with the \textit{mass\_init.properties.inertia\_orientation}
specification.

\paragraph{Trick07}
\begin{verbatim}
sim_object_A.mass_init.properties.inertia[0][0] {kg*M2} = 1.0, 0.0, 0.0;
sim_object_A.mass_init.properties.inertia[1][0] {kg*M2} = 0.0, 2.0, 0.0;
sim_object_A.mass_init.properties.inertia[2][0] {kg*M2} = 0.0, 0.0, 3.0;
\end{verbatim}

In this example, the axes are switched to specify the inertia on the
structure-axes.
\paragraph{Trick10}
\begin{verbatim}
sim_object_A.mass_init.properties.inertia_spec =
trick.MassPropertiesInit.Struct
sim_object_A.mass_init.properties.inertia[0]  =
                               trick.attach_units( "kg*m2",[ 1.0,  0.0,   0.0])
sim_object_A.mass_init.properties.inertia[1]  =
                               trick.attach_units( "kg*m2",[ 0.0,  2.0,   0.0])
sim_object_A.mass_init.properties.inertia[2]  =
                               trick.attach_units( "kg*m2",[ 0.0,  0.0,   3.0])

\end{verbatim}


\subsection{Adding Mass Points}
Mass points are useful for providing additional sets of axes from which
positions and orientations can be determined.  Perhaps the most useful
application of MassPoint objects is in attaching mass bodies to one another.
Mass points are typically defined using the same body action item, and at the
same time that the rest of the MassBody is being defined.  The following items
define a MassPoint:
 \begin{enumerate}
   \item Name
   \item Position with respect to the structural-axes
   \item The orientation of the point-axes with respect to the
   structural-axes.
   \end{enumerate}

We have already covered the initialization of both of these elements for the
MassBody, and the same system is followed for the MassPoint.  The position is
straightforward again, and the orientation can be specified in any one of the
many ways available from the Orientation model~\cite{dynenv:ORIENTATION}, and
interpreted as either structure-to-body or body-to-structure, just as outlined
above.

The MassPoint instances do need allocation.  In the following examples, notice
that the \textit{pt\_frame\_spec} value is not specified, so takes the default
value \textit{StructToPoint}.  Notice also that I have chosen to illustrate
two different ways of specifying the orientation in the two examples; this
does not mean that one is prefered in Trick07 and one in Trick10, they are
jsut illustrative examples.

\paragraph{Trick07}
\begin{verbatim}
sim_object_A.mass_init.num_points = 1;
sim_object_A.mass_init.points = alloc( 1 );

sim_object_A.mass_init.points[0].set_name( "interesting_name" );
sim_object_A.mass_init.points[0].pt_frame_spec =
MassPointInit::StructToPoint;
sim_object_A.mass_init.points[0].position[0] {M}  = 2.0, 0.0, 0.0;
sim_object_A.mass_init.points[0].pt_orientation.data_source =
                                               Orientation::InputEigenRotation;
sim_object_A.mass_init.points[0].pt_orientation.eigen_angle {d} = 180.0;
sim_object_A.mass_init.points[0].pt_orientation.eigen_axis[0]   = 0.0, 0.0,
1.0;
\end{verbatim}

\paragraph{Trick10}
\begin{verbatim}
 sim_object_A.mass_init.num_points = 1
 sim_object_A.mass_init.points =
                            trick.sim_services.alloc_type( 1, "jeod::MassPointInit" )
 sim_object_A.mass_init.points[0].set_name( "interesting_name" )
 sim_object_A.mass_init.points[0].position  =
                                      trick.attach_units( "m",[ 2.0, 0.0, 0.0])
 sim_object_A.mass_init.points[0].pt_orientation.data_source =
                                                  trick.Orientation.InputMatrix
 sim_object_A.mass_init.points[0].pt_orientation.trans[0]  = [ -1.0,  0.0, 0.0]
 sim_object_A.mass_init.points[0].pt_orientation.trans[1]  = [  0.0, -1.0, 0.0]
 sim_object_A.mass_init.points[0].pt_orientation.trans[2]  = [  0.0,  0.0, 1.0]

\end{verbatim}


\subsection{Log Files}
The output from the \ModelDesc\ is generally simple information (mass, center
of
mass, inertia matrix).  It is possible to output this information for all of
the bodies in a simulation.

Logging of the composite-properties is more common than the core-properties,
because the core-properties are set by the user (so are presumably already
known), whereas the composite-properties are generated by the \ModelDesc.


\paragraph{Trick07}
\begin{verbatim}
  "sim_object_A.body.composite_properties.mass";
  "sim_object_A.body.composite_properties.position[0]";
  "sim_object_A.body.composite_properties.inertia[0][0-2]";
  "sim_object_A.body.composite_properties.inertia[1][0-2]";
  "sim_object_A.body.composite_properties.inertia[2][0-2]";
\end{verbatim}

\paragraph{Trick10}
\begin{verbatim}
  dr_group.add_variable("sim_object_A.body.composite_properties.mass")
  for ii in range(0,3) :
    dr_group.add_variable("sim_object_A.body.composite_properties.position["+st
    r(ii)+"]")
\end{verbatim}
etc.
