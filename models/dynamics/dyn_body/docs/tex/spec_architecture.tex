\subsection{Model Architecture}
The \ModelDesc is implemented in the form of C++ classes. The model classes
are described below.
\begin{description}
\item[\bf DynBody] The DynBody class is (or should be) an abstract class.
  This class provides all of the basic capabilities needed to represent a
  dynamic body except for the ability to integrate state over time. This
  exception is by design. The equations of motion vary with selection of which
  frame is to be integrated (composite body versus structural frame) and with
  assumptions regarding vehicle behavior dynamics.
\item[\bf BodyRefFrame] The BodyRefFrame class extends the RefFrame class.
  Member data include a pointer to a MassBasicPoint object and an indicator
  of which state elements have been set.
\item[\bf BodyForceCollect] The BodyForceCollect class contains
  \begin{inparaenum}[(1)]
  \item the six STL vectors that collect the effector, environmental, and
    non-transmitted forces and torques that act on a DynBody object,
  \item the vector sums of the collected forces and torques, categorized
    by type, and
  \item the total external force and torque acting on a DynBody object.
  \end{inparaenum} The collected forces and torques are assumed to be
  represented in the structural frame.
\item[\bf CollectForce] A CollectForce instance contains a pointer to a flag
  indicating whether the force is active and a pointer to a 3-vector that in
  turn contains the components of the force in the DynBody's structural frame.
  The CollectForce class forms part of the interface
  between JEOD and the Trick \emph{vcollect} mechanism.
  The overloaded {\tt CollectForce::create()} method allocates and constructs a
  CollectForce object; exactly how depends on the argument to {\tt create()}.
  The Trick \emph{vcollect} depends on this Factory Method architecture.
  The STL force vectors in a BodyForceCollect instance that contain the
  collected effector, environmental, and non-transmitted forces contain
  pointers to CollectForce instances, each of which is constructed via
  {\tt create()}.
\item[\bf Force] A Force contains a flag indicating whether the force is active
  and contains a 3-vector that in turn contains the components of the force as
  represented in a DynBody's structural frame. A CollectForce instance can be
  easily and safely constructed from a Force instance. The Force class is the
  recommended approach for representing forces in JEOD.
\item[\bf CInterfaceForce] The class CInterfaceForce extends the CollectForce
  class. The intent is to provide the ability to create CollectForce instances
  from forces that are not represented using the Force class. This capability is
  essential for using JEOD with C-based force models. This capability is not
  particularly safe. A CInterfaceForce can be created from any double pointer.
  The burden falls solely upon the user to use this capability properly.
\item[{\bf CollectTorque}, {\bf Torque},and {\bf CInterfaceTorque}] These
  classes are the torque analogs of the CollectForce, Force, and CInterfaceForce
  classes.
\item[\bf FrameDerivs] The FrameDerivs class contains the second derivatives
  needed for state integration.
\item[\bf DynBodyMessages] The DynBodyMessages class defines the message
  types used in conjunction with the \hypermodelref{MESSAGE}. The
  DynBodyMessages is not an instantiable class.
\end{description}
