\test{Force/Torque}\label{test:force_torque}
\begin{description}
\item[Background] The primary purpose of this test is to test that
external forces and torques are properly accumulated and propagated
to parent bodies.

\item[Test description]
This test applies forces and torques to a pair of joined dynamic bodies
flying through otherwise empty space.
The situation is contrived to test that forces and torques, including
the non-transmitted forces and torques, are properly accumulated
and propagated. The bodies take a jaunt along the inertial z axis,
return to the origin, then another jaunt along the inertial x axis
and return once again to the origin. The bodies perform a 90 degree yaw and then
a 90 degree roll during the latter excursion.

\item[Test directory]
{\tt models/dynamics/dyn\_body/verif/SIM\_force\_torque} \\
The simulation \Sdefine file defines two dynamic bodies,
body actions that operate on these bodies,
and forces and torques that can be applied to these bodies.
The forces and torques are collected via the {\emph vcollect} mechanism.
There is one run for this test.
Table~\ref{tab:force_torque_summary} summarizes the actions, forces and
torques that are applied during the course of the run and
identifies the expected results from those operations.

\item[Success criteria] The analytic and simulated results
for the accumulated/propagated forces and torques,
and for the resulting accelerations,
must agree to within numerical precision. Deviations between
the analytic and simulated integrated states must be small and
attributable solely to the small numeric errors in the accelerations.

\item[Test results] The final position of the composite center of mass
is within $1.7\eneg{12}$ meters of the origin. This small error
is fully attributable to the integration of small errors. Along the way,
small errors are made in the accumulation and propagation of forces and
torques and calculations of accelerations. These small errors are purely
numerical.

The test passes.

\item[Applicable requirements]
This test demonstrates the partial or complete satisfaction of the
following requirements:
\begin{itemize}
\item \ref{reqt:mass_requirements}. Mass properties are properly initialized and
are properly updated due to attachments and detachments.
\item \ref{reqt:eom}. Forces and torques are properly accumulated and
propagated. Accelerations are properly calculated from forces and torques.
\item \ref{reqt:state_integ_prop}. The equations of motion are properly
integrated.
\item \ref{reqt:vehicle_points}. Vehicle points are properly constructed
and used during attachments.
\item \ref{reqt:attach_detach}. JEOD's magical teleportation capability works as
advertised.
\end{itemize}

\end{description}

\begin{table}[htp]
\centering
\caption{Force/Torque Test Summary}
\label{tab:force_torque_summary}
\vspace{1ex}
\begin{tabular}{||r@{}l|p{2.6in}|p{3in}|} \hline
\multicolumn{2}{||l|}{\bf Time} & {\bf Action} & {\bf Expected Result}
\\ \hline \hline
&0 &
  Bodies initialized and attached &
  \raggedright
  Composite body has spherical mass distribution and
  is at rest at the inertial origin.\tabularnewline[6pt]  
0&-3 &
  \raggedright
  Effector \& environmental z-axis forces increasingly applied to body2 &
  \raggedright
  Composite acceleration builds from 0 to 3~m/s$^2\,\hat z$
  in steps of 1 m/s$^2$.\tabularnewline[6pt]
4&-6 &
  Non-xmitted forces on body2 &
  Composite acceleration remains at 3 m/s$^2\hat z$.\tabularnewline[6pt]
&7 &
  Body2 detaches from body1 &
  Bodies are detached.\tabularnewline[6pt]
7&-9 &
  Non-xmitted z-axis force on body2 &
  \raggedright
  Body2 zooms back toward the origin,
  coming to rest with respect to body1 at t=9.\tabularnewline[6pt]
&9& Body2 reattaches to body1 &
  Attachment does not change body1 core state.\tabularnewline[6pt]
9&-19 &
  Non-transmitted z-axis forces on body1 &
  Composite acceleration is -3 m/s$^2\,\hat z$.\tabularnewline[6pt]
21&-27 &
  Non-transmitted z-axis forces on body1 &
  \raggedright
  Composite body moves along z toward origin,
  coming to rest at the origin at t=27.\tabularnewline[6pt]
30&-32 &
  Environmental x-axis force on body2,
  non-xmitted y-axis torque on body1 &
  \raggedright \ \\
  Composite acceleration is 1 m/s$^2\,\hat x$.\tabularnewline[6pt]
32&-52 &
  Environmental z-axis torque on body2, reversing sign at t=42 &
  \raggedright
  Composite body rotates about its z-axis,
   completing a 90 degree yaw at t=52.\tabularnewline[6pt]
52&-56 &
  Environmental y-axis force on body2,
  non-transmitted x-axis torque on body1 &
  \raggedright
  Composite acceleration is -1~m/s$^2\,\hat x$,
  making composite move back toward origin.\tabularnewline[6pt]
56&-76 &
  Environmental x-axis torque on body2, reversing sign at t=66 &
  \raggedright
  Composite body rotates about its x-axis,
   completing a 90 degree roll at t=76.\tabularnewline[6pt]
76&-78 &
  Non-transmitted z-axis force on body1 &
  Composite acceleration is 1 m/s$^2\,\hat x$.\tabularnewline[6pt]
&78 &&
  Composite body frame is at rest at the origin.\tabularnewline[6pt]
\hline
\end{tabular}
\end{table}
